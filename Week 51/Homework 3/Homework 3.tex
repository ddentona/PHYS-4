\documentclass[12pt]{article}
\usepackage{amsmath}
\usepackage{amssymb}
\usepackage{amsthm}
\usepackage{cancel}
\usepackage{enumitem}
\usepackage{esdiff}
\usepackage{graphicx}
\usepackage{siunitx}
% \usepackage{pgfplots}
\usepackage{wrapfig}

\newcommand{\e}[1]{e^{i\left(#1\right)}}
\newcommand{\E}[1]{\times 10^{#1}}

\renewcommand\qedsymbol{TENA}

\title{
    Homework \#3
    \\  \small
    PHYS 4D: Modern Physics
    }
\author{Donald Aingworth IV}
\date{January 26, 2026}

\begin{document}
    \maketitle

    \section{Question 1}
        How are the threshold frequency $f_0$ and wavelength $\lambda_0$ for the photoelectric effect related to the work function $W$ of a metal?

        \subsection{Solution}

    \section{Question 2}
        Write down an equation showing how the maximum kinetic energy of photoelectrons depends upon the wavelength of the incident light and the work function of the metal.

        \subsection{Solution}

    \section{Question 3}
        What would be the work function of a metal with threshold wavelength of 400 nm?

        \subsection{Solution}

    \section{Question 4}
        Would photoelectrons be emitted from a metal if the wavelength of the incident light were longer than the threshold wavelength?

        \subsection{Solution}

    \section{Question 6}
        What causes the dark lines in the solar spectra?

        \subsection{Solution}

    \section{Question 7}
        Suppose that an atom makes a transition from an energy level $E_2$ to an energy level $E_1$ emitting light. 
        How is the light's frequency f related to the difference of energy, $E_2 - E_1$ ?

        \subsection{Solution}

    \section{Question 8}
        Write down a formula for the energy levels of hydrogen.

        \subsection{Solution}

    \section{Question 11}
        What is the energy of the hydrogen atom in the $n = 4$ state?

        \subsection{Solution}

    \section{Question 12}
        What photon energy would be required to ionize a hydrogen atom in the $n = 5$ state?

        \subsection{Solution}

    \section{Question 15}
        What model of light can be used to describe the \textit{Compton effect}?

        \subsection{Solution}

    \section{Question 16}
        What is Bragg's law for X-ray diffraction?

        \subsection{Solution}

    \section{Question 17}
        How are the momentum and the wavelength of a particle related?

        \subsection{Solution}

    \section{Question 18}
        Express the momentum of an electron in terms of the angular wave vector $k$.

        \subsection{Solution}

    \section{Question 19}
        How would the de Broglie wavelength of an electron change if its velocity were to double?

        \subsection{Solution}

    \section{Question 20}
        Describe the interference pattern is produced when a beam of light having a single wavelength is incident upon two slits?

        \subsection{Solution}

    \section{Problem 1}
        Calculate the energy of the photons for light having a wavelength $\lambda = 200\,\unit{\nano\meter}$.

        \subsection{Solution}

    \section{Problem 2}
        Suppose that a 100 Watt beam of light is incident upon a metal surface. 
        If the light has a wavelength $\lambda = 200\,\unit{\nano\meter}$, how many photons strike the surface every second? 
        (Hint: Using the power of the beam, find how much energy is incident upon the surface every second.)

        \subsection{Solution}

    \section{Problem 4}
        What will the maximum kinetic energy of the emitted photoelectrons be when ultraviolet light having a wavelength of 200 nm is incident upon the following metal surfaces?
        \begin{enumerate}[label=\alph*)]
            \item   Na
            \item   Al
            \item   Ag
        \end{enumerate}

        \subsection{Solution}

    \section{Problem 6}
        Light having a wavelength of 460 nm is incident on a cathode, and electrons are emitted from the metal surface. 
        It is observed that the electrons may be prohibited from reaching the anode by applying a stopping potential of 0.72 eV.
        What is the work function of the metal in the cathode?
        \subsection{Solution}

    \section{Problem 7}
        By how much would the stopping potential in the previous problem increase if the wavelength of the radiation were reduced to 240 nm?

        \subsection{Solution}

    \section{Problem 11}
        An electron in the $n = 5$ state of hydrogen makes a transition to the $n = 2$ state. 
        What are the energy and wavelength of the emitted photon?

        \subsection{Solution}

    \section{Problem 16}
        Ultraviolet light of wavelength 45.0 nm is incident upon a collection of hydrogen atoms in the ground state. 
        Find the kinetic energy and the velocity of the emitted electrons.

        \subsection{Solution}

    \section{Problem 20}
        What potential difference must electrons be accelerated through to have the same wavelength as 40 keV X-rays?

        \subsection{Solution}

    \section{Problem 21}
        Electrons, protons, and neutrons have wavelengths of 0.01 nm. 
        Calculate their kinetic energies.

        \subsection{Solution}

    
\end{document}