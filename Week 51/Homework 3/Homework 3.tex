\documentclass[12pt]{article}
\usepackage{amsmath}
\usepackage{amssymb}
\usepackage{amsthm}
\usepackage{cancel}
\usepackage{enumitem}
\usepackage{esdiff}
\usepackage{graphicx}
\usepackage{siunitx}
\usepackage{ulem}
% \usepackage{pgfplots}
\usepackage{wrapfig}

\newcommand{\e}[1]{e^{i\left(#1\right)}}
\newcommand{\E}[1]{\times 10^{#1}}

\renewcommand\qedsymbol{TENA}

\title{
    Homework \#3
    \\  \small
    PHYS 4D: Modern Physics
    }
\author{Donald Aingworth IV}
\date{January 26, 2026}

\begin{document}
    \maketitle

    \section{Questions}

        \subsection{Question 1}
            How are the threshold frequency $f_0$ and wavelength $\lambda_0$ for the photoelectric effect related to the work function $W$ of a metal?

            \subsubsection{Solution}
                The work function of a metal is related to the threshold frequency and wavelength separately, since they are dependant on each other.
                \begin{gather}
                    W = hf_0 - (KE)_{\rm max}\\
                    W = \frac{hc}{\lambda_0}
                \end{gather}

        \subsection{Question 2}
            Write down an equation showing how the maximum kinetic energy of photoelectrons depends upon the wavelength of the incident light and the work function of the metal.

            \subsubsection{Solution}
                I'm taking this from the Basic Equations of this chapter.
                \begin{equation}
                    (KE)_{\rm max} = \frac{hc}{\lambda} - W
                \end{equation}

        \subsection{Question 3}
            What would be the work function of a metal with threshold wavelength of 400 nm?

            \subsubsection{Solution}
                \begin{align}
                    W   &=  \frac{hc}{\lambda_0}
                        =   \frac{1240\,\unit{\eV\cdot\nano\meter}}{400\,\unit{\nano\meter}}
                        =   \boxed{3.1\,\unit{\eV}}
                \end{align}

        \subsection{Question 4}
            Would photoelectrons be emitted from a metal if the wavelength of the incident light were longer than the threshold wavelength?

            \subsubsection{Solution}
                The minimum in light is set by the frequency.
                This means that the maximum is given in the wavelength.
                This means that if it were longer, no photoelectrons would be emitted.

        \subsection{Question 6}
            What causes the dark lines in the solar spectra?

            \subsubsection{Solution}
                These would be the colors of light waves that are absorbed by atoms somewhere between the sun and the earth.

        \subsection{Question 7}
            Suppose that an atom makes a transition from an energy level $E_2$ to an energy level $E_1$ emitting light. 
            How is the light's frequency $f$ related to the difference of energy, $E_2 - E_1$ ?

            \subsubsection{Solution}
                The frequency is directly proportional to the difference of energy.
                \begin{equation}
                    f = \frac{E_2 - E_1}{h}
                \end{equation}

        \subsection{Question 8}
            Write down a formula for the energy levels of hydrogen.

            \subsubsection{Solution}
                Let's use \sout{Ryback's} Rydberg's Equation.
                \begin{equation}
                    E = -R\frac{hc}{n} = -\frac{13.6\,\unit{\eV}}{n^2}
                \end{equation}

        \subsection{Question 11}
            What is the energy of the hydrogen atom in the $n = 4$ state?

            \subsubsection{Solution}
                Use Rydberg's equation.
                \begin{equation}
                    E_4 = -\frac{13.6\,\unit{\eV}}{4^2} = \boxed{-0.8375\,\unit{\eV}}
                \end{equation}

        \subsection{Question 12}
            What photon energy would be required to ionize a hydrogen atom in the $n = 5$ state?

            \subsubsection{Solution}
                Use Rydberg's equation.
                \begin{equation}
                    -E_5 = \frac{13.6\,\unit{\eV}}{5^2} = \boxed{0.544\,\unit{\eV}}
                \end{equation}

        \subsection{Question 15}
            What model of light can be used to describe the \textit{Compton effect}?

            \subsubsection{Solution}
                The particle model can be used to describe the Compton effect.

        \subsection{Question 16}
            What is Bragg's law for X-ray diffraction?

            \subsubsection{Solution}
                Bragg's law defines the requirements for constructive interference to occur between two X-rays of wavelength $\lambda$ hitting two lattices of a crystal a distance $d$ from each other at an angle $\theta$ with the perpendicular.
                It includes a coefficient $n \in \mathbb{N}$.
                \begin{equation}
                    2d\sin(\theta) = n\lambda
                \end{equation}

        \subsection{Question 17}
            How are the momentum and the wavelength of a particle related?

            \subsubsection{Solution}
                The momentum of a particle is defined by its energy, which is defined by its frequency, which is directly related to its wavelength.
                \begin{equation}
                    p   =   \frac{E}{c}
                        =   \frac{hf}{c}
                        =   \frac{h\cancel{c}}{\cancel{c}\lambda}
                        =   \frac{h}{\lambda}
                \end{equation}

        \subsection{Question 18}
            Express the momentum of an electron in terms of the angular wave vector $k$.

            \subsubsection{Solution}
                Use the equation from earlier and relate the wavelength and wave number.
                \begin{gather}
                    \lambda = \frac{2\pi}{k}; \hbar = \frac{h}{2\pi}\\
                    p = \frac{h}{\lambda} = \frac{hk}{2\pi} = \hbar k
                \end{gather}

        \subsection{Question 19}
            How would the de Broglie wavelength of an electron change if its velocity were to double?

            \subsubsection{Solution}
                If the electron's velocity doubles, its momentum doubles.
                Since wavelength is inversely related to momentum, a doubled momentum would lead to a \underline{halved} wavelength.

        \subsection{Question 20}
            Describe the interference pattern is produced when a beam of light having a single wavelength is incident upon two slits?

            \subsubsection{Solution}
                You would wind up with a pattern of light and dark spots (corresponding to where photons do and do not land).
                This is known as the two-slit experiment.

    \section{Problem 1}
        Calculate the energy of the photons for light having a wavelength $\lambda = 200\,\unit{\nano\meter}$.

        \subsection{Solution}
                Energy of a photon is the quotient between $hc$ (Planck's constant times the speed of light, $1240\,\unit{\eV\cdot\nano\meter}$) and the wavelength of the light.
                \begin{equation}
                    E   =   \frac{hc}{\lambda}
                        =   \frac{1240\,\unit{\eV\cdot\nano\meter}}{200\,\unit{\nano\meter}}
                        =   \boxed{6.2\,\unit{\eV}}
                \end{equation}

    \section{Problem 2}
        Suppose that a 100 Watt beam of light is incident upon a metal surface. 
        If the light has a wavelength $\lambda = 200\,\unit{\nano\meter}$, how many photons strike the surface every second? 
        (Hint: Using the power of the beam, find how much energy is incident upon the surface every second.)

        \subsection{Solution}
            First find the energy absorbed in one second, assuming all photons are absorbed by the metal surface (don't ask me why or how).
            \begin{gather}
                P   =   \frac{E}{t}\\
                E_{\rm bulb}    =   Pt 
                    =   100\,\unit{\watt} \times 1\,\unit{\second}
                    =   100\,\unit{\joule}
            \end{gather}

            Convert the energy of each photon from the previous problem to joules.
            \begin{equation}
                E_{\rm photon}  =   6.2\unit{\eV} \times \frac{1.6022\E{-19}\,\unit{\eV}}{1\,\unit{J}}
                    =   9.9335\E{-19}\,\unit{\joule}
            \end{equation}

            Lastly, divide the energy absorbed by the energy per photon to find the number of photons.
            \begin{align}
                n   &=  \frac{E_{\rm bulb}}{E_{\rm photon}}
                    =   \frac{100\,\unit{\joule}}{9.9335\E{-19}\,\unit{\joule}}
                    =   \boxed{1.007\E{20}}
            \end{align}
    \pagebreak

    \section{Problem 4}
        What will the maximum kinetic energy of the emitted photoelectrons be when ultraviolet light having a wavelength of 200 nm is incident upon the following metal surfaces?
        \begin{enumerate}[label=\alph*)]
            \item   Na
            \item   Al
            \item   Ag
        \end{enumerate}

        \subsection{Solution (a)}
            I will be using the same strategy for each of these.
            Take the work function value of the metal in question.
            \begin{equation}
                W_{\rm Na} = 2.28\,\unit{\eV}
            \end{equation}

            Plug this into the equation for the maximum kinetic energy.
            Also plug in the wavelength and the value of $hc$.
            Then compute that to find the answer.
            \begin{align}
                (KE)_{\rm max}  &=  \frac{hc}{\lambda} - W
                    =   \frac{1240\,\unit{\eV\cdot\nano\meter}}{200\,\unit{\nano\meter}} - 2.28\,\unit{\eV}\\
                    &=  6.2\,\unit{\eV} - 2.28\,\unit{\eV}
                    =   \boxed{3.92\,\unit{\eV}}
            \end{align}

        \subsection{Solution (b)}
            \begin{align}
                (KE)_{\rm max}  &=  \frac{hc}{\lambda} - W\\
                    &=  6.2\,\unit{\eV} - 4.08\,\unit{\eV}
                    =   \boxed{2.12\,\unit{\eV}}
            \end{align}

        \subsection{Solution (c)}
            \begin{align}
                (KE)_{\rm max}  &=  \frac{hc}{\lambda} - W\\
                    &=  6.2\,\unit{\eV} - 4.73\,\unit{\eV}
                    =   \boxed{1.47\,\unit{\eV}}
            \end{align}
    \pagebreak

    \section{Problem 6}
        Light having a wavelength of 460 nm is incident on a cathode, and electrons are emitted from the metal surface. 
        It is observed that the electrons may be prohibited from reaching the anode by applying a stopping potential of 0.72 eV.
        What is the work function of the metal in the cathode?

        \subsection{Solution}
            The maximum KE of electrons emitted is equal to the electron charge times the stopping potential.
            However, since our stopping potential value is in units of electron volts and not volts, we can assume this refers to the kinetic energy and not the stopping potential.
            \begin{equation}
                (KE)_{\rm max} = 0.72\,\unit{\eV}
            \end{equation}

            Plug this and the wavelength into the equation for maximum kinetic energy.
            Rounding in this problem will be to the closest hundredth of an electron-volt.
            \begin{gather}
                (KE)_{\rm max} = \frac{hc}{\lambda} - W\\
                \begin{align}
                    W   &=  \frac{hc}{\lambda} - (KE)_{\rm max}
                        =   \frac{1240\,\unit{\eV\,\nano\meter}}{460\,\unit{\nano\meter}} - 0.72\,\unit{\eV}\\
                        &=  2.70\,\unit{\eV} - 0.72\,\unit{\eV}
                        =   \boxed{1.98\,\unit{\eV}}
                \end{align}
            \end{gather}

    \section{Problem 7}
        By how much would the stopping potential in the previous problem increase if the wavelength of the radiation were reduced to 240 nm?

        \subsection{Solution}
            Let's calculate it. 
            Keep the calculated value of $W$. 
            \begin{align}
                (KE)_{\rm max}  &=  \frac{hc}{\lambda} - W
                    =   \frac{1240\,\unit{\eV\,\nano\meter}}{240\,\unit{\nano\meter}} - 1.98\,\unit{\eV}\\
                    &=  5.17\,\unit{\eV} - 1.98\,\unit{\eV}
                    =   \boxed{3.19\,\unit{\eV}}
            \end{align}
    \pagebreak

    \section{Problem 11}
        An electron in the $n = 5$ state of hydrogen makes a transition to the $n = 2$ state. 
        What are the energy and wavelength of the emitted photon?

        \subsection{Solution}
            Use Rydberg's equation.
            \begin{align}
                E_{\gamma}  &=  13.6\,\unit{\eV} \left( \frac{1}{n_2^2} - \frac{1}{n_1^2} \right)
                    =   13.6\,\unit{\eV} \left( \frac{1}{2^2} - \frac{1}{5^2} \right)
                    =   \frac{13.6\,\unit{\eV}}{4} - \frac{13.6\,\unit{\eV}}{25}\\
                    &=  3.4\,\unit{\eV} - 0.544\,\unit{\eV}
                    =   \boxed{2.9\,\unit{\eV}}
            \end{align}

    \section{Problem 16}
        Ultraviolet light of wavelength 45.0 nm is incident upon a collection of hydrogen atoms in the ground state. 
        Find the kinetic energy and the velocity of the emitted electrons.

        \subsection{Solution}
            Light will be able to transmit energy inversely relative to its wavelength.
            \begin{equation}
                E_{\rm photon} = \frac{hc}{\lambda} = \frac{1240\,\unit{\eV\,\nano\meter}}{45.0\,\unit{\nano\meter}} = 27.6\,\unit{\eV}
            \end{equation}

            This would be reduced by the energy necessary to ionize the atom.
            Said energy for reduction would be the energy from Rydberg's equation at ground state ($n = 1$).
            \begin{equation}
                KE = 27.6\,\unit{\eV} - 13.6\,\unit{\eV} = 14.0\,\unit{\eV}
            \end{equation}

            Before we move on, convert electron-volts to joules.
            \begin{equation}
                14.0\,\unit{\eV} = 2.236\E{-18}\,\unit{\joule}
            \end{equation}

            Multiply this by two, divide it by the mass of an electron, and take the square root to find the velocity (in no particular direction).
            \begin{gather}
                K   =   2.236\E{-18}\,\unit{\joule}\\
                2K  =   2 * 2.236\E{-18}\,\unit{\joule} = 4.472\E{-18}\,\unit{\joule}\\
                \frac{2K}{m_e} = \frac{4.472\E{-18}\,\unit{\joule}}{9.109\E{-31}\,\unit{\kilo\gram}} = 4.909\E{12}\,\unit{\meter^2/\second^2}\\
                \sqrt{\frac{2K}{m_e}} = \sqrt{4.909\E{12}\,\unit{\meter^2/\second^2}}
                    =   \boxed{2.216\E{6}\,\unit{\meter/\second}}
            \end{gather}

    \section{Problem 20}
        What potential difference must electrons be accelerated through to have the same wavelength as 40 keV X-rays?

        \subsection{Solution}
            First find the wavelength of the $40\,\unit{\kilo\eV}$ X-rays.
            \begin{equation}
                \lambda 
                    =   \frac{hc}{E} 
                    =   \frac{1240\,\unit{\eV\,\nano\meter}}{40\,\unit{\kilo\eV}}
                    =   0.031\,\unit{\nano\meter}
            \end{equation}

            Plug this into the de Broglie wavelength equation.
            \begin{equation}\label{eq:41}
                p   =   \frac{h}{\lambda}
                    =   \frac{6.626\E{-34}\,\unit{\joule\cdot\second}}{0.031\,\unit{\nano\meter}}
                    =   2.137\E{-23}\unit{\newton\cdot\second}
            \end{equation}

            Lastly, use this to find the necessary kinetic energy.
            \begin{equation}\label{eq:42}
                E   =   \frac{p^2}{2m_e}
                    =   \frac{\left( 2.137\E{-23}\unit{\newton\cdot\second} \right)^2}{2 \cdot 9.109\E{-31}\unit{\kilo\gram}}
                    =   2.507\E{-16}\,\unit{\joule}
            \end{equation}

            Divide this by the charge of the electron to find the necessary potential.
            \begin{equation}
                V   =   \frac{E}{q}
                    =   \frac{2.507\E{-16}\,\unit{\joule}}{1.602\E{-19}\,\unit{\coulomb}}
                    =   \boxed{1.565\,\unit{\kilo\volt}}
            \end{equation}
    \pagebreak

    \section{Problem 21}
        Electrons, protons, and neutrons have wavelengths of 0.01 nm. 
        Calculate their kinetic energies.

        \subsection{Solution}
            Combine (\ref{eq:41}) and (\ref{eq:42}).
            This will bring us an equation usable for each of these wave-particles.
            \begin{equation} \label{eq:44}
                E   =   \frac{h^2}{2 m \lambda^2}
                    =   \frac{(6.626\E{-34}\,\unit{\joule\cdot\second})^2}{2 m (0.01\,\unit{\nano\meter})^2}
                    =   \frac{2.195\E{-45}\unit{\joule\cdot\kilo\gram}}{m}
            \end{equation}

            \subsubsection{Electron}
                Use equation (\ref{eq:44}) with $m = 9.109\E{-31}\,\unit{\kilo\gram}$.
                \begin{equation}
                    E   =   \frac{2.195\E{-45}\unit{\joule\cdot\kilo\gram}}{9.109\E{-31}\,\unit{\kilo\gram}} = \boxed{2.41\E{-15}\,\unit{\joule}}
                \end{equation}

            \subsubsection{Proton}
                Use equation (\ref{eq:44}) with $m = 1.6726\E{-27}\,\unit{\kilo\gram}$.
                \begin{equation}
                    E   =   \frac{2.195\E{-45}\unit{\joule\cdot\kilo\gram}}{1.6726\E{-27}\,\unit{\kilo\gram}} = \boxed{1.3124\E{-18}\,\unit{\joule}}
                \end{equation}

            \subsubsection{Neutron}
                Use equation (\ref{eq:44}) with $m = 1.6749\E{-27}\,\unit{\kilo\gram}$.
                \begin{equation}
                    E   =   \frac{2.195\E{-45}\unit{\joule\cdot\kilo\gram}}{1.6749\E{-27}\,\unit{\kilo\gram}} = \boxed{1.3106\E{-18}\,\unit{\joule}}
                \end{equation}
\end{document}