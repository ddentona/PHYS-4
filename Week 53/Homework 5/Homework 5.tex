\documentclass[12pt]{article}
\usepackage{amsmath}
\usepackage{amssymb}
\usepackage{amsthm}
\usepackage{cancel}
\usepackage{enumitem}
\usepackage{esdiff}
\usepackage{graphicx}
\usepackage{multicol}
\usepackage{siunitx}
\usepackage{ulem}
% \usepackage{pgfplots}
\usepackage{wrapfig}
\usepackage{xcolor}

\newcommand{\e}[1]{e^{#1}}
\newcommand{\ei}[1]{e^{i\left(#1\right)}}
\newcommand{\E}[1]{\times 10^{#1}}

\renewcommand\qedsymbol{TENA}

\title{
    Homework \#5
    \\  \small
    PHYS 4D: Modern Physics
    }
\author{Donald Aingworth IV}
% \date{January 26, 2026}

\begin{document}
    \maketitle

    \section{Questions}
    % \begin{multicols}{2}
        \small
        \subsection{Question 1}
            Give the operators corresponding to the momentum and energy of a particle.

            \subsubsection{Answer}
                \begin{equation}
                    \hat{p} = -i \hbar \diffp{}{x};
                    \hat{H} = -\frac{\hbar^2}{2m} \diffp[2]{}{t} + V(x,t)
                \end{equation}
        
        \subsection{Question 2}
            Give the general form of an eigenvalue equation.

            \subsubsection{Answer}
                For operator $\hat{A}$, eigenfunction $\psi$, and eignenvalue $A$.
                \begin{equation}
                    \hat{A}\psi = A\psi
                \end{equation}

        \subsection{Question 3}
            What significance do the eigenvalues have?

            \subsubsection{Answer}
                The eigenvalue is the formula for or value of the component being searched for within the bounds of the eigenfunction.

        \subsection{Question 4}
            What significance do the eigenfunctions have?
                
            \subsubsection{Answer}
                The eigenfunctions serve as the conditions governing the system.

        \subsection{Question 5}
            Write down the momentum eigenvalue equation.
            
            \subsubsection{Answer}
                Ths is a \textit{solved} momentum eigenvalue equation.
                \begin{equation}
                    -i \hbar \diffp{}{x} \psi = \hbar k \psi
                \end{equation}

        \subsection{Question 6}
            Is the function $\cos(kx)$ an eigenfunction of the momentum operator?

            \subsubsection{Answer}
                No. 
                The first derivative of $\cos(kx)$ is $-k\sin(kx)$, which does not contain $\cos(kx)$ to the point that the sets of terms in the two equations are practically disjoint.

        \subsection{Question 7}
            Is the function $\cos(kx)$ an eigenfunction of the operator corresponding to the kinetic energy?

            \subsubsection{Answer}
                Yes. 
                The energy operator involves a second derivative.
                Taking the second derivative of $\cos(kx)$, we get $-k^2 \cos(kx)$.
                This does contain $\cos(kx)$, so it is an eigenfunction.

        \subsection{Question 8}
            An electron is described by the wave function $\psi(x)=Ae^{i \alpha x}$, where $\alpha$ denotes the Greek letter alpha. 
            What is the momentum of the electron?
            
            \subsubsection{Answer}
                Use the equation with the momentum operator.
                \begin{equation}
                    \hat{p} \psi = p \psi
                \end{equation}

                Plug in the momentum operator $\hat{p} = -i\hbar\diffp{}{x}$ and $\psi = Ae^{i \alpha x}$.
                \begin{align}
                    \hat{p} \psi    &=  -i\hbar \diffp{}{x}Ae^{i \alpha x}
                        =   \hbar\alpha\,Ae^{i \alpha x}
                \end{align}

                This does contain $\psi$, so we can divide that out of the equation.
                \begin{gather}
                    p \psi = \hbar\alpha\,Ae^{i \alpha x}\\
                    \boxed{p = \hbar\alpha}
                \end{gather}

        \subsection{Question 9}
            How would you determine whether or not a particular wave function represented a state of the system having a definite value of the momentum?

            \subsubsection{Answer}
                Use the eigenfunction equation for the momentum ($\hat{p} \psi$).
                If the result is a multiplel of $\psi$, then the wave function did represent a state of the system with a definitive momentum value.

        % \subsection{Question 10}
        %     Calculate the dot product of the following two vectors.
        %     \begin{equation}
        %         A = \begin{pmatrix} 4 \\ 3 \\ 2 \\ 1 \end{pmatrix}; 
        %         B = \begin{pmatrix} 1 \\ 1 \\ 2 \\ 2 \end{pmatrix}
        %     \end{equation}

        %     \subsubsection{Answer}
        %         \begin{align}
        %             A \cdot B   &=  4 * 1 + 3 * 1 + 2 * 2 + 1 * 2\\
        %                 &=  4 + 3 + 4 + 2
        %                 =   \boxed{13}
        %         \end{align}

        % \subsection{Question 11}
        %     For the two vectors, A and B, given in the preceding question, evaluate item by item $BA$ and from there $A \cdot BA$.

        %     \subsubsection{Answer}
        %         \begin{align}
        %             &BA =   \begin{pmatrix} 4 * 1 \\ 3 * 1 \\ 2 * 2 \\ 1 * 2 \end{pmatrix}
        %                 =   \boxed{\begin{pmatrix} 4 \\ 3 \\ 4 \\ 2 \end{pmatrix}} 
        %             &A \cdot BA =   \begin{pmatrix} 4 \\ 3 \\ 2 \\ 1 \end{pmatrix} \cdot \begin{pmatrix} 4 \\ 3 \\ 4 \\ 2 \end{pmatrix}
        %                 =   16 + 9 + 8 + 2
        %                 =   \boxed{35}
        %         \end{align}

        % \subsection{Question 12}
        %     Suppose a function f(x) is defined in an interval between 0 and 10 and an equally spaced grid is created by the command,\\
        %         {\fontfamily{qcr}\selectfont x = linspace(0.0,10.0,10)}
        %     and suppose the values of the function at the two Gauss quadrature points, $\xi_{i1}$ and $\xi_{i2}$, within each interval are known.
        %     Write down a formula for determining the value of the following integral in terms of the values of the function at the Gauss points.
        %     \begin{equation}
        %         \int_{0}^{10} f(x)\,dx
        %     \end{equation}

        % \subsection{Question 13}
        %     Suppose a differential equation is solved using spline collocation on a grid created by the command {\fontfamily{qcr}\selectfont x = linspace(0.0,10.0,10)}.
        %     How many rows and columns do the A, B, and C matrices have?

        \subsection{Question 14}
            Sketch the wave function for an electron incident upon a potential step when the energy $E$ of the electron is greater than the step height $V_0$.

            \subsubsection{Answer}
                \includegraphics[width=250pt]{Image_5-14.jpg}

        \subsection{Question 15}
            Sketch the wave function for an electron incident upon a potential step when the energy $E$ of the electron is less than the step height $V_0$.

            \subsubsection{Answer}
                \includegraphics[width=250pt]{Image_5-15.jpg}

        \subsection{Question 16}
            Is the wavelength of a particle that has passed over a potential barrier greater than or less than the wavelength of the incident particle?

            \subsubsection{Answer}
                While in the barrier, the wavelength would be larger. 
                After passing through the barrier, the wavelength would be equal.

        \subsection{Question 17}
            Sketch the wave function of an electron which tunnels through a potential barrier located between 0 and L.

            \subsubsection{Answer}
                \includegraphics[width=250pt]{Image_5-17.jpg}

        \subsection{Question 18}
            Write down the Heisenberg uncertainty relations for the position and the momentum and for the time and the energy.

            \subsubsection{Answer}
                \begin{equation}
                    \Delta x \Delta p = \frac{\hbar}{2};
                    \Delta t \Delta E = \frac{\hbar}{2}
                \end{equation}

        % \subsection{Question 19}
        %     Use the Heisenberg relation for the time and the energy to describe how the energy profile of an excited atomic state depends upon the lifetime of the state.

        % \subsection{Question 20}
        %     How would the wave function $\psi(x)$ of a particle change as the width of the Fourier transform increases?

        %     \subsubsection{Answer}
        %         It would decrease.

        \subsection{Question 21}
            Suppose that a particle localized between a and b is described by the wave function $\psi(x)$. Write down an equation for the average value of the kinetic energy of the particle.

            \subsubsection{Answer}
                Average values are dictated through the use of $\psi$ and $\psi*$.
                The value of $K$ has an operator for $K = -\frac{\hbar^2}{2m} \nabla^2$.
                \begin{equation}
                    \left\langle K \right\rangle    =   \int_{a}^{b} \psi*(x) K \psi(x)\,dx
                        =   \int_{a}^{b} \psi*(x) \left( -\frac{\hbar^2}{2m} \nabla^2 \psi(x) \right)\,dx
                \end{equation}
                

    \section{Problem 1}
        Evaluate the product of the momentum operator and each of the two functions
        \begin{gather}
            \phi_1(x) = \cos\,kx\\
            \phi_2(x) = \sin\,kx
        \end{gather}
        Are these functions eigenfunctions of the momentum operator?

        \subsection{Solution (1)}
            The momentum operator is $\hat{p} = -i\hbar\diffp{}{x}$, which we can apply here.
            \begin{align}
                \hat{p}\,\phi_1(x)  &=  -i\hbar\diffp{}{x} (\cos\,kx)
                    =   i\hbar k \sin\,kx
            \end{align}

            This does not contain $\phi_1$, so it is not an eigenfunction of $\hat{p}$.

        \subsection{Solution (2)}
            The momentum operator is $\hat{p} = -i\hbar\diffp{}{x}$, which we can apply here.
            \begin{align}
                \hat{p}\,\phi_2(x)  &=  -i\hbar\diffp{}{x} (\sin\,kx)
                    =   -i\hbar k \cos\,kx
            \end{align}

            This does not contain $\phi_2$, so it is not an eigenfunction of $\hat{p}$.
        \pagebreak
        
    \section{Problem 2}
        Find a linear combination of the functions, $\phi_1(x)$ and $\phi_2(x)$, defined in the previous problem, which is an eigenfunction of the momentum operator.

        \subsection{Solution}
            We know that Euler's identity is an eigenfunction of the momentum operator. 
            \begin{equation}
                e^{ix} = \cos(x) + i\sin(x)
            \end{equation}

            We can change $x$ into $kx$.
            \begin{equation}
                e^{ikx} = \cos(kx) + i\sin(kx)
            \end{equation}

            This is a linear combination of $\phi_1(x)$ and $\phi_2(x)$, just a complex one.
            We can check if it is an eigenfunction of the momentum operator.
            \begin{align}
                \hat{p}\,\psi(x)    &=  -i\hbar\diffp{}{x} (\cos(kx) + i\sin(kx))
                    =   -i\hbar\diffp{}{x} (e^{ikx})\\
                    &=  -i\hbar (ik\e{ikx})
                    =   \hbar k \e{ikx}
                    =   \hbar k \psi(x) \qed
            \end{align}
        \pagebreak

    \section{Problem 8}
        \begin{wrapfigure}{r}{0.33\textwidth}
            \includegraphics[width=0.9\linewidth]{Image_5-8.png}
            \caption[3.5]{Potential Step}
            \label{fig:3.5}
        \end{wrapfigure}
        For the scattering problem illustrated in Fig. \ref{fig:3.5}, a particle is incident upon a potential step with the energy of the incident particle being less than the step height.
        Using the notation for this problem given in the text, derive expressions the ratios $A/C$ and $B/C$ and show that $R = 1$.

        \subsection{Solution}
            I'll be doing this the way the book did it.
            I'll be using $D$ instead of $C$, largely because it works out with some thing more coherent this way.
            We begin with the Schrödinger equation.
            \begin{equation}
                -\frac{\hbar^2}{2m} \nabla^2 \psi + V \psi = E \psi
            \end{equation}

            This exists in two cases, $psi_1$ for $x < 0$ and $\psi_2$ for $x > 0$, with separate values of $V$.
            They give us two values of $k$.
            \begin{gather}
                k_1 = \frac{\sqrt{2mE}}{\hbar}\\
                k_{2i} = \frac{\sqrt{2m(E - V_0)}}{\hbar} = i\frac{\sqrt{2m(V_0 - E)}}{\hbar}
            \end{gather}

            We can also defne $k_{2} = -ik_{2i}$.
            \begin{equation}
                k_2 = \frac{\sqrt{2m(V_0 - E)}}{\hbar}
            \end{equation}

            $k_1$ has a specified value of $\psi$ as a traveling wave.
            Meanwhile, because of the $i$ in $k_{2i}$, the value of $\psi_2$ would be exponential rather than sinusoidal.
            \begin{gather}
                \psi_1  =   A \e{i k_1 x} + B \e{-i k_1 x}\\ \label{eq:8-6}
                \psi_2  =   E \e{i k_{2i} x} + F \e{-i k_{2i} x}
                        =   C \e{k_{2} x} + D \e{-k_{2} x}
            \end{gather}

            Given boundary conditions, we have some equations.
            \begin{gather}
                \psi_1(0) = \psi_2(0)\\
                \psi_1'(0) = \psi_2'(0)
            \end{gather}

            We can set them up through substituting.
            \begin{gather}
                A \e{i k_1 x} + B \e{-i k_1 x} = C \e{k_{2} x} + D \e{-k_{2} x}\\
                i k_1 (A \e{i k_1 x} - B \e{-i k_1 x}) = k_2 (C \e{k_{2} x} - D \e{-k_{2} x})
            \end{gather}

            Set $x = 0$.
            Anything to a power of $x$ will be reduced to $1$.
            \begin{gather}
                A + B = C + D\\
                i k_1 (A - B) = k_2 (C - D)\\
                A - B = \frac{k_2}{i k_1} (C - D)
            \end{gather}

            Add and subtract the first and the third together.
            \begin{align}
                A   &=  \frac{1}{2}(C + D) + \frac{k_2}{2i k_1} (C - D)
                    =   \frac{ik_1 + k_2}{2i k_1} C + \frac{ik_1 - k_2}{2i k_1} D\\
                B   &=  \frac{1}{2}(C + D) + \frac{k_2}{2i k_1} (D - C)
                    =   \frac{ik_1 - k_2}{2i k_1} C + \frac{ik_1 + k_2}{2i k_1} D
            \end{align}

            Look again at (\ref{eq:8-6}).
            \begin{equation}
                \psi_2 = C \e{k_{2} x} + D \e{-k_{2} x} \tag{\ref{eq:8-6}}
            \end{equation}

            It would be decreasing, not increasing, so it would requre that $C = 0$.
            \begin{align}
                A   =   \frac{ik_1 - k_2}{2i k_1} D \to \boxed{\frac{A}{D} = \frac{ik_1 - k_2}{2i k_1}}\\
                B   =   \frac{ik_1 + k_2}{2i k_1} D \to \boxed{\frac{B}{D} = \frac{ik_1 + k_2}{2i k_1}}
            \end{align}

            There is a defined formula for the reflection coefficient for waves.
            \begin{align}
                R   &=  \frac{B* B}{A* A}
                    =   \frac{\frac{-ik_1 + k_2}{\cancel{-2i k_1}}}{\frac{-ik_1 - k_2}{\cancel{-2i k_1}}} \cdot \frac{\frac{ik_1 + k_2}{\cancel{2i k_1}}}{\frac{ik_1 - k_2}{\cancel{2i k_1}}}
                    =   \frac{(-ik_1 + k_2)(ik_1 + k_2)}{(-ik_1 - k_2)(ik_1 - k_2)}
                    =   \frac{k_2^2 + k_1^2}{k_2^2 + k_1^2}
                    =   \boxed{1}
            \end{align}
        \pagebreak
    
    \section{Problem 10}
        Using the Heisenberg uncertainty principle, estimate the momentum of an electron confined to a 1.0 nm well.

        \subsection{Solution}
            Being traditionally trained in mathematics, I feel the need to note that the Heisenberg Uncertainty Principle is not about determining actual value but uncertainty, which are two different concepts, even if they come from the same source.
            Here, we can use the Heisenberg Uncertainty Principle to get a minimum possible uncertainty.
            \begin{equation}
                \Delta p    =   \frac{\hbar}{2 \Delta x}
                    =   \frac{1.055\E{-34}\,\unit{\joule\cdot\second}}{2 \times 1.0\E{-9}\,\unit{\meter}}
                    =   \boxed{5.275\E{-26}\,\unit{\frac{\kilo\gram\cdot\meter}{\second}}}
            \end{equation}
            
    % \end{multicols}
\end{document}