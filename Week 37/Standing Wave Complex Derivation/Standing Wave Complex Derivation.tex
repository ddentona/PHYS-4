\documentclass[12pt]{article}
\usepackage{amsmath}
\newcommand{\E}[1]{\times 10^{#1}}
\newcommand{\e}[1]{{\rm e}^{#1}}
\begin{document}
    \begin{center}
        \textbf{A Document Detailing the Complex Derivation of a Standing Wave Formula from that of Two Opposite Traveling Waves}\\
        Author: Think the Duck
    \end{center}
    \textit{For Prof Cauthen}

    This is a document meant for the sole purpose of deriving the formula for a standing wave in a string from its definition as two moving waves in opposite directions.

    We will be using the complex notation for a simple traveling wave. 
    We will be using the constants used for a wave in a string, but this could equally be applied to another situation.
    or the sake of this problem, I will be dividing the offset angle into two halves from the start. 
    They are constants and should be treated as such, including division into two or more parts at any necessary time.
    \begin{equation}
        \Psi_{\rm fwd}(x,t)   =   \psi_m \e{i(kx + \omega t + \phi + \phi')}
    \end{equation}

    This is the formula for a traveling wave in one direction (forward).
    The formula for a traveling wave in the other direction would involve inverting the sign of the angular speed ($\omega$).
    We can also invert the sign of $\phi'$, selecting it as the offset angle of the angular velocity of both waves.
    This does nto affect the nature of the wave because it affects neither the amplitude, nor any of the wave's constants, including those used in the wave equation.
    \begin{equation}
        \Psi_{\rm rev}(x,t)   =   \psi_m \e{i(kx - \omega t + \phi - \phi')}
    \end{equation}

    This gives us a pair of complex equations for waves. 
    It should be noted that both of the exponentials can be equated to two separate exponentials multiplied together.
    \begin{gather}
        \Psi_{\rm fwd}(x,t)   =   \psi_m \e{i(kx + \omega t + \phi + \phi')}
            =   \psi_m \e{i(kx + \phi)} \e{i(\omega t + \phi')}\\
        \Psi_{\rm rev}(x,t)   =   \psi_m \e{i(kx - \omega t + \phi - \phi')}
            =   \psi_m \e{i(kx + \phi)} \e{i(-\omega t - \phi')}
    \end{gather}

    It is said that a standing wave is equivalent to these twotraveling waves going in opposite directions.
    We can make use of this idea by adding these two values together.
    Since these waves are equal and opposite, all values of $\phi_m$ are the same.
    \begin{align}
        \Psi(x,t)   &=  \Psi_{\rm fwd}(x,t) + \Psi_{\rm rev}(x,t)
            =   \psi_m \e{i(kx + \omega t + \phi + \phi')} + \psi_m \e{i(kx - \omega t + \phi - \phi')}\\
            &=  \psi_m \e{i(kx + \phi)} \e{i(\omega t + \phi')} + \psi_m \e{i(kx + \phi)} \e{i(-\omega t - \phi')}\\
            &=  \psi_m \left( \e{i(kx + \phi)} \e{i(\omega t + \phi')} + \e{i(kx + \phi)} \e{i(-\omega t - \phi')} \right)
    \end{align}
    
    From here, we convert to trigonometric immediately.
    \begin{align}
        \Psi(x,t)   =   &\psi_m \left( \left( \cos(kx + \phi) + i\sin(kx + \phi) \right) \left( \cos(\omega t + \phi') + i\sin(\omega t + \phi') \right) \right.\\ &\left.+ \left( \cos(kx + \phi) + i\sin(kx + \phi) \right) \left( \cos(-\omega t - \phi') + i\sin(-\omega t - \phi') \right) \right)
    \end{align}

    We can separate this into the first and second parts of the equation that is in the parentheses.
    We start with the first half (the expansion of $\e{i(kx + \phi)} \e{i(\omega t + \phi')}$). 
    \begin{gather}
        \left( \cos(kx + \phi) + i\sin(kx + \phi) \right) \left( \cos(\omega t + \phi') + i\sin(\omega t + \phi') \right)
    \end{gather}

    This can be expanded. 
    \begin{multline}
        \cos(kx + \phi)\cos(\omega t + \phi') - \sin(kx + \phi)\sin(\omega t + \phi')\\
            +   i\left( \cos(kx + \phi)\sin(\omega t + \phi') + \sin(kx + \phi)\cos(\omega t + \phi') \right)
    \end{multline}

    Next, the second half (the expansion of $\e{i(kx + \phi)} \e{i(-\omega t - \phi')}$). 
    \begin{gather}
        \left( \cos(kx + \phi) + i\sin(kx + \phi) \right) \left( \cos(-\omega t - \phi') + i\sin(-\omega t - \phi') \right)
    \end{gather}

    This can be expanded.
    \begin{multline}
        \cos(kx + \phi)\cos(-\omega t - \phi') - \sin(kx + \phi)\sin(-\omega t - \phi')\\
            +   i\left( \cos(kx + \phi)\sin(-\omega t - \phi') + \sin(kx + \phi)\cos(-\omega t - \phi') \right)
    \end{multline}

    The only trigonometric identity we will need in this derivation is $\cos(x) = \cos(-x)$ and $\sin(-x) = -\sin(x)$. 
    We will be using that here.
    \begin{multline}
        \cos(kx + \phi)\cos(\omega t + \phi') + \sin(kx + \phi)\sin(\omega t + \phi')\\
            +   i\left( - \cos(kx + \phi)\sin(\omega t + \phi') + \sin(kx + \phi)\cos(\omega t + \phi') \right)
    \end{multline}

    The two halves can be added together. 
    Notice that how in the real plane/axis, the $\sin(kx + \phi)\sin(\omega t + \phi')$ and $-\sin(kx + \phi)\sin(\omega t + \phi')$ cancel each other out, while in the imaginary plane/axis, the $\cos(kx + \phi)\sin(\omega t + \phi')$ and $-\cos(kx + \phi)\sin(\omega t + \phi')$ cancel each other out. 
    \begin{multline}
        \Psi(x,t)   =   \psi_m (\cos(kx + \phi)\cos(\omega t + \phi') - \sin(kx + \phi)\sin(\omega t + \phi')\\
            +   i\left( \cos(kx + \phi)\sin(\omega t + \phi') + \sin(kx + \phi)\cos(\omega t + \phi') \right)\\ + \cos(kx + \phi)\cos(\omega t + \phi') + \sin(kx + \phi)\sin(\omega t + \phi')\\
            +   i\left( - \cos(kx + \phi)\sin(\omega t + \phi') + \sin(kx + \phi)\cos(\omega t + \phi') \right))
    \end{multline}
    \begin{align}
        \Psi(x,t)   &=  \psi_m (2\cos(kx + \phi)\cos(\omega t + \phi') + 2i \sin(kx + \phi)\cos(\omega t + \phi'))\\
        \Psi(x,t)   &=  2 \psi_m (\cos(kx + \phi)\cos(\omega t + \phi') + i \sin(kx + \phi)\cos(\omega t + \phi'))
    \end{align}

    Since we are observing the case of a string, we only will have to work the real part of this.
    For cases where the real and imaginary are both used (e.g. electromagnetic waves), the imaginary numbers would come into play, but that is not the case here. 
    \begin{align}
        y(x,t)  &=  {\rm Re}[\Psi(x,t)]\\
            &=  {\rm Re}[2 \psi_m (\cos(kx + \phi)\cos(\omega t + \phi') + i \sin(kx + \phi)\cos(\omega t + \phi'))]\\
            &=  2 \psi_m \cos(kx + \phi)\cos(\omega t + \phi')
    \end{align}

    Note the constant 2.
    We can get rid of that by setting $y_m = 2\psi_m$.
    This has to do with the setup of the waves themsleves, where the two original waves must have a lower amplitude than their product and the total amplitude would be that of the initial two waves added together.
    \begin{equation}
        y(x,t)  =   y_m \cos(kx + \phi)\cos(\omega t + \phi')
    \end{equation}

    This document has fulfilled its sole purpose.
\end{document}