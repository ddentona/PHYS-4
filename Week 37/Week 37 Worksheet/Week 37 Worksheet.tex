\documentclass[12pt]{article}
\usepackage{amsmath}
\usepackage{amssymb}
\usepackage{cancel}
\usepackage{enumitem}
\usepackage{esdiff}
\usepackage{graphicx}
\usepackage{siunitx}
\usepackage{pgfplots}
\usepackage{wrapfig}

\newcommand{\E}[1]{\times 10^{#1}}

\title{
    Worksheet \#6
    \\  \small
    PHYS 4C: Waves and Thermodynamics
    }
\author{Donald Aingworth IV}
\date{September 22, 2025}

\begin{document}
    \DeclareSIUnit{\celsiusdegree}{C^\circ}
    \DeclareSIUnit{\atm}{ atm}

    \maketitle

    \section{Problem}
        A wave in a string is described by the following function:
        \begin{equation}
            y(x,t) = (0.34\,\unit{\milli\meter}) \cos((24\,\unit{\radian/\meter})x - (960\,\unit{\radian/\second})t)
        \end{equation}

        The mass per unit length of the string is 1.2\,\unit{\gram/\meter}.
        Identify or calculate the following (include correct units).

        \begin{enumerate}[label=\alph*)]
            \item Amplitude:
            \item Wavenumber:
            \item Wavelength:
            \item Angular frequency:
            \item Cycle frequency:
            \item Period:
            \item Wave velocity (include direction):
            \item Tension in the string:
            \item Maximum speed of some specific part of the string:
            \item Kinetic energy density (per unit length), time-averaged:
            \item Potential energy density, time-averaged:
            \item Rate of energy transfer across some specific point, time-averaged:
            \item Total energy in a -long section:
            \item How much time does it take for the amount of energy calculated in part m to cross a specific point in the string?
        \end{enumerate}

        \subsection{Solution (a)}
            For the first four parts (except (c)), we will find the answers in the structure of the wave equation and the general wave equation.
            \begin{gather}
                \psi (x,t)  =   \underline{\psi_m} \cos(k x - \omega t)\\
                y(x,t)  =   (\underline{0.34\,\unit{\milli\meter}}) \cos((24\,\unit{\radian/\meter})x - (960\,\unit{\radian/\second})t)\\
                \boxed{\psi_m   =   3.4\E{-4}\,\unit{\meter}}
            \end{gather}
        
        \subsection{Solution (b)}
            For the first four parts (except (c)), we will find the answers in the structure of the wave equation and the general wave equation.
            \begin{gather}
                \psi (x,t)  =   \psi_m \cos(\underline{k} x - \omega t)\\
                y(x,t)  =   (0.34\,\unit{\milli\meter}) \cos((\underline{24\,\unit{\radian/\meter}})x - (960\,\unit{\radian/\second})t)\\
                \boxed{k   =   24\,\unit{\radian/\meter}}
            \end{gather}

        \subsection{Solution (c)}
            The wavelength is inversely proportional to the wave number.
            \begin{equation}
                \lambda =   \frac{2\pi}{k}
                    =   \frac{2\pi}{24\,\unit{\radian/\meter}}
                    =   \frac{\pi}{12}\,\unit{\meter}
                    =   \boxed{0.2618\,\unit{\meter}}
            \end{equation}
        
        \subsection{Solution (d)}
            For the first four parts (except (c)), we will find the answers in the structure of the wave equation and the general wave equation.
            \begin{gather}
                \psi (x,t)  =   \psi_m \cos(k x - \underline{\omega} t)\\
                y(x,t)  =   (0.34\,\unit{\milli\meter}) \cos((24\,\unit{\radian/\meter})x - (\underline{960\,\unit{\radian/\second}})t)\\
                \boxed{\omega   =   960\,\unit{\radian/\second}}
            \end{gather}

        \subsection{Solution (e)}
            The cycle frequency is proportional to the angular frequency.
            \begin{gather}
                f   =   \frac{\omega}{2\pi}
                    =   \frac{960\,\unit{\radian/\second}}{2\pi}
                    =   \frac{480\,\unit{\radian/\second}}{\pi}
                    =   \boxed{152.8\,\unit{\hertz}}
            \end{gather}

        \subsection{Solution (f)}
            The period is the reciprocal of the frequency.
            \begin{gather}
                T   =   \frac{1}{f}
                    =   \frac{\pi}{480\,\unit{\radian/\second}}
                    =   \boxed{6.545\,\unit{\milli\second}}
            \end{gather}

        \subsection{Solution (g)}
            The wave speed is determined by a fraction.
            \begin{align}
                v   =   \frac{\lambda}{T}
                    =   \frac{\frac{\pi}{12}}{\frac{\pi}{480}}
                    =   \frac{480}{12}
                    =   \boxed{40\,\unit{\meter/\second}}
            \end{align}

        \subsection{Solution (h)}
            The tension is measured in therms of the density and the wave speed. 
            \begin{gather}
                \mu =   1.2\,\unit{\gram/\meter}
                    =   1.2\E{-3}\,\unit{\kilo\gram/\meter}\\
                v   =   \sqrt{\frac{\tau}{\mu}}\\
                v^2 =   40^2
                    =   1600\,\unit{\meter^2/\second^2}
                    =   \frac{\tau}{\mu}\\
                \begin{align}
                    \tau    &=  \mu v^2
                        =   1.2\E{-3}\,\unit{\kilo\gram/\meter} * 1600\,\unit{\meter^2/\second^2}\\
                        &=  1.2 * 1.6\,\unit{\kilo\gram\,\meter^2/\second^2\,\meter}
                        =   \boxed{1.92\,\unit{\newton}}
                \end{align}
            \end{gather}

        \subsection{Solution (i)}
            We can first find the formula for the velocity.
            By nature, the velocity would only be vertical since the wire would not be moving horizontally.
            \begin{align}
                v   &=  \diff{}{t}\left( (0.34\,\unit{\milli\meter}) \cos((24\,\unit{\radian/\meter})x - (960\,\unit{\radian/\second})t) \right)\\
                    &=  - 3.4\E{-4} * 960\,\unit{\radian/\second} * \sin((24\,\unit{\radian/\meter})x - (960\,\unit{\radian/\second})t)
            \end{align}

            For the maximum value of this velocity, the sine would have to be equal to one (or negative one) so we can set it to that.
            \begin{equation}
                v   =   -3.4\E{-4}\,\unit{\meter} * 960\,\unit{\radian/\second} * (-1)
                    =   \boxed{0.3264\,\unit{\meter/\second}}
            \end{equation}

        \subsection{Solution (j)}
            We can measure the kinetic energy stored over the course of one period and use that to find the average kinetic energy.

        \subsection{Solution (k)}

        \subsection{Solution (l)}

        \subsection{Solution (m)}

        \subsection{Solution (n)}
\end{document}