\documentclass[12pt]{article}
\usepackage{amsmath}
\begin{document}
    \begin{center}
        \textbf{A Document Detailing the Derivation of Magnetic Field from a Toroid along its Central Rotating Axis}\\
        Author: Think the Duck
    \end{center}
    \textit{For Prof Cauthen}

    This is a document meant for the sole purpose of deriving the formula for the total magnetic field on the polar axis of the toroid at its center.

    We will be using Ampere's Law.
    We will aim to find the value of $B$.
    \begin{gather}
        \oint \vec{B} \cdot d\vec\ell = \mu_0 I_{enc}\\
        \oint B * \cos(\theta)\ d\ell = \mu_0 I_{enc}
    \end{gather}

    The area we will use will be the circle formed through the center of all rotations of the toroid.
    We can assume that each of the points lies at a radius $r$ from the center of the entire object.
    The formula for the line traversed by its boundary is $2\pi r$.
    $\cos(\theta) = 1$ will always be the case in this instance, since the magnetic field will always go around the toroid in the same direction.
    \begin{gather}
        B \oint d\ell = \mu_0 I_{enc}\\
        B * 2\pi r = \mu_0 I_{enc}\\
        B = \frac{\mu_0 I_{enc}}{2\pi r}
    \end{gather}

    The enclosed current is the sum of the current in each point along the $N$ rotations of the toroid that intersect with the area previously described.
    Since the current is a constant (suppose it to be $I$), we can say that $I_{enc} = NI$.
    \begin{equation}
        B = \frac{\mu_0 NI}{2\pi r}
    \end{equation}

    This document has fulfilled its sole purpose.
\end{document}