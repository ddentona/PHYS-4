\documentclass[12pt]{article}
\usepackage{amsmath}
\usepackage{amssymb}
\usepackage{cancel}
\usepackage{enumitem}
\usepackage{graphicx}
\usepackage{physics}
\usepackage{siunitx}
\usepackage{pgfplots}
\usepackage{wrapfig}

\AtBeginDocument{\RenewCommandCopy\qty\SI}
\newcommand{\E}[1]{\times 10^{#1}}

\title{
    Worksheet \#5
    \\  \small
    PHYS 4C: Waves and Thermodynamics
    }
\author{Donald Aingworth IV}
\date{September 22, 2025}

\begin{document}
    \DeclareSIUnit{\celsiusdegree}{C^\circ}
    \DeclareSIUnit{\atm}{ atm}

    \maketitle

    \section{Problem}
        (20 points) A Carnot heat engine operates on a four step cycle as follows:
        \begin{enumerate}
            \item Isothermal expansion at temperature $T_3$, where heat $Q_H$ flows into the system from an external temperature reservoir $T_4$ $(T_4 \geq T_3)$. 
            This step is reversible if $T_4 = T_3$.
            \item Adiabatic expansion (no heat flow), where the temperature drops from $T_3$ to $T_2$ $(T_2 < T_3)$. 
            This step is always reversible (as long as it is quasi-static).
            \item Isothermal compression at temperature $T_2$, where heat $Q_C$ flows out of the system to an external temperature reservoir $T_1$ $(T_1 \leq T_2)$. 
            This step is reversible if $T_1 = T_2$.
            \item Adiabatic compression (no heat flow), where the temperature rises from $T_2$ back up to $T_3$. 
            This step is always reversible.
        \end{enumerate}
        
        The efficiency of this cycle is given by $e = 1 - T_2/T_3$, and is maximized when $T_3 = T_4$ and $T_2 = T_1$ (reversible), although the work output rate in that case is zero (heat flowrates are zero for steps 1 and 3). 
        Allowing $T_1 < T_2 < T_3 < T_4$ enables us to consider a real heat engine that would run at a finite rate.
        
        Consider such a heat engine with $T_4$ = 600 K, $T_3$ = 500 K, $T_2$ = 400 K, and $T_1$ = 300 K.
        Suppose also that the heat conductance for the rods connecting the system to $T_4$ during step 1 and to $T_1$ during step 3 are each 10 W/K, and that the adiabatic steps (2 and 4) are both very rapid (essentially zero time).
        \begin{enumerate}[label=(\alph*)]
            \item (6 points) Calculate the efficiency of this heat engine. If $Q_H$ = 100 J for one cycle of this heat engine, how much heat flows into the cold reservoir ($Q_C$) and how much work is output ($W$) for each cycle?
            \item (6 points) Calculate the net change in entropy during one cycle. 
            During which steps does the positive entropy change occur?
            \item (8 points) Determine the rate of work production for this engine \\(work/time). 
            (Hint: calculate the total time for one cycle).
        \end{enumerate}

        \subsection{Solution (a)}
            It is given that the heat engine would have the starting temperature of $T_3 = 500\unit{\kelvin}$ and an ending temperature of $T_2 = 400\unit{\kelvin}$. 
            The formula given for efficiency can be used here. 
            The reason that we would use $T_2$ and $T_3$ instead of $T_1$ and $T_4$ because those are the reservoirs (remaining at constant temperature) directly driving the engine. 
            \begin{equation}
                e   =   1 - \frac{T_2}{T_3}
                    =   1 - \frac{400\unit{\kelvin}}{500\unit{\kelvin}}
                    =   \frac{1}{5}
                    =   0.20
            \end{equation}

            Now we can use the value of $Q_H$ to find the work done. 
            \begin{gather}
                e   =   \frac{W}{Q_H}\\
                W   =   e * Q_H
                    =   0.2 * 100\,\unit{\joule}
                    =   \boxed{20\,\unit{\joule}}
            \end{gather}

            We can use that to find the work done. 
            \begin{gather}
                Q_H =   W + Q_L\\
                Q_L =   Q_H - W
                    =   100 - 20
                    =   \boxed{80\,\unit{\joule}}
            \end{gather}

        \subsection{Solution (b)}
            Change in entropy is calculatable by an integral.
            \begin{align}
                \Delta S    &=  \int_{i}^{f}\frac{1}{T}\,dQ
            \end{align}

            This is a Carnot engine.
            In a Carnot engine, we have to assume that the change in entropy \textit{is} the zero. 
            This is necessary for the engine to have a cyclical process. 
            \begin{equation}
                \boxed{\Delta S = 0}
            \end{equation}

        \subsection{Solution (c)}
            We have the heat conductance for the engine. 
            From that, we can calculate the rate at which heat flows through both the rods. 
            \begin{align}
                \dv{Q_H}{t} &=  \frac{Q}{t}
                    =   K\,\Delta T
                    =   10\,\unit{\watt/\kelvin} * 100\unit{\kelvin}
                    =   1000\,\unit{\watt}\\
                \dv{Q_L}{t} &=  \frac{Q}{t}
                    =   K\,\Delta T
                    =   10\,\unit{\watt/\kelvin} * 100\unit{\kelvin}
                    =   1000\,\unit{\watt}
            \end{align}

            We can use these to calculate the time taken for the heats $Q_H$ and $Q_L$ to get through.
            \begin{align}
                t_H &=  Q_H * \frac{1}{\dv{Q_H}{t}}
                    =  \frac{100\,\unit{\joule}}{1000\,\unit{\joule/\second}}
                    =   0.1\,\unit{\second}
                    \\
                t_L &=  Q_L * \frac{1}{\dv{Q_L}{t}} 
                    =   \frac{80\,\unit{\joule}}{1000\,\unit{\joule/\second}}
                    =   0.08\,\unit{\second}
            \end{align}

            Since the entire process of the engine would be instantaneous, the only time that would be taken would be the time for the heat to go from the hot reservoir into the engine and from the engine into the cold reservoir.
            \begin{align}
                t   &=  t_H + t_C
                    =   0.1 + 0.08
                    =   0.18\,\unit{\second}
            \end{align}

            From here, knowing the work output, we can calculate the rate of work production.
            \begin{align}
                P_W &=  \frac{W}{t}
                    =   \frac{20\,\unit{\joule}}{0.18\,\unit{\second}}
                    =   \boxed{111\,\unit{\watt}}
            \end{align}
\end{document}