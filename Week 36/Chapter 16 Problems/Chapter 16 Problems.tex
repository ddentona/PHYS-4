\documentclass[12pt]{article}
\usepackage{amsmath}
\usepackage{amssymb}
\usepackage{cancel}
\usepackage{graphicx}
% \usepackage{physics}
\usepackage{siunitx}
\usepackage{wrapfig}

% \AtBeginDocument{\RenewCommandCopy\qty\SI}

\newcommand{\E}[1]{\times 10^{#1}}

\title{
    Chapter 16 End-of-Chapter Problems
    \\ \small
    Halliday \& Resnick, 10th Edition
}

\author{Donald Aingworth IV}

\date{\small Hit me where it Matters}

\begin{document}
    \DeclareSIUnit{\atm}{atm}
    \DeclareSIUnit{\cal}{\ cal}
    \DeclareSIUnit{\Cal}{\ Cal}
    \DeclareSIUnit{\calorie}{\ cal}
    \DeclareSIUnit{\Calorie}{\ Cal}
    \DeclareSIUnit{\celsiusdegree}{C^\circ}
    \DeclareSIUnit{\fahrenheit}{^\circ F}
    \DeclareSIUnit{\fahrenheitdegree}{F^\circ}
    \DeclareSIUnit{\torr}{\ torr}

    \maketitle

    \pagebreak
    \section{Problem 1}
        If a wave $y(x, t) = (6.0 \unit{\milli\meter}) \sin(kx + (600\,\unit{\radian/\second})t + \phi)$ travels along a string, how much time does any given point on the string take to move between displacements y = +2.0 mm and y = -2.0 mm?

        \subsection{Solution}
            Notice that the amplitude of the string is 6.0\unit{\milli\meter}. 
            Our positve and negative final amplitudes of y are 2 and -2 millimeters respectively.
            Note that these two values have a magnitude of one third of the amplitude.
            There are four points on the unit circle where the sine of the angle is $\frac{1}{3}$. 
            We can assume that the problem is asking the minumum time, since there are two ways that a point can travel between 2mm and -2mm.
            These two points would be for $kx + (600\,\unit{\radian/\second})t + \phi = \theta$, starting when $\theta = \arcsin\left( \frac{1}{3} \right)$ and ending when $\theta = \pi - \arcsin\left( \frac{1}{3} \right)$. 
            We can find the difference between these two. 
            \begin{align}
                \Delta \theta   &=  \theta_f - \theta_i
                    =   \pi - \arcsin\left( \frac{1}{3} \right) - \arcsin\left( \frac{1}{3} \right)\\
                    &=  2.462\,\unit{\radian}
            \end{align}

            As far as we consider it, the only thing in $kx + (600\,\unit{\radian/\second})t + \phi$ changing is $t$.
            Since we're generalizing, the other things besides $(600\,\unit{\radian/\second})t$ can be considered constants and initial conditons, so we can look at $t$.
            \begin{gather}
                \begin{align}
                    \Delta \theta &=    (kx + (600\,\unit{\radian/\second})t_f + \phi) - (kx + (600\,\unit{\radian/\second})t_i + \phi)\\
                        &=  (600\,\unit{\radian/\second}) (t_f - t_i)
                \end{align}\\
                \Delta t    =   \frac{\Delta \theta}{(600\,\unit{\radian/\second})}
                    =   \frac{2.462\,\unit{\radian}}{(600\,\unit{\radian/\second})}
                    =   \boxed{0.0041\,\unit{\second}}
            \end{gather}

            Note: if you were to find the minimum time between -2mm and 2mm, that would result in 1.1ms, but that is not the question posed here. 

    \pagebreak
    \section{Problem 3}
        A wave has an angular frequency of 110 rad/s and a wavelength of 1.80 m. 
        Calculate (a) the angular wave number and (b) the speed of the wave.

        \subsection{Solution}

    % \pagebreak
    \section{Problem 5}
        A sinusoidal wave travels along a string. 
        The time for a particular point to move from maximum displacement to zero is 0.170 s.
        What are the (a) period and (b) frequency? 
        (c) The wavelength is 1.40 m; what is the wave speed?

        \subsection{Solution}

    % \pagebreak
    \section{Problem 9}
        A sinusoidal wave moving along a string is shown twice in Fig. 16-33, as crest A travels in the positive direction of an x axis by distance d = 6.0 cm in 4.0 ms. 
        The tick marks along the axis are separated by 10 cm; height H = 6.00 mm.
        The equation for the wave is in the form $y(x, t) = y_m sin(kx \pm \omega t)$, so what are (a) $y_m$, (b) $k$, (c) $\omega$, and (d) the correct choice of sign in front of $\omega$?

        \subsection{Solution}

    % \pagebreak
    \section{Problem 15}
        A stretched string has a mass per unit length of 5.00 g/cm and a tension of 10.0 N. 
        A sinusoidal wave on this string has an amplitude of 0.12 mm and a frequency of 100 Hz and is traveling in the negative direction of an x axis. 
        If the wave equation is of the form $y(x, t) = y_m sin(kx \pm \omega t)$, what are (a) $y_m$, (b) $k$, (c) $\omega$, and (d) the correct choice of sign in front of $\omega$?

        \subsection{Solution}

    % \pagebreak
    \section{Problem 19}
        What is the speed of a transverse wave in a rope of length 2.00 m and mass 60.0 g under a tension of 500 N?

        \subsection{Solution}

    % \pagebreak
    \section{Problem 21}
        A 100 g wire is held under a tension of 250 N with one end at x = 0 and the other at x = 10.0 m. 
        At time t = 0, pulse 1 is sent along the wire from the end at x = 10.0 m. 
        At time t = 30.0 ms, pulse 2 is sent along the wire from the end at x = 0. 
        At what position x do the pulses begin to meet?

        \subsection{Solution}

    % \pagebreak
    \section{Problem 25}
        A uniform rope of mass m and length L hangs from a ceiling. 
        (a) Show that the speed of a transverse wave on the rope is a function of y, the distance from the lower end, and is given by $v = \sqrt{gy}$.
        (b) Show that the time a transverse wave takes to travel the length of the rope is given by $t = 2\sqrt{L/g}$.

        \subsection{Solution}

    % \pagebreak
    \section{Problem 30}
        Use the wave equation to find the speed of a wave given in terms of the general function $h(x, t)$:
        \begin{equation}
            y(x, t) = (4.00 \unit{\milli\meter}) h[(30 \unit{\meter^{-1}})x + (6.0\,\unit{\second^{-1}})t]
        \end{equation}

        \subsection{Solution}

    % \pagebreak
    \section{Problem 31}
        Two identical traveling waves, moving in the same direction, are out of phase by $\pi/2$ rad. 
        What is the amplitude of the resultant wave in terms of the common amplitude $y_m$ of the two combining waves?

        \subsection{Solution}

    % \pagebreak
    \section{Problem 35}
        Two sinusoidal waves of the same frequency travel in the same direction along a string. 
        If $y_{m1} = 3.0 \unit{\centi\meter}$, $y_{m2} = 4.0 cm$, $\psi_1 = 0$, and $\psi_2 = \pi/2$ rad, what is the amplitude of the resultant wave?

        \subsection{Solution}

    % \pagebreak
    \section{Problem 37}
        These two waves travel along the same string:
        \begin{align}
            y_1(x, t)   &=  (4.60 \unit{\milli\meter}) sin(2\pi x - 400\pi t)\\
            y_2(x, t)   &=  (5.60 \unit{\milli\meter}) sin(2\pi x - 400\pi t + 0.80\pi\,\unit{\radian}).
        \end{align}
        What are (a) the amplitude and (b) the phase angle (relative to wave 1) of the resultant wave? 
        (c) If a third wave of amplitude 5.00 mm is also to be sent along the string in the same direction as the first two waves, what should be its phase angle in order to maximize the amplitude of the new resultant wave?

        \subsection{Solution}

    % \pagebreak
    \section{Problem 41}
        A string fixed at both ends is 8.40 m long and has a mass of 0.120 kg. 
        It is subjected to a tension of 96.0 N and set oscillating. 
        (a) What is the speed of the waves on the string? 
        (b) What is the longest possible wavelength for a standing wave? 
        (c) Give the frequency of that wave.

        \subsection{Solution}

    % \pagebreak
    \section{Problem 43}
        What are (a) the lowest frequency, (b) the second lowest frequency, and (c) the third lowest frequency for standing waves on a wire that is 10.0 m long, has a mass of 100 g, and is stretched under a tension of 250 N?

        \subsection{Solution}

    % \pagebreak
    \section{Problem 45}
        A string that is stretched between fixed supports separated by 75.0 cm has resonant frequencies of 420 and 315 Hz, with no intermediate resonant frequencies. 
        What are (a) the lowest resonant frequency and (b) the wave speed?

        \subsection{Solution}

    % \pagebreak
    \section{Problem 53}
        A string oscillates according to the equation
        \begin{equation}
            y = (0.50 \unit{\centi\meter}) \sin \left[  ( \frac{\pi}{3} \unit{\centi\meter^{-1}})x \right] \cos\left[ (40\pi \unit{\second^{-1}} )t \right]
        \end{equation}
        What are the (a) amplitude and (b) speed of the two waves (identical except for direction of travel) whose superposition gives this oscillation? 
        (c) What is the distance between nodes? 
        (d) What is the transverse speed of a particle of the string at the position x = 1.5 cm when t = $\frac{9}{8}$ s?

        \subsection{Solution}

    \pagebreak
    \section{Problem 61}

        \subsection{Solution}

    \pagebreak
    \section{Problem 65}

        \subsection{Solution}

    \pagebreak
    \section{Problem 73}

        \subsection{Solution}

    \pagebreak
    \section{Problem 75}

        \subsection{Solution}

    \pagebreak
    \section{Problem 83}

        \subsection{Solution}

    \pagebreak

    \tableofcontents
    
\end{document}