\documentclass[12pt]{article}
\usepackage{amsmath}
\usepackage{amssymb}
\usepackage{cancel}
\usepackage{enumitem}
\usepackage{esdiff}
\usepackage{graphicx}
\usepackage{siunitx}
% \usepackage{pgfplots}
\usepackage{wrapfig}

\newcommand{\e}[1]{e^{i(#1)}}
\newcommand{\E}[1]{\times 10^{#1}}

\title{
    Worksheet \#14
    \\  \small
    PHYS 4C: Waves and Thermodynamics
    }
\author{Donald Aingworth IV}
\date{November 17, 2025}

\begin{document}
    \DeclareSIUnit{\celsiusdegree}{C^\circ}
    \DeclareSIUnit{\atm}{ atm}

    \maketitle

    \setcounter{section}{0}
    \section{Problem 1}
        (10 points) A light ray is incident upon a triangular prism (n = 1.50) as shown below.
        \begin{center}
            \includegraphics[width=0.25\textwidth]{P1.png}
        \end{center}

        \begin{enumerate}[label=\alph*)]
            \item   (5 points) Determine the direction of the transmitted light ray (i) as it passes through the prism, and (ii) as it leaves the prism. In each case, indicate the direction by giving the angle relative to horizontal.
            \item   (5 points) Assuming that the incident light is polarized parallel to the plane of incidence, determine the fraction of light that is transmitted through the prism and out the other side.
        \end{enumerate}

        \subsection{Solution (a)}
            \subsubsection{(i) Passing through}
                First find the angle from the normal at which the light ray hits the prism.
                \begin{equation}
                    \theta  =   90\unit{\degree} - (65\unit{\degree} - 20\unit{\degree})
                            =   45\unit{\degree}
                \end{equation}

                Now use Snell's Law to find the angle at which it would be in the prism.
                Assume this to be in a vacuum.
                \begin{gather}
                    n_1 \sin(\theta_1) = n_2 \sin(\theta_2)\\
                    \begin{align}
                        \theta_2    &=  \arcsin\left( \frac{n_1}{n_2}\sin(\theta_1) \right)
                            =   \arcsin\left( \frac{1}{1.50}\sin(45\unit{\degree}) \right)\\
                            &=  \arcsin\left( \frac{2}{3} * \frac{\sqrt{2}}{2} \right)
                            =   \arcsin\left( \frac{\sqrt{2}}{3} \right)
                            =   28.1255\unit{\degree}
                    \end{align}
                \end{gather}

                Convert to the angle from the horizontal. 
                \begin{equation}
                    28.1255\unit{\degree} - 25\unit{\degree} = \boxed{3.1255\unit{\degree}}
                \end{equation}

            \subsubsection{(ii) Leaving the Prism}
                First find the incident angle at which the light hits the edge.
                \begin{equation}
                    3.1255\unit{\degree} - 15\unit{\degree} = -11.8745\unit{\degree}
                \end{equation} 

                Use Snell's Law.
                \begin{align}
                    \theta_2    &=  \arcsin\left( \frac{n_1}{n_2}\sin(\theta_1) \right)
                        =   \arcsin\left( \frac{1.50}{1}\sin(11.8745\unit{\degree}) \right)\\
                        &=  17.978\unit{\degree}
                \end{align}

                Next, we return to the angle from the horizontal rather than the angle of incidence.
                \begin{equation}
                    17.978\unit{\degree} - 15\unit{\degree} = \boxed{2.978\unit{\degree}}
                \end{equation}

        \subsection{Solution (b)}
            If it is parallel to the angle of incidence, there is a formula for the fraction of the ampliude that reflects.
            \begin{equation}
                r_\parallel = \frac{n_1\cos(\theta_2) - n_2\cos(\theta_1)}{n_1\cos(\theta_2) + n_2\cos(\theta_1)}
            \end{equation}

            We can plug values into this, starting with the angle entering the prism.
            \begin{align}
                r_1 &=  \frac{\cos(28.1255\unit{\degree}) - 1.5\cos(45\unit{\degree})}{\cos(28.1255\unit{\degree}) + 1.5\cos(45\unit{\degree})}
                    =   \frac{-0.178743}{1.942577}
                    =   -0.0920\\
                R_1 &=  \left| r_1 \right|^2
                    =   0.008466\\
                T_1 &=  1 - R_1
                    =   0.9915
            \end{align}

            Then for the angle at whch it exits the prism.
            \begin{align}
                r_2 &=  \frac{1.50\cos(17.978\unit{\degree}) - \cos(11.8745\unit{\degree})}{1.50\cos(17.978\unit{\degree}) + \cos(11.8745\unit{\degree})}
                    =   \frac{0.44816}{2.40536}
                    =   0.18632\\
                R_2 &=  0.03471\\
                T_2 &=  0.96529
            \end{align}

            Multiply the two together to get the resultant ratio.
            \begin{equation}
                T_m = T_1 T_2 = \boxed{0.957}
            \end{equation}
\end{document}