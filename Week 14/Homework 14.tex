\documentclass[12pt]{article}
\usepackage{amsmath}
\usepackage{array}
% \usepackage{gensymb}
\usepackage{geometry}
\usepackage{graphicx}
\usepackage{pgfplots}
\usepackage{siunitx}
\usepackage{wrapfig}

\title{Homework \#14}
\author{Donald Aingworth IV}
\date{November 27, 2024}

\pgfplotsset{width=8cm,compat=1.9}
\usepgfplotslibrary{external}
% \tikzexternalize

\begin{document}

\DeclareSIUnit{\mile}{mi}
\DeclareSIUnit{\gal}{gal}
\DeclareSIUnit{\foot}{ft}
\DeclareSIUnit{\hour}{h}
\DeclareSIUnit{\rad}{rad}
\DeclareSIUnit{\unit}{u}
\DeclareSIUnit{\dyne}{dyn}

\maketitle

\pagebreak

\section{Problem 1}
A wheel, whose moment of inertia is 0.0300 \unit{\kilo\gram*\meter^2}, is accelerated from rest to 20.0 rad/s in 5.00 s. When the external torque is removed, the wheel stops in 1 min. Find: (a) the frictional torque; (b) the external torque.

\subsection{Solution}
\subsubsection{Section (a)}
When only the frictonal torque is active, the radial velocty goes from 20 rad/s to 0 rad/s in 60 seconds. We can find the acceleration here by using a simple angular velocity formula.
\begin{align}
    \theta  &=  \theta_0 + \alpha t \rightarrow \alpha = \frac{\theta - \theta_0}{t}\\
    \alpha  &=  \frac{0 - 20}{60} = -\frac{1}{3}\unit{\rad/\second^2}
\end{align}

We now can use the formula for torque from angular acceleration to find the solution.
\begin{equation}
    \tau_{friction} =   I \alpha 
        = 0.030 * \left(-\frac{1}{3}\right) 
        = \boxed{0.010\unit{\newton\meter}}
\end{equation}

\subsubsection{Section (b)}
\begin{align}
    \alpha  &=  \frac{\theta - \theta_0}{t}
        =   \frac{20.0 - 0.0}{5.0}
        =   4.0 \unit{\second^{-2}}\\
    \tau_{net}  &=  I\alpha
        =   0.0300 * 4.0
        =   0.120 \unit{\newton\meter}
        =   \tau_{ext} + \tau_{friction}\\
    \tau_{ext}  &=  \tau_{net} - \tau_{friction}
        =   0.120 - 0.010
        =   \boxed{0.110 \unit{\newton\meter}}
\end{align}


\pagebreak
\section{Problem 2}
A block of mass m = 2.00 kg hangs vertically from a frictionless pulley of mass M = 4.00 kg and radius R = 15.0 cm. Find: (a) the acceleration of the block; (b) the tension in the rope; (c) the speed of the block after it has fallen 40.0 cm—assuming it started at rest. Treat the pulley as a solid disk.

\subsection{Solution}
\subsubsection{Section (a)}
We can start by creating an equation for the net vertical force on the block. There are two forces: the gravitational force and the tension force.
\begin{equation} F_{net} = T + F_g = T + mg = ma \end{equation}

We can next form the net rotational force on the pulley. To do this, we could calculate the torque of the disc. To do that, we would need to calculate the inertia of the pulley, which is the moment of inertia of a disc.
\begin{equation} I = \frac{1}{2}MR^2 \end{equation}

We here have two equations for torque, one from inertia \(\tau = I\alpha\), and one from force \(\tau = Fr\sin(\theta)\), with $\theta = \pi/2$ so $\sin(\theta) = 1$ since the force is downwards, perpendicular to the radius applied. We can combine these to get a formula for the force. and use that to get the net force and tension equation. We can also use $a = r\alpha$ to get $\alpha = \frac{a}{r}$, which we can use. 
\begin{align}
    I\alpha &=  Fr\\
    F_\tau  &=  \frac{MR^2}{2R}\alpha
        =   \frac{1}{2}Ma\\
    F_{net} &=  Ma
\end{align}

We next can add equation (7) and equation (11) together to get a formula.
\begin{align}
    Ma + ma &=  \frac{1}{2}Ma - mg\\
    2Ma + 2ma   &=  Ma - 2mg\\
    (M + 2m)a   &=  -2mg\\
    a   &=  -\frac{2mg}{M + 2m} 
        =   \boxed{4.905\unit{\meter/\second^2}}
\end{align}

\subsubsection{Section (b)}
We can substitute the acceleration into a formula including tension, in this instance \(ma = T + mg\).
\begin{align}
    T   &=  ma - mg
        =   F_\tau - T
        =   \frac{2m^2g}{M + 2m} + mg
        =   \boxed{24.525\unit{\kilo\gram*\meter/\second^2}}
\end{align}

\subsubsection{Section (c)}
We have the acceleration and the change in position. We can apply this to a kinematic formula to get a final velocity.
\begin{align}
    v^2 &=  v_0^2 + 2a\Delta y
        =   \frac{4mg}{M + 2m}\Delta y\\
    v   &=  \sqrt{\frac{4mg}{M + 2m}\Delta y}
        =   \boxed{1.981\unit{\meter/\second^2}}
\end{align}

\pagebreak
\section{Problem 3}
\begin{wrapfigure}{r}{0.35\textwidth}
    \vspace{-30pt}
    \includegraphics[width=0.35\textwidth]{graph_3.png} 
    % \label{fig:wrapfig}
\end{wrapfigure}
A block of mass m = 2.00 kg can slide down a frictionless 53\unit{\degree} incline, but it is connected to a pulley of mass M = 4.00 kg and radius R = 0.500 m, as shown in the figure below. The pulley can be treated as a disk. Find: (a) the angular acceleration of the pulley; (b) the speed of the block after it has slid 1.00 m, starting from rest.

\subsection{Solution}
\subsubsection{Section (a)}
First, there is the set of formulas for the torque, that we can combine. Since the pulley can be treated as a disc, we can use its formula for the moment of inertia \(I = \frac{1}{2}MR^2\).
\begin{align}
    I\alpha &=  F R\\
    F_\tau  &=  \frac{I\alpha}{R}
        =   \frac{1}{2}MR\alpha
\end{align}

This can be applied to two equations, both derived from Newton's second law.
\begin{align}
    F_{net} &=  ma\\
    Ma  &=  F_\tau - T
        =   \frac{1}{2}MR\alpha - T\\
    ma  &=  T - F_g
        =   T - mg\sin(\theta)
\end{align}

Add equations (20) and (21) together. We can also recal that \(a = \alpha r\).
\begin{align}
    Ma + ma &=  \frac{1}{2}MR\alpha - mg\sin(\theta)\\
    2MR\alpha + 2mR\alpha   &=  MR\alpha - 2mg\sin(\theta)\\
    \alpha (MR + 2mR)   &=  -2mg\sin(\theta)\\
    \alpha  &=  -\frac{2mg\sin(\theta)}{MR + 2mR}
        =   -\frac{2*2*9.81*\sin(53\unit{\degree})}{4*0.5 + 2*2*0.5}\\
        &=  -\frac{39.24*\sin(53\unit{\degree})}{4}
        =   -9.81*0.7986 \unit{\second^{-2}}\\
        &=  \boxed{-7.835 \unit{\second^{-2}}}
\end{align}

\subsubsection{Section (b)}
We can use the equation $a = r\alpha$ and the kinematic equation $v^2 = v_0^2 + 2a\Delta x$.
\begin{align}
    a   &=  r\alpha 
        =   0.500 * -7.835
        =   3.9173 \unit{\meter/\second^2}\\
    v^2 &=  v_0^2 + 2a\Delta x
        =   0 + 2 * (-3.9173) * (-1)
        =   7.8346 \unit{\meter^2/\second^2}\\
    v   &=  \sqrt{7.8346}
        =   \boxed{2.799 \unit{\meter/\second}}
\end{align}


\pagebreak
\section{Problem 4}
A car is traveling at 75.0 km/h has tires of 70.0 cm diameter. (a) What is the angular velocity of the tires about their axles? (b) If the car is brought to a stop uniformly in 30.0 complete turns of the tires (without skidding), what is the magnitude of the angular acceleration of the wheels? (c) How far does the car move during the braking?

\subsection{Solution}
\subsubsection{Section (a)}
\begin{align}
    75\unit{\kilo\meter/\hour}  &=  75 \unit{\kilo\meter/\hour} * \frac{10\unit{\meter/\second}}{36\unit{\kilo\meter/\hour}}
        =   \frac{375}{18}\unit{\meter/\second}
        =   \frac{125}{6}\unit{\meter/\second}\\
    v   &=  r \omega\\
    \omega  &=  \frac{v}{r}
        =   \frac{125}{6*0.35}\unit{\second^{-1}}
        =   \frac{1250}{21}\unit{\second^{-1}}
        =   \boxed{59.524\unit{\second^{-1}}}
\end{align}

\subsubsection{Section (b)}
\begin{align}
    \omega^2    &=  \omega_0^2  +   2\alpha \Delta\theta\\
    \alpha  &=  -\frac{\omega_0^2}{2\Delta\theta}
        =   -\frac{\left(\frac{1250}{21}\unit{\second^{-1}}\right)^2}{2*60\pi}
        =   -\frac{1250^2}{42^2*120\pi}
        =   -9.398 \unit{\second^{-2}}
\end{align}
This makes the magnitude \boxed{9.398 \unit{\second^{-2}}}.

\subsubsection{Section (c)}
Since the car moves only 30 turns, we can just convert from turns to meters.
\begin{align}
    x   &=  r*\theta
        =   0.35\unit{\meter}*2\pi*30
        =   \boxed{65.9734\unit{\meter}}
\end{align}


\pagebreak
\section{Problem 5}
\begin{wrapfigure}{r}{0.35\textwidth}
    \vspace{-50pt}
    \includegraphics[width=0.35\textwidth]{graph_5.png} 
    % \label{fig:wrapfig}
\end{wrapfigure}
The graph below shows the speed v versus time t for a 0.500 kg object of radius 6.50 cm that rolls smoothly down a 30\unit{\degree} ramp. The scale on the velocity axis is set by $v_s = 4.0$ m/s. What is the rotational inertia of the object?

\subsection{Solution}
We can see that at 1s, the velocity is around $\frac{7}{2}$m/s from rest at t = 0s, so the acceleration is easily to calculate.
\begin{equation}
    a_{com} =   \frac{\Delta v}{\Delta t} 
        = \frac{\frac{7}{2}\unit{\meter/\second}}{1\unit{\second}} 
        = \frac{7}{2}\unit{\meter/\second^2}
\end{equation}

The gravitiational force is directed downwards, which can be divided into the components into the the ramp and down the ramp. The downhill force contributes to the net force. Meanwhile, the force into the ramp contributes to the frictional force $f_s$, which we can isolate.
\begin{align}
    F_{net} &=  Ma_{com}\\
    Ma_{com}    &=  F_{g;x} - f_s
        =   Mg\sin(\theta) - f_s\\
    f_s &=  M(g\sin(\theta) - a_{com})
\end{align}

We also know that since the frictional force acts on an edge of the object, it has a torque, which we can use our favorite combination of equations to express. We can then isolate the inertia and replace the rotational acceleration with the translational acceleration.
\begin{align}
    \alpha  &=  \frac{a_{com}}{R}\\
    f_s R   &=  I\alpha
        =   I\frac{a_{com}}{R}\\
    I   &=  \frac{f_s R^2}{a_{com}}
        =   \frac{MR^2(g\sin(\theta) - a_{com})}{a_{com}}
\end{align}

We can substitute known values into here to find the value of the inertia.
\begin{align}
    I   &=  \frac{MR^2(g\sin(\theta) - a_{com})}{a_{com}}
        =   0.50*0.065^2*(9.81*\sin(30\unit{\degree}) - \frac{7}{2})*\frac{2}{7}\\
        &=  \boxed{8.48 \times 10^{-4}\unit{\kilo\gram*\meter^2}}
\end{align}


% This is the acceleration of the center of mass. From these, we can find the angular velocity and acceleration.
% \begin{align}
%     \omega  &=  \frac{v_{com}}{R}
%         =   \frac{7}{2}*\frac{1000}{65}
%         =   \frac{700}{13}\unit{\second^{-1}}\\
%     \alpha  &=  \frac{a_{com}}{R}
%         =   \frac{7}{2}*\frac{1000}{65}
%         =   \frac{700}{13}\unit{\second^{-2}}
% \end{align}

% We know that the net torque equations, whih we can set equal to each other and apply equation (40), then solve for the inertia.
% \begin{align}
%     I\alpha &=  F*R*\sin(\phi)\\
%     I   &=  \frac{F*R*\sin(\phi)}{\alpha}
%         =   \frac{F*R^2*\sin(\phi)}{a_{com}}
% \end{align}

% We then should know that the force in question is the frictional force, which acts on the bottom of the object. It is equal to the force resultant from the acceleration downhill, which we know from the above. We also keep Newton's second law in mind to substitute in for the force and keep in mind that since the force is applied perpendicular to the radius and tangential to the object, \(\sin(\phi) = 1\).
% \begin{align}
%     F   &=  Ma\\
%     I   &=  \frac{M*a_{com}*R^2}{a_{com}}
%         =   M*R^2
% \end{align}


\pagebreak
\section{Problem 6}
A hollow sphere of radius 0.15 m, with rotational inertia I = 0.040 \unit{\kilo\gram*\meter^2} about a line through its center of mass, rolls without slipping up a surface inclined at 30\unit{\degree} to the horizontal. At a certain initial position, the sphere's total kinetic energy is 20 J. (a) How much of this initial kinetic energy is rotational? (b) What is the speed of the center of mass of the sphere at the initial position? When the sphere has moved 1.0 m up the incline from its initial position, what are (c) its total kinetic energy and (d) the speed of its center of mass?

\subsection{Solution}
\subsubsection{Section (a)}
We have our equation about rotational kinetic energy. We can adjust it to find our ratio that we need to find. 
\begin{equation}
    K   =   \frac{1}{2}mv^2 + \frac{1}{2}I\omega^2
    \rightarrow \frac{I\omega^2}{2K}
\end{equation}

We next have to find the angular velocity $\omega$. We can translate this into the translational velocity by using \(v = r\omega\). 
\begin{align}
    K   &=  \frac{v^2}{2}\left(m + \frac{I}{r^2}\right)
\end{align}

From the book, the formula for the inertia of a hollow sphere of radius R is \(I = \frac{2}{3}mr^2\), so \(m = \frac{3I}{2r^2}\). We can then translate back for \(\omega =\frac{v}{r}\).
\begin{equation}
    K   =   \frac{5}{2}*\frac{1}{2}*\frac{v^2}{r^2}I
        =   \frac{5}{2}*\frac{1}{2}I\omega^2
\end{equation}

From here, we know that the rotational kinetic energy is \(K_R = \frac{1}{2}I\omega^2\).
\begin{align}
    K   &=  \frac{5}{2}K_R\\
    \frac{K_R}{K} &= \frac{2}{5} = 0.4
\end{align}
This means that \boxed{40\%} of the kinetic energy is rotational. 

\subsubsection{Section (b)}
We have above an equation for the kinetic energy from only the velocity, radius, and inertia in the form of the first half of (49). We use that here. 
\begin{align}
    K   &=  \frac{5}{2}*\frac{1}{2}*\frac{v^2}{r^2}I
        =   \frac{5I}{4r^2}v^2\\
    v^2 &=  \frac{4r^2}{5I}K\\
    v   &=  \sqrt{\frac{4r^2}{5I}K}
        =   \sqrt{\frac{4*0.15^2}{5*0.04}*20}
        =   \frac{0.15}{0.2}\sqrt{16}
        =   \frac{3}{4}*4
        =   \boxed{3\unit{\meter/\second}}
\end{align}

\subsubsection{Section (c)}
If the ball moves up 1 m, the kinetic energy will decrease by change in gravitational potential energy. The change in position is equal to the distance times the sine of the angle of the incline, so \(\Delta y = 1\unit{\meter} * \sin(30\unit{\degree}) = \frac{1}{2}\unit{\meter}\). We can find the ending kinetic energy here.
\begin{align}
    K_f &=  K_i - W
        =   K_i - mg\Delta y\\
        &=  20\unit{\joule} - \frac{9.81}{2}m
        =   20\unit{\joule} - \frac{9.81}{2}*\frac{3I}{2r^2}\\
        &=  20\unit{\joule} - \frac{9.81}{2}*\frac{3*0.04}{2*0.0225}\\
        &=  (20 - 13.08)\unit{\joule}
        =   \boxed{6.92\unit{\joule}}
\end{align}

\subsubsection{Section (4)}
Here, we use the same strategy as in part (b). We also knwo that the formula for the inerita fro a hollow sphere is \(I = \frac{2}{3}mr^2\) 
\begin{align}
    K   &=  \frac{1}{2}mv^2 + \frac{1}{2}I\omega^2
        =   \frac{1}{2}m\omega^2 r^2 + \frac{1}{2}I\omega^2
        =   \frac{1}{2}*\frac{3}{2}*\frac{2}{3}mr^2*\omega^2 + \frac{1}{2}I\omega^2\\
        &=  \frac{3}{4}I\omega^2 + \frac{1}{2}I\omega^2
        =   \frac{5}{4}I\omega^2
        =   \frac{5}{4}I\frac{v^2}{r^2}\\
    v^2 &=  \frac{4r^2}{5I^2}K\\
    v   &=  \sqrt{\frac{4r^2}{5I}K}
        =   \sqrt{\frac{4*0.15^2}{5*0.04}K}
        =   \frac{3}{4}\sqrt{\frac{4}{5}*6.92}
        =   \boxed{1.77\unit{\meter/\second}}
\end{align}


\pagebreak
\section{Problem 7}
A yo-yo has a rotational inertia of 900 \unit{\gram*\centi\meter^2} and a mass of 100 g. Its axle radius is 3.2 mm, and its string is 120 cm long. The yo-yo rolls from rest down to the end of the string. (a) What is the magnitude of its linear acceleration? (b) How long does it take to reach the end of the string? As it reaches the end of the string, what are its (c) linear speed, (d) translational kinetic energy, (e) rotational kinetic energy, and (f) angular speed?

\subsection{Solution}
\subsubsection{Section (a)}
We can first list the forces acting on the yo-yo. There is a force of gravity acting on the center of mass, and a tension force acting on the edge of the cylinder. We can generate a torque formula from this and find an acceleration thereof. The cylinder would be rolling clockwise, so in this instance, \(\alpha = -\frac{a}{r}\). 
\begin{align}
    F_{net} = ma
        &=  T - mg\\
    I\alpha &=  Fr
        =   Tr\\
    T   &=  I\frac{\alpha}{r}
        =   -I\frac{a}{r^2}\\
    ma  &=  -I\frac{a}{r^2} - mg\\
    a\left(m + \frac{I}{r^2}\right) &=  -mg\\
    a   &=  -\frac{mg}{m + \frac{I}{r^2}}
        =   -\frac{981\unit{\centi\meter/\second^2}*100\unit{\gram}}{100\unit{\gram} + \frac{900\unit{\gram*\centi\meter^2}}{(0.32\unit{\centi\meter})^2}}\\
        &=  -\frac{98100\unit{\centi\meter/\second^2}}{100 + \frac{900\unit{\gram*\centi\meter^2}}{0.1024\unit{\gram*\centi\meter^2}}}
        =   -\frac{62784}{5689}\unit{\centi\meter/\second^2} \approx -11.04\unit{\centi\meter/\second^2}
\end{align}
So, the magnitude is \boxed{11.04\unit{\centi\meter/\second^2}}.

\subsubsection{Section (b)}
The time it would take would be calculatable if we used the kinematic equation that does not require time to find the final linear velocity, and used that to find the time.
\begin{align}
    v_f^2   &=  v_i^2 + 2a\Delta y
        =   0^2 + 2*-\frac{62784}{5689}\unit{\centi\meter/\second^2}*(-120\unit{\centi\meter})\\
        &=  \frac{15868160}{5689}\unit{\centi\meter^2/\second^2}
        =   2648.6\unit{\centi\meter^2/\second^2}\\
    v_f &=  51.465\unit{\centi\meter/\second}\\
    v_f &=  v_0 + at\\
    t   &=  \frac{v_f}{a}
        =   \frac{51.465\unit{\centi\meter/\second}}{11.04\unit{\centi\meter/\second^2}}
        =   \boxed{4.6634\unit{\second}}
\end{align}

\subsubsection{Section (c)}
The answer comes from part (b). The answer is \boxed{v_f =  51.465\unit{\centi\meter/\second}}.

\subsubsection{Section (d)}
The translational kinetic energy is going to be \(K_{trans} = \frac{1}{2}mv^2\).
\begin{align}
    K_{trans}   &=  \frac{1}{2}mv^2
        =   \frac{1}{2}100\unit{\gram}*\frac{15868160}{5689}\unit{\centi\meter^2/\second^2}\\
        &=  \boxed{\frac{753408000}{5689}\unit{\dyne*\centi\meter} \approx 1.1324\times10^5\unit{\dyne*\centi\meter}}
\end{align}

\subsubsection{Section (e)}
Since the initial velocity (linear and rotational) is zero, the initial total kinetic energy is zero. This means that we can calculate the final kinetic energy by the work done by gravity.
\begin{align}
    W_g &=  mg\Delta y
        =   100\unit{\gram}*981\unit{\centi\meter/\second^2}*120\unit{\centi\meter}\\
        &=  1.1772\times10^7\unit{\dyne*\centi\meter}
\end{align}

We can subtract the linear kinetic energy, which we calculated, to get the remaining (rotational) kinetic energy.
\begin{align}
    K_{rot} &=  W_g - K_{trans}
        =   1.1772\times10^7\unit{\dyne*\centi\meter} - 1.1324\times10^5\unit{\dyne*\centi\meter}\\
        &=  \boxed{1.164\times10^7\unit{\dyne*\centi\meter}}
\end{align} 

\subsubsection{Section (f)}
We can use the formula for rotational kinetic energy \(K = \frac{1}{2}I\omega^2\).
\begin{align}
    K   &=  \frac{1}{2}I\omega^2\\
    \omega^2    &=  \frac{2K}{I} \rightarrow \omega = \sqrt{\frac{2K}{I}}\\
    \omega  &=  \sqrt{\frac{2*1.164\times10^7\unit{\dyne*\centi\meter}}{900\unit{\gram*\centi\meter^2}}}
        =   \boxed{160.8\unit{\second^{-1}}}
\end{align}


\pagebreak
\section{Problem 8}
\begin{wrapfigure}{r}{0.35\textwidth}
    \vspace{-30pt}
    \includegraphics[width=0.35\textwidth]{graph_8.png} 
    % \label{fig:wrapfig}
\end{wrapfigure}
A uniform rod is held vertically by two strings of negligible mass, as shown below. (a) Immediately after the string is cut, what is the linear acceleration of the free end of the stick? (b) Of the middle of the stick?

\subsection{Solution}
\subsubsection{Section (a)}
The linear acceleration of the free end of the stick can be calculated using the moment of inertia and the gravitational force and the radius. The radial arm is going to be the distance to the center of mass, or $r = \frac{L}{2}$.
\begin{align}
    Fr  &=  I\alpha\\
    \alpha  &=  \frac{Fr}{I}
        =   \frac{mgr}{I}
        =   \frac{mgL}{2I}\\
    a   &=  \frac{mgL^2}{2I}
\end{align}

The moment of inertia can be found by using the parallel axis theorem on the inertia of the rod, where h is half the length of the rod.
\begin{align}
    I_{com} &=  \frac{1}{12}ML^2\\
    I   &=  I_{com} + Mh^2
        =   I_{com} + \frac{1}{4}ML^2
        =   I_{com} + 3I_{com}
        =   4I_{com}
        =   \frac{1}{3}ML^2\\
    a   &=  \frac{MgL^2}{2I}
        =   \frac{gML^2}{\frac{2}{3}ML^2}
        =   \boxed{\frac{3g}{2}}
\end{align}

\subsubsection{Section (b)}
The center of mass is half the distance between the start and end of the rod. As such, with Thales's theorem, the acceleration would be half that of the end of the rod. As such, \boxed{a_{center} = \frac{3g}{4}}

\end{document}