\documentclass[12pt]{article}
\usepackage{amsmath}
\usepackage{array}
% \usepackage{gensymb}
\usepackage{geometry}
\usepackage{graphicx}
\usepackage{pgfplots}
\usepackage{siunitx}
\usepackage{wrapfig}

\title{Homework \#14}
\author{Donald Aingworth IV}
\date{November 27, 2024}

\pgfplotsset{width=8cm,compat=1.9}
\usepgfplotslibrary{external}
% \tikzexternalize

\begin{document}

\DeclareSIUnit{\mile}{mi}
\DeclareSIUnit{\gal}{gal}
\DeclareSIUnit{\foot}{ft}
\DeclareSIUnit{\h}{h}
\DeclareSIUnit{\rad}{rad}
\DeclareSIUnit{\u}{u}

\maketitle

\pagebreak

\section{Problem 1}
A wheel, whose moment of inertia is 0.0300 \unit{\kilo\gram*\meter^2}, is accelerated from rest to 20.0 rad/s in 5.00 s. When the external torque is removed, the wheel stops in 1 min. Find: (a) the frictional torque; (b) the external torque.

\subsection{Solution}


\pagebreak

\section{Problem 2}
A block of mass m = 2.00 kg hangs vertically from a frictionless pulley of mass M = 4.00 kg and radius R = 15.0 cm. Find: (a) the acceleration of the block; (b) the tension in the rope; (c) the speed of the block after it has fallen 40.0 cm—assuming it started at rest. Treat the pulley as a solid disk.

\subsection{Solution}


\pagebreak

\section{Problem 3}
A block of mass m = 2.00 kg can slide down a frictionless 53\unit{\degree} incline, but it is connected to a pulley of mass M = 4.00 kg and radius R = 0.500 m, as shown in the figure below. The pulley can be treated as a disk. Find: (a) the angular acceleration of the pulley; (b) the speed of the block after it has slid 1.00 m, starting from rest.


\subsection{Solution}


\pagebreak

\section{Problem 4}

\subsection{Solution}


\pagebreak

\section{Problem 5}

\subsection{Solution}


\pagebreak

\section{Problem 6}

\subsection{Solution}


\pagebreak

\section{Problem 7}

\subsection{Solution}


\pagebreak

\section{Problem 8}

\subsection{Solution}


\end{document}