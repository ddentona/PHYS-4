\documentclass[12pt]{article}
\usepackage{amsmath}
\begin{document}
    \begin{center}
        \textbf{A Document Detailing the Derivation of Magnetic Field from a Solenoid along its Axis}\\
        Author: Think the Duck
    \end{center}
    This is a document meant for the sole purpose of deriving the formula for the total magnetic field from the circular center of and on the axis perpendicular to a solenoid at a distance relative to one edge of the solenoid.
    We will be doing this in cartesian coordinates and by only using one axis.
    The derivation of the magnetic field from a ring of current can be found on a different document.

    For the sake of this, we will be using Biot-Savart's Law.
    \begin{equation}
        d\vec{B} = \frac{\mu_0 I}{4\pi} \cdot \frac{d\vec{s} \times \hat{r}}{r^2}
    \end{equation}

    Biot-Savart's law indicates a requirement of existence of a vector $d\vec{s}$ to travel along the solenoid.
    We can assume that the solenoid encircles (part of) the $z$-axis, and we can assume that one end of it is on the $xy$-plane. 
    The sollenoid would have a radius of $R$ and would travel in circles.
    We will be assuming that the sollenoid's position of each point rotates clockwise as the $z$-position increases.
    This can give us values for the $x$ and $y$ components of $\vec{s}$.
    \begin{gather}
        s_x = R \cdot \cos(t)\\
        s_y = R \cdot \sin(t)
    \end{gather}

    The solenoid of length $L$ would be composed of $N$ loops of equal length.
    This would mean that each loop would be of length $\frac{L}{N}$.
    Since we would be traveling away from the point we are measusing at, the cange in position along each loop would be $-\frac{L}{N}$.
    We are already traversing the solenoid using $t$, in units of radians (which technically are unitless, to be fair).
    Since one loop travels a distance of $-\frac{L}{N}$ whenever it completes one loop of $2\pi$ radians, we can combine these for $s_z$.
    This can give us a complete version of $\vec{s}$.
    \begin{gather}
        s_z = -\frac{L \cdot t}{N \cdot 2\pi}\\
        \vec{s} =   \begin{pmatrix}
            R \cdot \cos(t)\\
            R \cdot \sin(t)\\
            -\frac{L \cdot t}{N \cdot 2\pi}
        \end{pmatrix}
    \end{gather}

    We can then calculate $d\vec{s}$.
    \begin{gather}
        d\vec{s}    =   \begin{pmatrix}
                -R \cdot \sin(t) dt\\
                R \cdot \cos(t) dt\\
                -\frac{L}{N \cdot 2\pi} dt
            \end{pmatrix}
            =   \begin{pmatrix}
                -R \cdot \sin(t)\\
                R \cdot \cos(t)\\
                -\frac{L}{N \cdot 2\pi}
            \end{pmatrix} dt
    \end{gather}

    The value of $\vec{r}$ can be calculated by taking the negative of $\vec{s}$ and adding the distance from the origin to the position we are aiming for the magnetic field at.
    We can call the position at which we will be calculating the magnetic field $\vec{P}$ and assuming it to be a distance $z$ from the origin on the $z$-axis.
    \begin{equation}
        \vec{r} =   -\vec{s} + \vec{P} 
            =   \begin{pmatrix}
                -R \cdot \cos(t)\\
                -R \cdot \sin(t)\\
                \frac{L \cdot t}{N \cdot 2\pi}
            \end{pmatrix} + 
            \begin{pmatrix} 0\\ 0\\ z \end{pmatrix}
            =   \begin{pmatrix}
                -R \cdot \cos(t)\\
                -R \cdot \sin(t)\\
                \frac{L \cdot t}{N \cdot 2\pi} + z
            \end{pmatrix}
    \end{equation}

    From this, we can calculate $d\vec{s} \times \vec{r}$.
    \begin{align}
        d\vec{s} \times \vec{r} &=  \begin{pmatrix}
                -R \cdot \sin(t)\\
                R \cdot \cos(t)\\
                -\frac{L}{N \cdot 2\pi}
            \end{pmatrix} dt \times \begin{pmatrix}
                -R \cdot \cos(t)\\
                -R \cdot \sin(t)\\
                \frac{L \cdot t}{N \cdot 2\pi} + z
            \end{pmatrix}\\
            &=  \det \begin{bmatrix}
                \hat{i} &   \hat{j} &   \hat{k}\\
                -R \cdot \sin(t)    &   R \cdot \cos(t) &   -\frac{L}{N \cdot 2\pi}\\
                -R \cdot \cos(t)    &   -R \cdot \sin(t)&   \frac{L \cdot t}{N \cdot 2\pi} + z
            \end{bmatrix} dt\\
            &=  \begin{pmatrix}
                    R\cos(t)\left(\frac{L*t}{N*2\pi} + z\right) - R\sin(t)\frac{L}{N*2\pi}\\
                    \cos(t)\frac{L*R}{N*2\pi} + R\sin(t)\left(\frac{L*t}{N*2\pi} + z\right)\\
                    R^2
                \end{pmatrix}\\
            &=  \begin{pmatrix}
                    \frac{RL}{2\pi N} \left(t * \cos(t) - \sin(t)\right) + Rz\cos(t)\\
                    \frac{RL}{2\pi N} \left(t * \sin(t) + \cos(t)\right) + Rz\sin(t)\\
                    R^2
                \end{pmatrix}
    \end{align}
    
    The $z$-axis is easiest and the one that cancels itself out the least, so we can compute that.
    \begin{equation}
        B_z =   \frac{\mu_0 I}{4\pi}\int_{0}^{N*2\pi} \frac{R^2}{\left(R^2 + \left(\frac{L*t}{N*2\pi} + z\right)^2\right)^{3/2}}dt
    \end{equation}

    Using substitution, we can have a set of three usable values to insert into this.
    \begin{gather}
        a = \frac{L*t}{N*2\pi} + z\\
        \frac{da}{dt} = \frac{L}{N*2\pi}\\
        dt  =   \frac{N*2\pi}{L} da
    \end{gather}

    As a reminder (I needed this myself), integration substitution is covered in the following way.
    It derives from the formula of substitution in derivatives.
    \begin{equation}
        \int_{u(a)}^{u(b)} f'(x) dx  =   \int_{a}^{b} f'(u(y)) \cdot u'(y) dy
    \end{equation}

    We can apply that to this.
    \begin{align}
        B_z &=  \frac{\mu_0 I}{4\pi} \int_{z}^{L + z} \frac{R^2}{\left(R^2 + a^2\right)^{3/2}} \frac{N * 2\pi}{L} da\\
            &=  \frac{\mu_0 I}{2} \frac{N}{L} \int_{z}^{L + z} \frac{R^2}{\left(R^2 + a^2\right)^{3/2}} da
    \end{align}

    We can then in turn apply trigonometric substitution here, for a triangle of legs of length $R$ and $a$. 
    For some context, our angle we would be observing would be between the $xy$-plane at the $z$-position we are testing and the vector from $\vec{P}$ to $d\vec{s}$.
    We can give our starting and ending angles values of $\theta_1$ and $\theta_2$, respectively. 
    There are three trigonometric values we will consider.
    \begin{gather}
        \cos(\theta) = \frac{R}{\sqrt{R^2 + a^2}}\\
        a = R*\tan(\theta)\\
        da = R*\sec^2(\theta) dt
    \end{gather}

    This in turn gives an equation for the magnetic field along the z-axis.
    \begin{align}
        B_z &=  \frac{\mu_0 I N}{2 L} \int_{\theta_1}^{\theta_2} \frac{\cos^3(\theta)}{R} R * \sec^2(\theta) d\theta\\
            &=  \frac{\mu_0 I N}{2 L} \int_{\theta_1}^{\theta_2} \frac{R*\cos^3(\theta)}{R*\cos^2(\theta)} d\theta\\
            &=  \frac{\mu_0 I N}{2 L} \int_{\theta_1}^{\theta_2} \cos(\theta) d\theta\\
            &=  \frac{\mu_0 I N}{2 L} \left[\sin(\theta)\right]_{\theta_1}^{\theta_2}\\
            &=  \frac{\mu_0 I N}{2 L} \left( \sin(\theta_2) - \sin(\theta_1) \right)
    \end{align}

    This gives us our z-value and is indeed the answer we were looking for.
    The other two axes have every term involve heavy amounts of sines and cosines and a range that reaches around the circle. 
    We can assume them to cancel themselves out.
    This document has fulfilled its sole purpose.
\end{document}