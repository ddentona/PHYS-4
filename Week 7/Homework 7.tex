\documentclass[12pt]{article}
\usepackage{amsmath}
\usepackage{array}
% \usepackage{gensymb}
\usepackage{geometry}
\usepackage{graphicx}
\usepackage{pgfplots}
\usepackage{siunitx}
\usepackage{wrapfig}

\title{Homework \#7}
\author{Donald Aingworth IV}
\date{October 9, 2024}

\pgfplotsset{width=8cm,compat=1.9}
\usepgfplotslibrary{external}
% \tikzexternalize

\begin{document}

\DeclareSIUnit{\mile}{mi}
\DeclareSIUnit{\gal}{gal}
\DeclareSIUnit{\foot}{ft}
\DeclareSIUnit{\h}{h}

\maketitle
\pagebreak

\section*{Problem 1}
A 3.15-kg block is acted on by a 24.0-N force that acts at 37.0\unit{\degree} below the horizontal, as shown in the figure. Take $\mu_k$ = 0.200 and $\mu_s$ = 0.500. (a) Does the block move if it is initially at rest? (b) If it is initially moving to the right, what is the blocks acceleration?

\begin{center}
    \includegraphics*[width=10cm]{graph_1.png}
\end{center}

\subsection*{Solution}
\subsubsection*{Section (a)}
For the block to move, it must be that $F_x > f_s$.
\begin{align*}
    F_x &= F_{app,x} = F_{app}*\cos(\theta) = 24\unit{\newton}*\cos(37\unit{\degree}) = 19.167 \unit{\newton}\\
    f_s &= \mu_s * F_N = \mu_s * (F_g + F_{app,y}) 
        = \mu_s * (m*g + F_{app}*\sin(\theta))\\
        &= 0.50 * (3.15\unit{\kilo\gram}*9.81\unit{\meter/\second^2} + 24\unit{\newton}*\sin(37\unit{\degree})) = 22.672\unit{\newton}
\end{align*}
Since $22.672\unit{\newton} > 19.167 \unit{\newton}$, $F_x < f_s$, so \underline{it does not move}.

\subsubsection*{Section (b)}
\begin{align*}
    F_{net} &= m*a = F_{x} - f_k = F_{app}\cos(\theta) - \mu_k F_N\\
    a &= \frac{F_{app}*\cos(\theta) - \mu_k * (m*g + F_{app}*\sin(\theta))}{m}\\
        &= \frac{19.167 \unit{\newton} - 0.20 * (3.15\unit{\kilo\gram}*9.81\unit{\meter/\second^2} + 24\unit{\newton}*\sin(37\unit{\degree})))}{3.15\unit{\kilo\gram}}\\
        &= \boxed{3.206\unit{\meter/\second^2}}
\end{align*}

\pagebreak
\section*{Problem 2}
A block is released at the top of a 25\unit{\degree} incline. Determine the coefficient of kinetic friction given that it slides 2.30 m in 3.15 s.

\subsection*{Solution}
Here we merely use the formulae for the force in different directions and solve for $\mu_k$.
\begin{align*}
    F_N &= F_{gy} = F_g \cos(\theta)\\
    \Sigma F_x &= ma = F_g \sin(\theta) - F_N \mu_k\\
    a   &= \frac{F_g \sin(\theta) - F_N \mu_k}{m}\\
    \Delta x &= \frac{1}{2}at^2 &= \frac{t^2(mg \sin(\theta) - mg \mu_k)}{2m}\\
    \frac{2\Delta r}{t^2} &= g\cos(\theta) - g\sin(\theta)\mu_k\\
    g\sin(\theta)\mu_k &= \frac{g\cos(\theta)t^2 - 2\Delta x}{t^2}\\
    \mu_k &= \frac{g\cos(\theta)t^2 - 2\Delta x}{g\sin(\theta)t^2} = \boxed{0.414}
\end{align*}

\pagebreak
\section*{Problem 3}
A circular off ramp has a radius of 57.0 m and a posted speed limit of 50.0 km/h. If the road is horizontal, what is the minimum coefficient of friction required?

\subsection*{Solution}
The maximum coefficient would require the centripital force to be equal to the static frictional force. 
\begin{align*}
    F_c &= f_k\\
    \frac{mv^2}{r} &= \mu_s F_N\\
    \mu_s   &= \frac{mv^2}{mgr} = \frac{v^2}{gr} = \frac{\frac{125}{9}^2}{57*9.81} 
            = \boxed{0.345}
\end{align*}

\pagebreak
\section*{Problem 4}
A car travels at speed v around a frictionless curve of radius r that is banked at an angle $\theta$ to the horizontal. Show that the proper angle of banking is given by $\tan(\theta) = \frac{v^2}{rg}$. (Hint, this is easier if you don't rotate the coordinate system like most other incline problems, and treat the x-axis as the horizontal direction, and the y-axis as the vertical direction. This is because the centripetal force is horizontal.)

\subsection*{Solution}
We start by noticing that the horizontal component of the normal force is equal to $F_N\sin(\theta)$ and the centripital force. We also note that the vertical force from the normal force is equal to $F_N\sin(\theta)$ and the gravitational force.
\begin{eqnarray*}
    F_c = F_N\sin(\theta)\\
    mg = F_N\cos(\theta)\\
    F_N = \frac{mg}{\cos(\theta)}\\
    F_c = \frac{mv^2}{r} = F_N\sin(\theta)\\
    \frac{mv^2}{r} = mg\frac{\sin(\theta)}{\cos(\theta)}\\
    \boxed{\tan(\theta) = \frac{v^2}{gr}}
\end{eqnarray*}

\pagebreak
\section*{Problem 5}
A button is at the rim of a turntable of radius 15.0 cm rotating at 45.0 rpm. What is the minimum coefficient of friction needed for it to stay on?

\subsection*{Solution}
We start by converting to full SI units. 15 cm $\rightarrow$ 0.15 m. 45 rpm $\rightarrow$ 0.75 rps. We then find the centripital velocity in meters per second and then find the acceleration. 
\begin{align*}
    v   &= \text{circumference} * \text{rate} 
        = 2\pi r * \text{rate} 
        = 0.30\pi * 0.75 \unit{\meter/\second}\\
        &= \pi 0.225 \unit{\meter/\second} 
        = 0.7068 \unit{\meter/\second}\\
    a   &= \frac{v^2}{r} 
        = \frac{0.7068^2}{0.15} 
        = 3.33099\unit{\meter/\second^2}
\end{align*}

We can then apply the static frictional force and Newton's second law to this.
\begin{align*}
    f_s     &= \mu_s F_N 
            = \mu_s mg\\
    F_{net} &= ma\\
    \mu_s mg &= ma\\
    \mu_s   &= \frac{a}{g} 
            = \frac{3.33099\unit{\meter/\second^2}}{9.81\unit{\meter/\second^2}}
            = \boxed{0.340}
\end{align*}

\pagebreak
\section*{Problem 6}
A box is dropped onto a conveyor belt moving at 3.40 m/s. If the coefficient of friction between the box and the belt is 0.270, how long will it take before the box moves without slipping?

\subsection*{Solution}
We begin wth the fully expanded formula for the maximum force of static friction and the application to Newton's second law.
\begin{align*}
    F_N &= \mu_s mg\\
    ma &= \mu_s mg\\
    a &= \mu_s g
\end{align*}

We can next apply the kinematic equation to solve for time and substitute in values.
\begin{align*}
    v = v_0 + at &\rightarrow t = \frac{v - v_0}{a} = \frac{v - v_0}{\mu_s g} 
                                = \frac{3.40 - 0}{0.270*9.81} \unit{\second}
                                = \boxed{1.28 \unit{\second}}
\end{align*}

\pagebreak
\section*{Problem 7}
Two blocks are stacked as shown below, and rest on a frictionless surface. There is friction between the two blocks with a coefficient of friction $\mu_s$. An external force is applied to the top block at an angle $\theta$ with the horizontal. What is the maximum force F that can be applied for the two blocks to move together?

\begin{center}
    \includegraphics*[width=10cm]{graph_7.png}
\end{center}

\subsection*{Solution}
\begin{align*}
    f_s &= F_{app,x}\\
    F_N \mu_s &= F_{app}\cos(\theta)\\
    F_{app}\cos(\theta) &= \mu_s * mg + F_{app}\sin(\theta)\mu_s\\
    F_{app} &= \boxed{\frac{\mu_s mg}{\cos(\theta) - \sin(\theta)\mu_s}}
\end{align*}

\end{document}