\documentclass[12pt]{article}
\usepackage{amsmath}
\usepackage{amssymb}
\usepackage{cancel}
\usepackage{enumitem}
\usepackage{esdiff}
\usepackage{graphicx}
\usepackage{siunitx}
% \usepackage{pgfplots}
\usepackage{wrapfig}

\newcommand{\e}[1]{e^{i(#1)}}
\newcommand{\E}[1]{\times 10^{#1}}

\title{
    Worksheet \#8
    \\  \small
    PHYS 4C: Waves and Thermodynamics
    }
\author{Donald Aingworth IV}
\date{October 20, 2025}

\begin{document}
    \DeclareSIUnit{\celsiusdegree}{C^\circ}
    \DeclareSIUnit{\atm}{ atm}

    \maketitle

    \setcounter{section}{0}
    \section{Problem 1}
        The speed of sound in steel is 5941\,\unit{\meter/\second}.
        Steel has a density of around 7900\,\unit{\kilo\gram/\meter^3} (depends somewhat on the alloy content).
        
        a. Based on this information, what is the bulk modulus of steel?

        \begin{wrapfigure}{r}{0.25\textwidth}
            % \vspace{-30pt}
            \includegraphics[width=0.25\textwidth]{4c-wspic-1617B.jpg} 
        Sound pulse graph
            % \label{fig:wrapfig}
        \end{wrapfigure}
        b. A steel rod of cross-sectional area A is placed on the x-axis.  
        The following sound pulse is sent through the steel rod in the +$x$ direction:
        
        $f(x) = s(x,0) = s_m(1 - |x|/a)$, if $|x| < a$.
        
        $f(x) = s(x,0) = 0$, if $|x| \geq a$.

        % \includegraphics{4c-wspic-1617B.jpg}
        
        Determine the total energy of this pulse.
        
        c. Now suppose a sinusoidal sound wave with frequency $440\,\unit{\hertz}$ is sent through steel with a sound level of $100\,\unit{\deci\bel}$.  
        Calculate $s_m$ and $\Delta p_m$ for this sound wave.

        \subsection{Solution (a)}
            The bulk modulus is used as part of an equation for the velocity.
            \begin{equation}
                v   =   \sqrt{\frac{B}{\rho}}
            \end{equation}

            This can be solved for the bulk modulus.
            \begin{equation}
                B   =   \rho v^2
            \end{equation}

            We know all the values necessary, so we can solve this equation.
            \begin{equation}
                B   =   (7900\,\unit{\kilo\gram/\meter^3}) (5941\,\unit{\meter/\second})^2
                    =   \boxed{2.788\E{11}\,\unit{\kilo\gram/\meter\cdot\second^2}}
            \end{equation}
        
        \subsection{Solution (b)}
            If this is the initial point, there is no motion, so $\diff{s}{t} = 0$ at all points.
            There is an equation for the potential energy.
            \begin{align}
                dU  &=  \frac{1}{2} B \left( \diffp{s}{x} \right)^2 A\,dx
            \end{align}

            At time $t = 0$, the partial derivative of $s$ is calculatable, but it would have to be divided into two cases ($x \geq 0$ and $x < 0$).
            For the case of $x \geq 0$:
            \begin{align}
                \diffp{s}{x}    &=  \diffp{}{x}\left( s_m \left( 1 - \frac{x}{a} \right) \right)
                    =   -\frac{s_m}{a}
            \end{align}

            For the case of $x < 0$:
            \begin{align}
                \diffp{s}{x}    &=  \diffp{}{x}\left( s_m \left( 1 + \frac{x}{a} \right) \right)
                    =   \frac{s_m}{a}
            \end{align}

            Since the negative is the only thing differentiating these, it is safe to say that $\left( \diffp{s}{x} \right)^2$ is identical for both.

            These in turn can be plugged into our equation for the potential energy.
            \begin{align}
                dU  &=  \frac{1}{2} B \frac{s_m^2}{a^2} A\,dx
            \end{align}

            This in turn can be integrated between $-a$ and $a$.
            Technically, we would be integrating between $-\infty$ and $\infty$, but since $s(x,0) = 0$ everywhere outside the range of $(-a,a)$, all the other spots would result in zero to begin with.
            \begin{align}
                \int_{-\infty}^{\infty}\,dU &=  \int_{-\infty}^{\infty} \frac{1}{2} BA \frac{s_m^2}{a^2}\,dx\\
                U   &=  \int_{-a}^{a} \frac{1}{2} BA \frac{s_m^2}{a^2}\,dx
                    =   \frac{1}{2} BA \frac{s_m^2}{a^2}\int_{-a}^{a}\,dx\\
                    &=  \frac{1}{2} BA \frac{s_m^2}{a^2} \left[ x \right]_{-a}^{a}
                    =   \frac{1}{2} BA \frac{s_m^2}{a^2} (a - (-a))
                    =   \boxed{BA \frac{s_m^2}{a}}
            \end{align}

        \subsection{Solution (c)}
            We know the sound level.
            This can be used to calculate the intensity.
            \begin{gather}
                \beta   =   100\,\unit{\deci\bel}
                    =   (10\,\unit{\deci\bel})\,\log_{10}\frac{I}{I_0}\\
                10  =   \log_{10}\frac{I}{I_0}\\
                10^{10} =   \frac{I}{10^{-12}\,\unit{\watt/\meter^2}}\\
                10^{10} * 10^{-12}  =   10^{-2}\,\unit{\watt/\meter^2}
                    =   I
            \end{gather}
            
            Next, we know the bulk modulus ($B$), the speed of sound ($v$), and the frequency ($f$).
            The latter can be turned into the angular speed ($\omega$) by arithmetic.
            \begin{align}
                \omega = 2\pi f
            \end{align}

            We have an equation for the intensity that includes all of these values and the displacement amplitude, the latter of which can be solved for.
            \begin{gather}
                I   =   \frac{1}{2}\rho v \omega^2 s_m^2\\
                \begin{align}
                    s_m &=  \sqrt{\frac{2I}{\rho v \omega^2}}
                        =   \sqrt{\frac{2I}{\rho v (2\pi f)^2}}
                        =   \sqrt{\frac{I}{2\rho v (\pi f)^2}}\\
                        &=  \sqrt{\frac{10^{-2}\,\unit{\watt/\meter^2}}{2(7900\,\unit{\kilo\gram/\meter^3}) (5941\,\unit{\meter/\second}) (\pi * 440\,\unit{\hertz})^2}}\\
                        &=  \boxed{7.47\E{-9}\,\unit{\meter}}
                \end{align}
            \end{gather}

            The pressure amplitude is calculatable similarly.
            \begin{align}
                \Delta p_m  &=  (v \rho \omega) s_m\\
                    &=  (5941\,\unit{\meter/\second}) (7900\,\unit{\kilo\gram/\meter^3}) (2\pi * 440\,\unit{\hertz}) (7.47\E{-9}\,\unit{\meter})\\
                    &=  \boxed{968.85\,\unit{\pascal}}
            \end{align}
        
    \pagebreak
    \section{Problem 2}
        A half-open organ pipe is tuned to A(440) (i.e., the fundamental frequency is $440\,\unit{\hertz}$).
        Air has a density of $1.21\,\unit{\kilo\gram/\meter^3}$ and a speed of sound of $343\,\unit{\meter/\second}$ at $20\unit{\celsius}$.
        
        \begin{enumerate}[label=\alph*.]
            \item   What is the length of the pipe?
            \item   What is the maximum kinetic energy density (per unit volume) at the open end of the pipe if $s_m = 2.0\,\unit{\micro\meter}$?  At the closed end?
            \item   If the ambient temperature were raised from 20\unit{\celsius} to 40\unit{\celsius}, what would be the new fundamental frequency of the pipe (ignore changes in the length of the pipe due to the temperature change)?
        \end{enumerate}
    
        \subsection{Solution (a)}
            This is calculatable from the formula for the $n$th frequency of a sound wave in what is in this case a half-open tube.
            \begin{gather}
                f   =   \frac{nv}{4L}
            \end{gather}

            We should solve for $L$. Since it is the fundamental frequency, the value of $n$ would be $1$.
            \begin{align}
                L   &=  \frac{nv}{4f}
                    =   \frac{1 * 343\,\unit{\meter/\second}}{4 * 440\,\unit{\hertz}}
                    =   \boxed{0.195\,\unit{\meter}}
            \end{align}

        \subsection{Solution (b)}
            The lecture gives us an equation for the kinetic energy density.
            We have a known equation for the sound wave.
            \begin{gather}
                \rho_K  =   \frac{1}{2} \rho \left( \diffp{s}{t} \right)^2\\
                s(x,t)  =   s_m \cos(kx - \omega t)
            \end{gather}

            Our value for $\omega$ would be $2\pi$ times the frequency of the wave, which we do have.
            At this point, we can take the dericative of $s(x,t)$ with respect to $t$.
            \begin{equation}
                \diffp{s}{t}    =   -s_m \omega \cos(kx - \omega t)
            \end{equation}

            Put that into the equation for the kinetic enregy density.
            This will allow us to find the kinetic energy density.
            \begin{align}
                \rho_K  &=  \frac{1}{2} \rho s_m^2 \omega^2 \cos^2(kx - \omega t)
            \end{align}

            The kinetic energy density will be at its maximum when $\cos(kx - \omega t) = 1$ and consequently $\cos^2(kx - \omega t) = 1$.
            We can substitute those in here.
            Not that this would be the maximum kinetic energy density at the open end, where there would be an antinode.
            \begin{align}
                \rho_{\rm K;max;open}   &=  \frac{1}{2} \rho s_m^2 \omega^2\\
                    &=  \frac{1.21\,\unit{\kilo\gram/\meter^3}}{2} (2.0\E{-6}\,\unit{\meter})^2 (2\pi * 440\,\unit{\hertz})^2\\
                    &=  \boxed{1.85\E{-5}\,\unit{\joule/\meter^3}}
            \end{align}

            On the other hand, at the closed end, there would be a node rather than an antinode. 
            At a node, the kinetic energy is always zero, so at the closed end, the maximum kinetic energy density would be \boxed{0}.

        \subsection{Solution (c)}
            There is a formula for the speed of sound in air, equal to the square root of the adiabatic index times the universal gas constant times the temperature divided by the molar mass.
            \begin{equation}
                v = \sqrt{\frac{\gamma R T}{M}}
            \end{equation}

            We can rearrange this to isolate the speed of sound and the temperature.
            \begin{equation}
                \frac{v}{\sqrt{T}} = \sqrt{\frac{\gamma R}{M}}
            \end{equation}

            In this case, everything on the left side of the equation will be constant, so we can use a comparison of fractions.
            From there, we can solve for the speed of sound at 40\unit{\celsius} relative to 20\unit{\celsius}.
            \begin{gather}
                \frac{v_{20\unit{\celsius}}}{\sqrt{T_{20\unit{\celsius}}}} = \frac{v_{40\unit{\celsius}}}{\sqrt{T_{40\unit{\celsius}}}}\\
                v_{40\unit{\celsius}}   =   v_{20\unit{\celsius}}\sqrt{\frac{T_{40\unit{\celsius}}}{T_{20\unit{\celsius}}}}
            \end{gather}

            We can reuse our formula for the frequency in a half-open tube.
            This would still have the fundamental frequency ($n = 1$).
            The temperatures we use should be in Kelvin.
            \begin{align}
                f_{40\unit{\celsius}}   &=  \frac{nv}{4L}
                    =   \frac{v_{20\unit{\celsius}}}{4*\frac{v_{20\unit{\celsius}}}{4f_{20\unit{\celsius}}}} \cdot \sqrt{\frac{T_{40\unit{\celsius}}}{T_{20\unit{\celsius}}}}
                    =   f_{20\unit{\celsius}}\,\sqrt{\frac{T_{40\unit{\celsius}}}{T_{20\unit{\celsius}}}}\\
                    &=  440\,\unit{\hertz} \cdot \sqrt{\frac{313.15\unit{\kelvin}}{293.15\unit{\kelvin}}}
                    =   \boxed{454.8\,\unit{\hertz}}
            \end{align}
\end{document}