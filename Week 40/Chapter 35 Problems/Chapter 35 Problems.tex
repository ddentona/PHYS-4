\documentclass[12pt]{article}
\usepackage{amsmath}
\usepackage{amssymb}
\usepackage{cancel}
\usepackage{graphicx}
% \usepackage{physics}
\usepackage{siunitx}
\usepackage{wrapfig}

% \AtBeginDocument{\RenewCommandCopy\qty\SI}

\newcommand{\E}[1]{\times 10^{#1}}

\title{
    Chapter 35 End-of-Chapter Problems
    \\ \small
    Halliday \& Resnick, 10th Edition
}

\author{Donald Aingworth IV}

\date{\small Hit me where it Matters}

\begin{document}
    \DeclareSIUnit{\atm}{atm}
    \DeclareSIUnit{\cal}{\ cal}
    \DeclareSIUnit{\Cal}{\ Cal}
    \DeclareSIUnit{\calorie}{\ cal}
    \DeclareSIUnit{\Calorie}{\ Cal}
    \DeclareSIUnit{\celsiusdegree}{C^\circ}
    \DeclareSIUnit{\fahrenheit}{^\circ F}
    \DeclareSIUnit{\fahrenheitdegree}{F^\circ}
    \DeclareSIUnit{\torr}{\ torr}

    \maketitle

    % Problems start page 1073

    \pagebreak
    \section{Problem 1}
        \begin{wrapfigure}{r}{0.25\textwidth}
            \vspace{-30pt}
            \includegraphics[width=0.25\textwidth]{35-31.png} 
            % \label{fig:wrapfig}
        \end{wrapfigure}
        In Fig. 35-31, a light wave along ray $r_1$ reflects once from a mirror and a light wave along ray $r_2$ reflects twice from that same mirror and once from a tiny mirror at distance $L$ from the bigger mirror. 
        (Neglect the slight tilt of the rays.) 
        The waves have wavelength 620 nm and are initially in phase. 
        (a) What is the smallest value of L that puts the final light waves exactly out of phase? 
        (b) With the tiny mirror initially at that value of L, how far must it be moved away from the bigger mirror to again put the final waves out of phase?

        \subsection{Solution (a)}
            For the two to be completely out of phase, one of the light waves would have to travel half a wavelength more.
            We can approximate the distance traveled between big and little mirrors to be equivalent to the distance between the big and little mirror.
            \begin{gather}
                2L = \frac{\lambda}{2}\\
                L = \frac{\lambda}{4} = \frac{620\,\unit{\nano\meter}}{4} = \boxed{155\,\unit{\nano\meter}}
            \end{gather}

        \subsection{Solution (b)}
            Replace $\frac{\lambda}{2}$ with $\frac{3\lambda}{2}$.
            \begin{gather}
                2L_2 = \frac{3\lambda}{2}\\
                L_2 = \frac{3\lambda}{4} = \frac{3 \times 620\,\unit{\nano\meter}}{4} = 465\,\unit{\nano\meter}
            \end{gather}

            Now find the change in $L$.
            \begin{equation}
                \Delta L = L_2 - L = 465\,\unit{\nano\meter} - 155\,\unit{\nano\meter} = \boxed{310\,\unit{\nano\meter}}
            \end{equation}

    \pagebreak
    \section{Problem 3}
        \begin{wrapfigure}{r}{0.4\textwidth}
            \vspace{-30pt}
            \includegraphics[width=0.4\textwidth]{35-4.png} 
            % \label{fig:wrapfig}
        \end{wrapfigure}
        In Fig. 35-4, assume that two waves of light in air, of wavelength 400 nm, are initially in phase. 
        One travels through a glass layer of index of refraction $n_1 = 1.60$ and thickness $L$. 
        The other travels through an equally thick plastic layer of index of refraction $n_2 = 1.50$. 
        (a) What is the smallest value $L$ should have if the waves are to end up with a phase difference of 5.65 rad? 
        (b) If the waves arrive at some common point with the same amplitude, is their interference fully constructive, fully destructive, intermediate but closer to fully constructive, or intermediate but closer to fully destructive?

        \subsection{Solution (a)}
            What matters in this case is the difference in values of $kx$ be equal to 5.65 rad.
            In both cases, $x$ will be equal to $L$.
            We have a known formula for $k$ from $\lambda_n$, the latter of which can be found from $\lambda$ and $n$.
            \begin{gather}
                k_1x - k_2x = 5.65\,\unit{\radian}\\
                L\left( \frac{2\pi}{\lambda / n_1} - \frac{2\pi}{\lambda / n_2} \right) = 5.65\,\unit{\radian}\\
                L \left( n_1 - n_2 \right) = \lambda \times \frac{5.65\,\unit{\radian}}{2\pi\,\unit{\radian}}\\
                \begin{align}
                    L   &=  \frac{\lambda}{n_1 - n_2} \times \frac{5.65\,\unit{\radian}}{2\pi\,\unit{\radian}}
                        =   \frac{400\,\unit{\nano\meter}}{1.60 - 1.50} \times \frac{5.65\,\unit{\radian}}{2\pi\,\unit{\radian}}\\
                        &=  \frac{400\,\unit{\nano\meter} \times 5.65}{0.1 \times 2\pi}
                        =   \boxed{3.60\,\unit{\micro\meter}}
                \end{align}
            \end{gather}

        \subsection{Solution (b)}
            Divide the phase difference by $2\pi$.
            \begin{equation}
                \frac{5.65}{2\pi} = 0.9
            \end{equation}

            This means it is \underline{intermediate but closer to fully constructive}.

    \pagebreak
    \section{Problem 5}
        How much faster, in meters per second, does light travel in sapphire than in diamond? 
        See Table 33-1 (p. 992).

        \subsection{Solution}
            According to table 33-1, the index of refraction of light in sapphire is 1.77, while in diamond it is 2.42.
            For the speed of light, we use the equation for the speed of light in a medium.
            \begin{equation}
                c_n = \frac{c}{n}
            \end{equation}

            The difference in speeds of light is calculatable by taking the difference between two speeds.
            \begin{align}
                \Delta c    &=  c_{\rm sapphire} - c_{\rm diamond}
                    =   \frac{c}{n_{\rm sapphire}} - \frac{c}{n_{\rm diamond}}
                    =   c\left( \frac{1}{1.77} - \frac{1}{2.42} \right)\\
                    &=  c\left( 0.1517 \right)
                    =   \boxed{45.5\E{6}\,\unit{\meter/\second}}
            \end{align}

    \pagebreak
    \section{Problem 9}
        In Fig. 35-4, assume that the two light waves, of wavelength 620 nm in air, are initially out of phase by $\pi$ rad.
        The indexes of refraction of the media are $n_1 = 1.45$ and $n_2 = 1.65$. 
        What are the (a) smallest and (b) second smallest value of $L$ that will put the waves exactly in phase once they pass through the two media?

        \subsection{Solution (a)}
            What matters in this case is the difference in values of $kx$ be equal to $\pi$ rad.
            In both cases, $x$ will be equal to $L$.
            We have a known formula for $k$ from $\lambda_n$, the latter of which can be found from $\lambda$ and $n$.
            \begin{gather}
                k_1 x - k_2 x = L\left( \frac{2\pi}{\lambda / n_1} - \frac{2\pi}{\lambda / n_2} \right) = \pi\\
                L (n_1 - n_2) = \frac{\lambda}{2}\\
                L   =   \left| \frac{620\,\unit{\nano\meter}}{2 (1.45 - 1.65)} \right|
                    =   \frac{620\unit{\nano\meter}}{0.4}
                    =   \boxed{1.55\,\unit{\micro\meter}}
            \end{gather}

        \subsection{Solution (b)}
            Since we're going from $\pi$ to $3\pi$, just multiply it by 3.
            \begin{equation}
                3 * 1.55\,\unit{\micro\meter} = \boxed{4.650\,\unit{\micro\meter}}
            \end{equation}

    \pagebreak
    \section{Problem 15}
        A double-slit arrangement produces interference fringes for sodium light ($\lambda = 589\,\unit{\nano\meter}$) that have an angular separation of $3.50\E{-3}\,\unit{\radian}$. 
        For what wavelength would the angular separation be 10.0\% greater?

        \subsection{Solution}
            10\% greater than the current angular separation is $3.85\E{-3}\,\unit{\radian}$.
            We can develop a ratio of the initial equation from Young's experiment and the other case of the equation.
            \begin{equation}
                \frac{d\,\sin\theta_2}{d\,\sin\theta_2} = \frac{m_2 \lambda_2}{m_1 \lambda_1}
            \end{equation}

            The value of $d$ does not change, nor does the value of $m$, so we are left with only the thetas and lambdas and we can solve for the second wavelength.
            \begin{gather}
                \frac{\sin\theta_2}{\sin\theta_2} = \frac{\lambda_2}{\lambda_1}\\
                \begin{align}
                    \lambda_2   &=  \lambda_1\frac{\sin\theta_2}{\sin\theta_2}
                        =   589\,\unit{\nano\meter} \times \frac{\sin(3.85\E{-3})}{\sin(3.50\E{-3})}
                        =   \boxed{648\,\unit{\nano\meter}}
                \end{align}
            \end{gather}

    \pagebreak
    \section{Problem 17}
        \begin{wrapfigure}{r}{0.4\textwidth}
            \vspace{-30pt}
            \includegraphics[width=0.4\textwidth]{35-37.png} 
            % \label{fig:wrapfig}
        \end{wrapfigure}
        In Fig. 35-37, two radio-frequency point sources $S_1$ and $S_2$, separated by distance $d = 2.0\,\unit{\meter}$, are radiating in phase with $\lambda = 0.50\,\unit{\meter}$.
        A detector moves in a large circular path around the two sources in a plane containing them. 
        How many maxima does it detect?

        \subsection{Solution}
            For my first half of my reasoning, see my answer to Chapter 17 Problem 19 (Week 37).
            \begin{gather}
                \frac{\Delta L}{\lambda} \equiv 0.5\,\mod 1\\
                0 \leq \Delta L < d\\
                0 \leq \frac{\Delta L}{\lambda} < 2d\\
                0 \leq \frac{L}{\lambda} < 4
            \end{gather}

            There are four cases of this. 
            Multiply this by four quarter-circles to get the answer of \boxed{16}.

    \pagebreak
    \section{Problem 19}
        Suppose that Young's experiment is performed with blue-green light of wavelength 500 nm. 
        The slits are 1.20 mm apart, and the viewing screen is 5.40 m from the slits. 
        How far apart are the bright fringes near the center of the interference pattern?

        \subsection{Solution}
            Use the small angle approximation $\sin\theta \approx \theta \approx \tan\theta$.
            We can put this into the equation for the bright fringes in Young's interference experiment.
            \begin{gather}
                d \tan\theta = d \frac{\Delta y}{D} = \lambda\\
                \Delta y    =   \frac{\lambda D}{d}
                    =   \frac{500\,\unit{\nano\meter} \times 5.40\,\unit{\meter}}{1.20\,\unit{\milli\meter}}
                    =   \boxed{2.25\,\unit{\milli\meter}}
            \end{gather}

    \pagebreak
    \section{Problem 23}
        \begin{wrapfigure}{r}{0.3\textwidth}
            \vspace{-30pt}
            \includegraphics[width=0.3\textwidth]{35-38.png} 
            % \label{fig:wrapfig}
        \end{wrapfigure}
        In Fig. 35-38, sources $A$ and $B$ emit long-range radio waves of wavelength 400 m, with the phase of the emission from $A$ ahead of that from source $B$ by 90\unit\degree. 
        The distance $r_A$ from $A$ to detector $D$ is greater than the corresponding distance $r_B$ by 100 m. 
        What is the phase difference of the waves at $D$?

        \subsection{Solution}
            \boxed{0}

    \pagebreak
    \section{Problem 25}
        \begin{wrapfigure}{r}{0.3\textwidth}
            \vspace{-30pt}
            \includegraphics[width=0.3\textwidth]{35-40.png} 
            % \label{fig:wrapfig}
        \end{wrapfigure}
        In Fig. 35-40, two isotropic point sources of light ($S_1$ and $S_2$) are separated by distance 2.70 \unit{\micro\meter} along a $y$ axis and emit in phase at wavelength 900 nm and at the same amplitude. 
        A light detector is located at point $P$ at coordinate $x_P$ on the $x$ axis. 
        What is the greatest value of $x_P$ at which the detected light is minimum due to destructive interference?

        \subsection{Solution}
            Pretend it's Young's experiment.
            \begin{gather}
                d \frac{y}{D} = \frac{\lambda}{2}\\
                \begin{align}
                    D   &=  \frac{2yd}{\lambda}
                        =   \frac{2 \times 1.35\E{-6}\,\unit{\meter} \times 2.70\E{-6}\,\unit{\meter}}{900\E{-9}\,\unit{\meter}}
                        =   \boxed{8.1\E{-6}\,\unit{\meter}}
                \end{align}
            \end{gather}
            % For destructive interference to occur, the difference in distance traveled would have to be equal to half the wavelength.
            % \begin{equation}
            %     s_2 - s_1 = 450\,\unit{\nano\meter}
            % \end{equation}

            % $s_2$ above is related to $s_1$ by the pythagorean theorem.
            % \begin{gather}
            %     s_2 = \sqrt{s_1^2 + 2.70\,\unit{\micro\meter}}\\
            %     \sqrt{s_1^2 + (2.70\,\unit{\micro\meter})^2} - s_1 = 450\,\unit{\nano\meter}
            % \end{gather}

            % From here, we can solve for $s_1$.
            % \begin{gather}
            %     s_1 + 450\,\unit{\nano\meter} = \sqrt{s_1^2 + (2.70\,\unit{\micro\meter})^2}\\
            %     \cancel{s_1^2} + 900\E{-9}s_1 + 202.5\E{-15}\,\unit{\meter} = \cancel{s_1^2} + 7.29\E{-12}\\
            %     900\E{-9}s_1 = 728.8\E{-12}
            % \end{gather}

    \pagebreak
    \section{Problem 27}
        A thin flake of mica (n = 1.58) is used to cover one slit of a double-slit interference arrangement. 
        The central point on the viewing screen is now occupied by what had been the seventh bright side fringe (m = 7). 
        If $\lambda = 550\,\unit{\nano\meter}$, what is the thickness of the mica?

        \subsection{Solution}

    \pagebreak
    \section{Problem 29}

        \subsection{Solution}

    \pagebreak
    \section{Problem 31}

        \subsection{Solution}

    \pagebreak
    \section{Problem 35}

        \subsection{Solution}

    \pagebreak
    \section{Problem 39}

        \subsection{Solution}

    \pagebreak
    \section{Problem 43}

        \subsection{Solution}

    \pagebreak
    \section{Problem 45}

        \subsection{Solution}

    \pagebreak
    \section{Problem 51}

        \subsection{Solution}

    \pagebreak
    \section{Problem 63}

        \subsection{Solution}

    \pagebreak
    \section{Problem 69}

        \subsection{Solution}

    \pagebreak
    \section{Problem 71}

        \subsection{Solution}

    \pagebreak
    \section{Problem 73}

        \subsection{Solution}

    \pagebreak
    \section{Problem 75}

        \subsection{Solution}

    \pagebreak
    \section{Problem 85}

        \subsection{Solution}

    \pagebreak
    \section{Problem 91}

        \subsection{Solution}

    \pagebreak
    \section{Problem 103}

        \subsection{Solution}

    \pagebreak
    \tableofcontents
\end{document}