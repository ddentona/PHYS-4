\documentclass[12pt]{article}
\usepackage{amsmath}
\usepackage{esdiff}
\begin{document}
    \begin{center}
        \textbf{A Document Detailing the Derivation of the Charge and Current in an LC Circuit}\\
        Author: Think the Duck
    \end{center}
    \textit{For Prof Cauthen}

    This is a document meant for the sole purpose of deriving the formula for the charge and current in an LC Circuit.

    An LC Crcuit is derived as having an inductor of inductance $L$ and a capacitor of capacitance $C$ separating a charge of $Q_0$. 
    We will begin with the formula for the potential energy of the circuit.
    The potential energy of the circuit is equal to the potential energy of its components (the capacitor and the inductor).
    \begin{equation}
        U = U_C + U_L
    \end{equation}
    
    We have formulas for the potential energy in a capacitor ($U_C = \frac{Q(t)^2}{2C}$) and an inductor ($U_L = \frac{LI^2}{2}$).
    \begin{equation}
        U = \frac{Q(t)^2}{2C} + \frac{LI(t)^2}{2}
    \end{equation}

    We can take the derivative of both sides with respect to time. 
    \begin{gather}
        \diff{U}{t} = \frac{Q(t)}{C} \cdot \diff{Q}{t} + LI(t)\diff{I(t)}{t}
    \end{gather}

    Since the current $I$ is equal to the derivative of charge with respect to time, we can substitute this in.
    Furthermore, due to the conservation of energy, the derivative of potential energy with respect to time would be zero.
    \begin{gather}
        \diff{U}{t} = 0 = \frac{1}{C} \cdot Q(t) \diff{Q(t)}{t} + L \cdot \diff{Q(t)}{t} \cdot \diff[2]{Q(t)}{t}
    \end{gather}

    This is an ordinary differential equation.
    As such, we will rewrite the differential equation for an easier-to-read format.
    \begin{gather}
        0 = \frac{1}{C} \cdot Q(t) Q'(t) + L \cdot Q'(t) \cdot Q''(t)
    \end{gather}

    The entire equation can be divided by $Q'(t)$.
    For the purpose of solving a differential equation, we can also divide it all by $L$.
    \begin{gather}
        0 = \frac{1}{LC} \cdot Q(t) + Q''(t)
    \end{gather}

    We can say that at time 0, $Q(0) = Q_0$.
    Furthermore, at time 0, the current ($Q'(t)$) is equal to zero.
    This sets us up with an initial value problem.
    \begin{equation}
        0 = \frac{1}{LC} \cdot Q(t) + Q''(t) \text{ IVP } \begin{cases}
            Q(0) = Q_0\\
            Q'(0) = 0
        \end{cases}
    \end{equation}

    Using Laplace transforms, we can find a formula we can solve.
    For context, Laplace transforms allow us to turn a differential equation we can't solve yet into a separate equation that we can solve, then turn it back into the original format.
    In this context, the Laplace transforms that are relevant are three.
    \begin{gather}
        \mathcal{L} (Q(t)) = \int_{0^-}^{\infty} Q(t) e^{-st} dt = Q^{(-1)}(s)\\
        \mathcal{L} (Q''(t)) = s^2 Q^{(-1)}(s) - s Q(0^-) - Q'(0^-)\\
        \mathcal{L} (cos(\omega t)) = \frac{s}{s^2 + \omega^2}
    \end{gather}

    We can apply these to the equation we have.
    \begin{align}
        0   &=  \frac{1}{LC} \cdot Q(t) + Q''(t)\\
        \mathcal{L} (0) &=  \mathcal{L} (\frac{1}{LC} \cdot Q(t) + Q''(t))\\
        0   &=  \frac{1}{LC} \mathcal{L} (Q(t)) + \mathcal{L} (Q''(t))\\
        0   &=  s^2 Q^{(-1)}(s) - s Q(0^-) - Q'(0^-) + \frac{1}{LC}Q^{(-1)}(s)
    \end{align}

    Now, we can substitute in our known values and solve for $Q^{(-1)}(s)$.
    \begin{align}
        0   &=  s^2 Q^{(-1)}(s) - s Q(0^-) - Q'(0^-) + \frac{1}{LC}Q^{(-1)}(s)\\
            &=  (s^2 + \frac{1}{LC}) Q^{(-1)}(s) - s \cdot Q_0\\
        s \cdot Q_0 &=  (s^2 + \frac{1}{LC}) Q^{(-1)}(s)\\
        Q^{(-1)}(s) &=  Q_0 \frac{s}{s^2 + \frac{1}{LC}}
            =   Q_0 \frac{s}{s^2 + (\frac{1}{\sqrt{LC}})^2}
    \end{align}

    This can be acted upon by an inverse Laplace transform.
    \begin{gather}
        \mathcal{L}^{-1} (Q^{(-1)}(s))  =   Q(t)\\
        \mathcal{L}^{-1} (Q_0 \frac{s}{s^2 + (\frac{1}{\sqrt{LC}})^2})
            =   Q_0 \cos\left( \frac{1}{\sqrt{LC}}t \right)\\
        Q(t)    =   Q_0 \cos\left( \frac{1}{\sqrt{LC}}t \right)
    \end{gather}

    This brings us to a point where we ahve a solution to the nitial value problem.
    Note that this applies rather exclusively for the case that the current starts with no current.
    To allow it to function with a different initial value, we can add a phase change constant $\phi$ to the interior of the cosine.
    This is par for the course in terms of wave functions with sines and cosines like this one.
    \begin{equation}
        Q(t)    =   Q_0 \cos\left( \frac{1}{\sqrt{LC}}t + \phi \right)
    \end{equation}

    We can take the derivative of this with respect to time to find the current.
    \begin{equation}
        \diff{Q(t)}{t}  =   I(t)    =   -\frac{Q_0}{\sqrt{LC}} \sin\left( \frac{1}{\sqrt{LC}}t + \phi \right)
    \end{equation}

    This gives us our final equations for charge and current with respect to time in an LC circuit.
    \begin{gather}
        \boxed{ \begin{matrix}
            Q(t)    =   Q_0 \cos\left( \frac{1}{\sqrt{LC}}t + \phi \right)\\
            I(t)    =   - Q_0 \frac{1}{\sqrt{LC}} \sin\left( \frac{1}{\sqrt{LC}}t + \phi \right)
        \end{matrix}}
    \end{gather}

    This document has fulfilled its sole purpose.
\end{document}