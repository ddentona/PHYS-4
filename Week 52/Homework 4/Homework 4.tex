\documentclass[12pt]{article}
\usepackage{amsmath}
\usepackage{amssymb}
\usepackage{amsthm}
\usepackage{cancel}
\usepackage{enumitem}
\usepackage{esdiff}
\usepackage{graphicx}
\usepackage{siunitx}
\usepackage{ulem}
% \usepackage{pgfplots}
\usepackage{wrapfig}

\newcommand{\e}[1]{e^{#1}}
\newcommand{\ei}[1]{e^{i\left(#1\right)}}
\newcommand{\E}[1]{\times 10^{#1}}

\renewcommand\qedsymbol{TENA}

\title{
    Homework \#4
    \\  \small
    PHYS 4D: Modern Physics
    }
\author{Donald Aingworth IV}
\date{January 26, 2026}

\begin{document}
    \maketitle

    \section{Questions}
        \subsection{Question 4}
            Denoting the wave function of a particle by $\psi(x)$, write down an expression for the probability that the particle will be found between $a$ and $b$.

            \subsubsection{Solution}
                Use an integral and the relationship between $\psi(x)$ and probability.
                $\psi^\star(x)$ denotes the complex conjugate of $\psi(x)$.
                \begin{equation}
                    P(a < x < b) = \int_{a}^{b} \psi^\star(x) \psi(x)\,dx
                \end{equation}

        \subsection{Question 5}
            Denoting the wave function of a particle by $\psi(x)$, write down an equation for the average value of $x$.

            % \subsubsection{Solution}
            
        \subsection{Question 6}
            Suppose that a particle, which is confined to move in one-dimension between 0 and $L$, is described by the wave function, $\psi(x)= Ax(L - x)$. 
            What condition could be imposed upon the wave function $\psi(x)$ to determine the constant $A$?

            % \subsubsection{Solution}
            
        \subsection{Question 7}
            Suppose that a perfectly elastic ball were bouncing back and forth between two rigid walls with no gravity. 
            Which of the variables, $p$, $|p|$, $E$, would have a constant value?

            % \subsubsection{Solution}
            
        \subsection{Question 8}
            Sketch the form of the wave functions corresponding to the three lowest energy levels of a particle confined to an infinite potential well.

            % \subsubsection{Solution}
            
        \subsection{Question 10}
            What is the value of the kinetic energy of a particle at the classical turning points of an oscillator?

            % \subsubsection{Solution}
            
        \subsection{Question 12}
            Suppose that a harmonic oscillator made a transition from the $n = 3$ to the $n = 2$ state. What would be the energy of the emitted photon?

            % \subsubsection{Solution}
            
        \subsection{Question 13}
            Describe in qualitative terms the form of the wave functions of the harmonic oscillator between the classical turning points?

            % \subsubsection{Solution}
            
        \subsection{Question 14}
            How does the form of the wave function of the harmonic oscillator change as x increase beyond the classical turning point.

            % \subsubsection{Solution}
            
        \subsection{Question 18}
            Describe the wave functions obtained by multiplying the stationary wave $A\e{ikx}$ by the function $\e{-i\omega t}$.

            % \subsubsection{Solution}
    
    \section{Problem 3}
        An electron in a $10\,\unit{\nano\meter}$-wide infinite well makes a transition from the $n = 3$ to the $n = 2$ state emitting a photon.
        Calculate (a) the energy of the photon and (b) the wavelength of the light.

        \subsection{Solution (a)}
            The equation of the energy in a well is given in equation (2.17).
            \begin{equation}
                \label{2.17} \tag{2.17} E = \frac{n^2 h^2}{8mL^2}
            \end{equation}

            This can be used to calculate the change in energy.
            \begin{align}
                \Delta E    &=  E_2 - E_3
                    =   \frac{2^2 h^2}{8mL^2} - \frac{3^2 h^2}{8mL^2}
                    =   (2^2 - 3^2)\frac{h^2}{8mL^2}\\
                    &=  -5\frac{h^2}{8mL^2}
                    =   -\frac{5 * (6.626\E{-34})^2}{8 (9.109\E{-31}) (10\E{-9})^2}\\
                    &=  -3.01\E{-21}\,\unit{\joule}
            \end{align}

            The energy of the photon would be the negative of this.
            \begin{equation}
                E_{\rm photon} = -\Delta E = \boxed{3.01\E{-21}\,\unit{\joule}}
            \end{equation}

        \subsection{Solution (b)}
            Turn energy to wavelength.
            \begin{equation}
                \lambda =   \frac{hc}{E}
                    =   \frac{1.986\E{-25}\,\unit{\joule\cdot\meter}}{3.01\E{-21}\,\unit{\joule}}
                    =   \boxed{65.9\,\unit{\micro\meter}}
            \end{equation}
    \pagebreak
    
    \section{Problem 4}
        Show by direct substitution that the wave function (\ref{wave}) satisfies Eq. (\ref{2.32}) for the harmonic oscillator. 
        Calculate the corresponding energy.
        \begin{gather}\label{wave}
            \psi(x)= A\e{-\frac{m\omega x^2}{2\hbar}}\\
            \label{2.32}
            -\diffp[2]{\psi}{x} + \left( \frac{m\omega^2 x^2}{\hbar^2} \right) \psi = \left( \frac{2mE}{\hbar^2} \right) \psi \tag{2.32}
        \end{gather}

        % \subsection{Solution}

    \section{Problem 5}
        Determine the constant A in the preceding problem by requiring that the wave function be normalized.
        Hint: For an arbitrary value of the constant a, the integral that arises in doing this problem may be evaluated using equation (\ref{5.0}).
        \begin{equation}\label{5.0}
            \int_{0}^{\infty} \e{-ax^2}\,dx = \frac{1}{2} \sqrt{\frac{\pi}{a}}
        \end{equation}

        % \subsection{Solution}

    \section{Problem 6}
        A particle is described by the below wave function where $A$ and $a$ are constants.
        \begin{equation}
            \psi(x) = \left\{ \begin{matrix}
                Ax\,\e{-ax},    & x > 0\\
                0,              & x \leq 0
            \end{matrix} \right.
        \end{equation}

        \begin{enumerate}[label=\textbf{(\alph*)}]
            \item   Sketch the wave function.
            \item   Use the normalization condition to determine the constant $A$.
            \item   Find the most probable position of the particle.
            \item   Calculate the average value of the position of the particle.
        \end{enumerate}

        % \subsection{Solution}

    \section{Problem 7}
        (a) For a particle moving in the potential well shown in Fig. 2.7, write down the Schrödinger equations for the region where $0 \leq x \leq L$ and the region where $x \geq L$. 
        (b) Give the general form of the solution in the two regions. 
        (c) Assuming that the potential is infinite at $x = 0$, impose boundary conditions that are natural for this problem and derive an equation that can be used to find the energy levels for the bound states.

        % \subsection{Solution}

    \section{Problem 8}
        Show that the wave function of a traveling wave (\ref{2.41}) satisfies the time-dependent Schrödinger equation (\ref{2.47}).
        \begin{gather}
            \label{2.41} \tag{2.41} \psi(x,t) = Ae^{ikx} \cdot e^{-i\omega t}\\
            \label{2.47} \tag{2.47} \left[ \frac{-\hbar^2}{2m}\,\diffp[2]{}{x} + V(x,t) \right] \psi(x,t) = i \hbar \diffp[2]{\psi(x,t)}{t}
        \end{gather}

        % \subsection{Solution}

    \section{Problem 9}
        Show how the wave function of the even states of a particle in an infinite well extending from $x = -L/2$ to $x = L/2$ evolve in time.

        % \subsection{Solution}

\end{document}