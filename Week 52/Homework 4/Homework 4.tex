\documentclass[12pt]{article}
\usepackage{amsmath}
\usepackage{amssymb}
\usepackage{amsthm}
\usepackage{cancel}
\usepackage{enumitem}
\usepackage{esdiff}
\usepackage{graphicx}
\usepackage{siunitx}
\usepackage{ulem}
% \usepackage{pgfplots}
\usepackage{wrapfig}
\usepackage{xcolor}

\newcommand{\e}[1]{e^{#1}}
\newcommand{\ei}[1]{e^{i\left(#1\right)}}
\newcommand{\E}[1]{\times 10^{#1}}

\renewcommand\qedsymbol{TENA}

\title{
    Homework \#4
    \\  \small
    PHYS 4D: Modern Physics
    }
\author{Donald Aingworth IV}
\date{January 26, 2026}

\begin{document}
    \maketitle

    \section{Questions}
        \subsection{Question 4}
            Denoting the wave function of a particle by $\psi(x)$, write down an expression for the probability that the particle will be found between $a$ and $b$.

            \subsubsection{Solution}
                Use an integral and the relationship between $\psi(x)$ and probability.
                $\psi^\star(x)$ denotes the complex conjugate of $\psi(x)$.
                \begin{gather}
                    \label{pdf}
                    1 = \int_{-\infty}^{\infty} \psi^\star(x) \psi(x)\,dx\\
                    P(a < x < b) = \int_{a}^{b} \psi^\star(x) \psi(x)\,dx
                \end{gather}

        \subsection{Question 5}
            Denoting the wave function of a particle by $\psi(x)$, write down an equation for the average value of $x$.

            \subsubsection{Solution}
                This is done by integral over all values of $x$.
                Every value of $x$ should be multiplied by its probability, which is given by the wave function of the particle.
                This is where we get (\ref{2.21}).
                \begin{equation}
                    \label{2.21} \tag{2.21}
                    \boxed{\left\langle x \right\rangle = \int_{-\infty}^{\infty} x \left| \psi(x) \right|^2\,dx}
                \end{equation}
            % \pagebreak
            
        \subsection{Question 6}
            Suppose that a particle, which is confined to move in one-dimension between 0 and $L$, is described by the wave function, $\psi(x)= Ax(L - x)$. 
            What condition could be imposed upon the wave function $\psi(x)$ to determine the constant $A$?

            \subsubsection{Solution}
                I'm thinking something to do with integrating.
                Knowing the probability density curve, we can use something similar to (\ref{2.21}).
                \begin{equation}
                    \int_{-\infty}^{\infty} \psi^\star(x) \psi(x)\,dx = \int_{-\infty}^{\infty} A^2 (Lx - x^2)^2\,dx = 1
                \end{equation}

                This can be integrated after we pull out the $A^2$.
                We can also change the bounds to $0$ and $L$ because the particle is bound to that area.
                \begin{align}
                    \label{1.3.2}
                    1   &=  A^2 \int_{0}^{L} (Lx - x^2)^2\,dx
                        =   A^2 \int_{0}^{L} L^2 x^2 - 2 Lx^3 + x^4\,dx\\\tag{\ref{1.3.2}b}
                        &=  A^2 \left[ \frac{L^2 x^3}{3} - \frac{2Lx^4}{4} + \frac{x^5}{5} \right]_0^L
                        =   A^2 \left( \frac{L^5}{3} - \frac{L^5}{2} + \frac{L^5}{5} \right)\\\tag{\ref{1.3.2}c}
                        &=  A^2 \left( \frac{10 L^5 - 15 L^5 + 6L^5}{30} \right)
                        =   A^2 \left( \frac{L^5}{30} \right)
                \end{align}

                This can be solved for $A$.
                \begin{gather}
                    A^2 = \frac{30}{L^5}\\
                    \boxed{A = \sqrt{\frac{30}{L^5}}}
                \end{gather}

                That being said, we could also just measure $\psi\left( \frac{L}{2} \right)$ and divide it by $\frac{L^2}{4}$.
                I guess it just depends on what equipment we have.
            \pagebreak
            
        \subsection{Question 7}
            Suppose that a perfectly elastic ball were bouncing back and forth between two rigid walls with no gravity. 
            Which of the variables, $p$, $|p|$, $E$, would have a constant value?

            \subsubsection{Solution}
                In a perfectly elestic collision, no magnitude of momentum is lost, so such will be the case involving the magnitude of the momentum.
                However, since momentum is a vector and the ball has limitations of where it can go, the momentum itself will not have a constant value.
                The energy, however, will have a constant value, especially since this is a closed system.
                \begin{center}
                    \begin{tabular}{| l | c | c | c |}
                        \hline
                        Variable    & $p$   & $|p|$ & $E$\\
                        \hline
                        Const?      & No    & Yes   & Yes\\
                        \hline
                    \end{tabular}
                \end{center}
            
        \subsection{Question 8}
            Sketch the form of the wave functions corresponding to the three lowest energy levels of a particle confined to an infinite potential well.

            \subsubsection{Solution}
                I used Desmos to make the below image. 
                The lowest energy level is in \textcolor{teal}{green}, second lowest in \textcolor{orange}{orange}, and third lowest in \textcolor{violet}{purple}.\\
                \includegraphics[width=\textwidth]{Images/Q8.png}
            \pagebreak
            
        \subsection{Question 10}
            What is the value of the kinetic energy of a particle at the classical turning points of an oscillator?

            \subsubsection{Solution}
                Assuming that a turning point is a point where the tangent line is horizontal (i.e. a local maximum/minimum), the oscillator would be unmoving and the kinetic energy woud be \boxed{0}.
            
        \subsection{Question 12}
            Suppose that a harmonic oscillator made a transition from the $n = 3$ to the $n = 2$ state. What would be the energy of the emitted photon?

            \subsubsection{Solution}
                I will be answering this in terms of the angular frequency.
                \begin{align}
                    E_{\rm photon}  &=  -\Delta E
                        =   E_3 - E_2
                        =   \hbar\omega * 3.5 - \hbar\omega * 2.5\\
                        &=  \hbar\omega (3.5 - 2.5)
                        =   \boxed{\hbar\omega}
                \end{align}
            \pagebreak
            
        \subsection{Question 13}
            Describe in qualitative terms the form of the wave functions of the harmonic oscillator between the classical turning points?

            \subsubsection{Solution}
                Between the turning points, it follows a sinusoidal wave.
            
        \subsection{Question 14}
            How does the form of the wave function of the harmonic oscillator change as $x$ increases beyond the classical turning point.

            \subsubsection{Solution}
                The potential energy would be greater than the total energy, so it would taper off quickly.
            
        \subsection{Question 18}
            Describe the wave functions obtained by multiplying the stationary wave $A\e{ikx}$ by the function $\e{-i\omega t}$.

            \subsubsection{Solution}
                This is a traveling wave that moves over time.
        \pagebreak

    \section{Problem 3}
        An electron in a $10\,\unit{\nano\meter}$-wide infinite well makes a transition from the $n = 3$ to the $n = 2$ state emitting a photon.
        Calculate (a) the energy of the photon and (b) the wavelength of the light.

        \subsection{Solution (a)}
            The equation of the energy in a well is given in equation (2.17).
            \begin{equation}
                \label{2.17} \tag{2.17} E = \frac{n^2 h^2}{8mL^2}
            \end{equation}

            This can be used to calculate the change in energy.
            \begin{align}
                \Delta E    &=  E_2 - E_3
                    =   \frac{2^2 h^2}{8mL^2} - \frac{3^2 h^2}{8mL^2}
                    =   (2^2 - 3^2)\frac{h^2}{8mL^2}\\
                    &=  -5\frac{h^2}{8mL^2}
                    =   -\frac{5 * (6.626\E{-34})^2}{8 (9.109\E{-31}) (10\E{-9})^2}\\
                    &=  -3.01\E{-21}\,\unit{\joule}
            \end{align}

            The energy of the photon would be the negative of this.
            \begin{equation}
                E_{\rm photon} = -\Delta E = \boxed{3.01\E{-21}\,\unit{\joule}}
            \end{equation}

        \subsection{Solution (b)}
            Turn energy to wavelength.
            \begin{equation}
                \lambda =   \frac{hc}{E}
                    =   \frac{1.986\E{-25}\,\unit{\joule\cdot\meter}}{3.01\E{-21}\,\unit{\joule}}
                    =   \boxed{65.9\,\unit{\micro\meter}}
            \end{equation}
    \pagebreak
    
    \section{Problem 4}
        Show by direct substitution that the wave function (\ref{wave}) satisfies Eq. (\ref{2.32}) for the harmonic oscillator. 
        Calculate the corresponding energy.
        \begin{gather}\label{wave}
            \psi(x)= A\e{-\frac{m\omega x^2}{2\hbar}}\\
            \label{2.32}
            -\diffp[2]{\psi}{x} + \left( \frac{m\omega^2 x^2}{\hbar^2} \right) \psi = \left( \frac{2mE}{\hbar^2} \right) \psi \tag{2.32}
        \end{gather}

        \subsection{Solution}
            Before active substitution, take the second derivative of the wave equation with respect to $x$.
            \begin{gather}
                \diffp{\psi}{x} =   -A\frac{m\omega x}{\hbar} \e{-\frac{m\omega x^2}{2\hbar}}\\
                \label{4.2}
                \diffp[2]{\psi}{x}  =   -A\frac{m\omega}{\hbar} \e{-\frac{m\omega x^2}{2\hbar}} + A\frac{m^2\omega^2 x^2}{\hbar^2} \e{-\frac{m\omega x^2}{2\hbar}}
                    =   \frac{m^2\omega^2 x^2}{\hbar^2} \psi - \frac{m\omega}{\hbar} \psi
            \end{gather}

            This actually does not work with Equation (\ref{2.32}), because of its term $\frac{m\omega^2 x^2}{\hbar^2} \psi$, which does not appear in (\ref{4.2}).
            It doesn't even work with Equation (\ref{2.31}), which is supposed to form (\ref{2.32}) when divided by $\frac{\hbar^2}{2m}$.
            Equation (\ref{2.31}) divided by $\frac{\hbar^2}{2m}$ actually is equal to what I'm calling (\ref{2.32v}).
            \begin{gather}
                \label{2.31}\tag{2.31}
                -\frac{\hbar^2}{2m}\diffp[2]{\psi}{x} + \frac{1}{2}m\omega x^2 \psi = E \psi\\
                \label{2.32v}\tag{$2.32\nu$}
                -\diffp[2]{\psi}{x} + \left( \frac{m^2\omega^2 x^2}{\hbar^2} \right) \psi = \left( \frac{2mE}{\hbar^2} \right) \psi
            \end{gather}

            Since (\ref{2.32v}) works with (\ref{4.2}), I will assume a typo in the textbook and move forward using (\ref{2.32v}) instead.
            First, I will substitute in for $\diffp[2]{\psi}{x}$.
            First thing's first, cancel out all $\phi$ and cancel out $\pm \frac{m^2\omega^2 x^2}{\hbar^2}$.
            \begin{gather}
                \frac{m\omega}{\hbar} \cancel{\psi} - \bcancel{\frac{m^2\omega^2 x^2}{\hbar^2}} \cancel{\psi} + \bcancel{\left( \frac{m^2\omega^2 x^2}{\hbar^2} \right)} \cancel{\psi} = \left( \frac{2mE}{\hbar^2} \right) \cancel{\psi}\\
                \frac{m\omega}{\hbar} = \frac{2mE}{\hbar^2} \label{4.6}
            \end{gather}

            This brings us to Equation (\ref{4.6}).
            We can cancel out some more values and solve for $E$.
            \begin{gather}
                \frac{\cancel{m}\omega}{\cancel{\hbar}} = \frac{2\cancel{m}E}{\hbar^{\cancel{2}}}\\
                \boxed{E = \frac{\hbar\omega}{2}}
            \end{gather}

    \section{Problem 5}
        Determine the constant A in the preceding problem by requiring that the wave function be normalized.
        Hint: For an arbitrary value of the constant a, the integral that arises in doing this problem may be evaluated using equation (\ref{5.0}).
        \begin{equation}\label{5.0}
            \int_{0}^{\infty} \e{-ax^2}\,dx = \frac{1}{2} \sqrt{\frac{\pi}{a}}
        \end{equation}

        \subsection{Solution}
            A Gaussian, I see.
            Recall the value of $\psi$.
            \begin{equation}
                \tag{\ref{wave}}
                \psi(x)= A\e{-\frac{m\omega x^2}{2\hbar}}
            \end{equation}

            The thing to remember about $\psi$ is its relationship to the probability density function.
            \begin{equation}
                \tag{\ref{pdf}}
                1 = \int_{-\infty}^{\infty} \psi^\star(x) \psi(x)\,dx
            \end{equation}
            
            Our value of $\psi$ is its own complex conjugate, so we can plug in values of $\psi$.
            \begin{gather}
                1   =   \int_{-\infty}^{\infty} \left( A\e{-\frac{m\omega x^2}{2\hbar}} \right)^2\,dx
                    =   A^2 \int_{-\infty}^{\infty} \e{-\frac{m\omega x^2}{\hbar}}\,dx
            \end{gather}

            With $a = \frac{m\omega}{\hbar}$, we can use ($\ref{5.0}$).
            \begin{gather}
                1   =   A^2 * \frac{1}{2} \sqrt{\frac{\pi}{\frac{m\omega}{\hbar}}}
                    =   A^2 \sqrt{\frac{\pi\hbar}{4m\omega}}\\
                A^2 =   \sqrt{\frac{4m\omega}{\pi\hbar}} \to
                \boxed{A = \sqrt[4]{\frac{4m\omega}{\pi\hbar}}}
            \end{gather}

    \section{Problem 6}
        A particle is described by the below wave function where $A$ and $a$ are constants.
        \begin{equation}
            \psi(x) = \left\{ \begin{matrix}
                Ax\,\e{-ax},    & x > 0\\
                0,              & x \leq 0
            \end{matrix} \right.
        \end{equation}

        \begin{enumerate}[label=\textbf{(\alph*)}]
            \item   Sketch the wave function.
            \item   Use the normalization condition to determine the constant $A$.
            \item   Find the most probable position of the particle.
            \item   Calculate the average value of the position of the particle.
        \end{enumerate}

        \subsection{Solution (a)}
            Here's a Desmos picture I made of it.\\
            \includegraphics[width=\textwidth]{Images/P6a.png}

        \subsection{Solution (b)}
            I'm going to normalize it over the span of $[0,\infty)$.
            Since it's equal to zero at all ponts, we don't need to consider the span of $(-\infty,0)$.
            Here, $\psi$ is its own complex conjugate. 
            \begin{align}
                \tag{\ref{pdf}}
                1   &=  \int_{-\infty}^{\infty} \psi^\star(x) \psi(x)\,dx\\
                1   &=  \int_{0}^{\infty} \psi(x)^2\,dx
                    =   \int_{0}^{\infty} A^2 x^2\,\e{-2ax}\,dx
                    =   A^2 \int_{0}^{\infty} x^2\,\e{-2ax}\,dx\\
                    &   \left[ \begin{matrix} u = x^2 & du = 2x \\ v = -\frac{1}{2a}\e{-2ax} & dv = \e{-2ax} \end{matrix} \right]\\
                \frac{1}{A^2}   &=  -\frac{x^2\e{-2ax}}{2a} - \int_{0}^{\infty} -\frac{2x}{2a}\e{-2ax}\,dx
                    =   -\frac{x^2\e{-2ax}}{2a} + \int_{0}^{\infty} \frac{x}{a}\e{-2ax}\,dx\\
                    &   \left[ \begin{matrix} u = x & du = 1 \\ v = -\frac{1}{2a^2}\e{-2ax} & dv = \frac{1}{a}\e{-2ax} \end{matrix} \right]\\
                    &=  -\frac{x^2\e{-2ax}}{2a} - \frac{x}{2a^2}\e{-2ax} + \int_{0}^{\infty} \frac{1}{2a^2}\e{-2ax}\,dx\\
                    &=  \left[ -\frac{x^2\e{-2ax}}{2a} - \frac{x\e{-2ax}}{2a^2} - \frac{\e{-2ax}}{4a^3} \right]_0^\infty
            \end{align}

            This is getting a bit big, so I'm just going to outsource this to Halliday and Resnick's Physics textbook, which has a citable integral for this in its Appendix E (10th Edition).
            \begin{equation}\label{gamma} \tag{E.15}
                \int_{0}^{\infty} x^n \e{-ax}\,dx = \frac{n!}{a^{n + 1}}
            \end{equation}

            We can use this for $n = 2$ and $a = 2a$.
            \begin{gather}
                \frac{1}{A^2}   =  \int_{0}^{\infty} x^2\,\e{-2ax}\,dx
                    =   \frac{2!}{8a^3}\\
                \boxed{A = \sqrt{4a^3}}
            \end{gather}

        \subsection{Solution (c)}
            The most probable value is the maximum point of the graph, which we can fnd ourselves by first taking the derivative of $\psi$.
            \begin{equation}
                \diff{\psi}{x} = A \left( \e{-ax} - ax\e{-ax} \right)
                    =   A \e{-ax} \left( 1 - ax \right)
            \end{equation}

            The only point where this would be equal to zero is when \boxed{x = \frac{1}{a}}.

        \subsection{Solution (d)}
            Recall Equation (\ref{2.21}) from Question 5.
            \begin{equation}
                \tag{\ref{2.21}}
                \left\langle x \right\rangle = \int_{-\infty}^{\infty} x \left| \psi(x) \right|^2\,dx
            \end{equation}

            This is adaptable to our value of $\psi$.
            We use equation (\ref{gamma}) here again, like how we used it in part (b).
            We ignore the range of $(-\infty,0)$ for the same reason as in part (b).
            \begin{align}
                \left\langle x \right\rangle    &= \int_{0}^{\infty} x^3 A^2 \e{-2ax}\,dx
                    =   A^2 \frac{3!}{16a^4}
                    =   \boxed{\frac{3A^2}{8a^4}}
            \end{align}

    \section{Problem 7}
        (a) For a particle moving in the potential well shown in Fig. 2.7, write down the Schrödinger equations for the region where $0 \leq x \leq L$ and the region where $x \geq L$. 
        (b) Give the general form of the solution in the two regions. 
        (c) Assuming that the potential is infinite at $x = 0$, impose boundary conditions that are natural for this problem and derive an equation that can be used to find the energy levels for the bound states.

        % \subsection{Solution}

    \section{Problem 8}
        Show that the wave function of a traveling wave (\ref{2.41}) satisfies the time-dependent Schrödinger equation (\ref{2.47}).
        \begin{gather}
            \label{2.41} \tag{2.41} \psi(x,t) = Ae^{ikx} \cdot e^{-i\omega t}\\
            \label{2.47} \tag{2.47} \left[ \frac{-\hbar^2}{2m}\,\diffp[2]{}{x} + V(x,t) \right] \psi(x,t) = i \hbar \diffp{\psi(x,t)}{t}
        \end{gather}

        % \subsection{Solution}

    \section{Problem 9}
        Show how the wave function of the even states of a particle in an infinite well extending from $x = -L/2$ to $x = L/2$ evolve in time.

        \subsection{Solution}
            There exist three sectons we can use for the potential energy of the particle.
            \begin{equation}
                V(x)    =   \left\{ \begin{matrix}
                    \infty  & x > \frac{L}{2}\\
                    0       & -\frac{L}{2} \leq x \leq \frac{L}{2}\\
                    \infty  & x < -\frac{L}{2}
                \end{matrix} \right.
            \end{equation}

            The central prong of this is what interests us.
            It gives us a version of the Schrödinger time-dependant Equation where $V = 0$.
            \begin{equation}
                \label{9.2}
                -\frac{\hbar^2}{2m} \diffp[2]{\psi}{x} = i\hbar\diffp{\psi}{t}
            \end{equation}

            % We can solve this differential equation pretty easily.
            % It's exponential, so we can describe it in those terms, in the form of a traveling wave.
            % \begin{gather}
            %     \psi = A\ei{kx - \omega t}\\
            %     \diffp{\psi}{t} = -i\omega A\ei{kx - \omega t}\\
            %     \diffp[2]{\psi}{x} = -k^2 A\ei{kx - \omega t}
            % \end{gather}

            % Knowing these, we compare these to our values in Schrödinger's Time-Dependant Equation (\ref{9.2}) to get values of $k$ and $\omega$.
            % \begin{gather}
            %     \tag{\ref{9.2}}
            %     -\frac{\hbar^2}{2m} \diffp[2]{\psi}{x} = i\hbar\diffp{\psi}{t}\\
            %     \left( -\frac{\hbar^2}{2m} \right) \left( -k^2 A\ei{kx - \omega t} \right) = (i\hbar)\left( -i\omega A\ei{kx - \omega t} \right)\\ \label{9.7}
            %     \left( \frac{\hbar^{\cancel{2}}}{2m} \right) \left( k^2 \cancel{A\ei{kx - \omega t}} \right) = \cancel{\hbar}\left( \omega \cancel{A\ei{kx - \omega t}} \right) \\
            %     \frac{\hbar k^2}{2m} = \omega
            % \end{gather}

            % There are plenty of ways to solve this, but to keep ths simple, let's bring this back to slightly after (\ref{9.7}).
            % Only cancel out the exponetial and the constant $A$.
            % \begin{gather}
            %     \tag{\ref{9.7}}
            %     \left( \frac{\hbar^{2}}{2m} \right) \left( k^2 \cancel{A\ei{kx - \omega t}} \right) = \hbar\left( \omega \cancel{A\ei{kx - \omega t}} \right)\\
            %     k^2 \frac{\hbar^2}{2m} = \hbar\omega
            % \end{gather}

            % We can generate potential values of $\omega$ and $k$.
            % \begin{gather}
            %     k = \frac{\sqrt{2m}}{\hbar}\\
            %     \omega = \frac{1}{\hbar}\\
            %     \psi = A\ei{\frac{\sqrt{2m}}{\hbar}x - \frac{1}{\hbar} t}
            % \end{gather}
\end{document}