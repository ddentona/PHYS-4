\documentclass[12pt]{article}
\usepackage{amsmath}
\usepackage{array}
% \usepackage{gensymb}
\usepackage{geometry}
\usepackage{graphicx}
\usepackage{pgfplots}
\usepackage{siunitx}
\usepackage{wrapfig}

\title{Homework \#16}
\author{Donald Aingworth IV}
\date{December 11, 2024}

\pgfplotsset{width=8cm,compat=1.9}
\usepgfplotslibrary{external}
% \tikzexternalize

\begin{document}

\DeclareSIUnit{\mile}{mi}
\DeclareSIUnit{\gal}{gal}
\DeclareSIUnit{\foot}{ft}
\DeclareSIUnit{\hour}{h}
\DeclareSIUnit{\rad}{rad}
\DeclareSIUnit{\unit}{u}
\DeclareSIUnit{\dyne}{dyn}

\maketitle

\pagebreak
\section*{Question 2}
\includegraphics[width=\textwidth]{picture_1.png}
Figure 21-12 shows three pairs of identical spheres that are to be touched together and then separated. The initial charges on them are indicated. Rank the pairs according to (a) the magnitude of the charge transferred during touching and (b) the charge left on the positively charged sphere, greatest first.

\subsection*{Solution}
a) (3) $>$ (1) $>$ (2)\\
In this problem, we just need to rank the differences between the charges on the balls. In instance 1, the difference is $6 - (-4) = 10$. In instance 2, the difference is $2 - 0 = 2$. In instance 3, the difference is $14 - (-12) = 26$. We can then rank the three, and the result is \boxed{(3) > (1) > (2)}.\\
b) (1) = (2) = (3)
In this In this problem, we just need to rank the average charge of each of the two balls. In instance 1, the average is $\frac{6 + (-4)}{2} = 1$. In instance 2, the difference is $\frac{2 + 0}{2} = 1$. In instance 3, the difference is $\frac{14 + (-12)}{2} = 1$. We can then rank the three, and the result is \boxed{(1) = (2) = (3)}.


\pagebreak
\section*{Question 8}
% \begin{wrapfigure}{r}{0.15\textwidth}
%     \vspace{-30pt}
    \includegraphics[width=\textwidth]{picture_2.png} 
%     % \label{fig:wrapfig}
% \end{wrapfigure}

\subsection*{Solution}
\subsubsection*{(a) = (d) $>$ (b) = (c)}
For (a) and (d), the magnitude is always positive and the charge of a positron is the same as the charge of an electron, and the other particles are working together to move +Q more than separately. This means that they have the same magnitude.\\
For (b) and (c), the magnitude is always positive, but the two charged particles are exerting force on the +Q particle in relatively opposite directions. This means that the net force on the +Q particle is the same, but it is less than the net force on it in (a) or (d).\\
Hence, \boxed{(a) = (d) > (b) = (c)}


\pagebreak
\section*{Question 10}
\includegraphics[width=\textwidth]{picture_3.png} 

\subsection*{Solution}
In this case, the first thing we can do is rule out the charges with charges of equal magnitude and equal distance to the central charge. Almost all charges are canceled out, as follows (in clockwise order): \(-7q; +4q; -3q; -5q; +2q\). The only remaining charge is $+3q$. As such, we can find the net force using only the force from that remaining charge. Since the charges of $+3q$ and $-2q$ are opposite and opposites attract, the direction is towards the $+3q$ charge, so to the \boxed{left}.
\[F = \frac{k\left|q_1\right|\left|q_2\right|}{r^2} 
    = \frac{k\left|3q\right|\left|-2q\right|}{d^2} 
    = \boxed{\frac{6kq^2}{d^2}} \]

\pagebreak
\section*{Problem 10}
\begin{wrapfigure}{r}{0.25\textwidth}
    \vspace{-50pt}
    \includegraphics[width=0.25\textwidth,height=0.25\textwidth]{picture_4.png} 
    % \label{fig:wrapfig}
\end{wrapfigure}
In Fig. 21-25, four particles form a square. The charges are $q_1 = q_4 = Q$ and $q_2 = q_3 = q$. (a) What is Q/q if the net electrostatic force on particles 1 and 4 is zero? (b) Is there any value of $q$ that makes the net electrostatic force on each of the four particles zero? Explain.

\subsection*{Solution}
The distance between particles 1 \& 4 (and 2 \& 3) is found by the pythagorean theorem.
\[r_{14} = r_{23} = \sqrt{a^2 + a^2} = a\sqrt{2}\]

\subsubsection*{(a) Find Q/q}
From this, we know that net electrostatic force on 1 from 4 in the direction of the line between 1 and 2 has to be equal to the force on 1 from 4. Roughly the same applies in the direction of the line between 1 and 3. As such, we can write a formula for the net force (which is 0).
\begin{align*}
    F_{net} = 0 = \frac{k\left|q_1\right|\left|q_2\right|}{r^2} &- \frac{k\left|q_1\right|\left|q_4\right|}{r^2}*\cos(\theta)\\
    \frac{k\left|q_1\right|\left|q_2\right|}{r^2} &= \frac{k\left|q_1\right|\left|q_4\right|}{r^2}*\cos(\theta)\\
    \frac{k\left|Q\right|\left|q\right|}{a^2} &= \frac{k\left|Q\right|\left|Q\right|}{(a\sqrt{2})^2}*\cos(\theta)\\
    \frac{\left|q\right|}{a^2} &= \frac{\left|Q\right|}{2a^2}*\frac{\sqrt{2}}{2}\\
    \frac{\left|Q\right|}{\left|q\right|} = \left|\frac{Q}{q}\right| &= \frac{4}{\sqrt{2}} = 2\sqrt{2}
\end{align*}
Since the forces must be in opposite directions, the charges must be opposite, so the ration has to be negative. This means that \boxed{Q/q = -2\sqrt{2}}.

\pagebreak
\subsubsection*{(b) For what $q$ is $F_{net} = 0$?}
We can put aside the situation where \boxed{q = 0}, since that feels like cheating. We have two formulas that must be true, one for the particles of charge $Q$ (1) and one for the particles of charge $q$ (2). 
\begin{equation}
    0 = \frac{k*|q|*|Q|}{a^2} - \frac{k*|q|*|q|}{(a\sqrt{2})^2}
\end{equation}
\begin{equation}
    0 = \frac{k*|q|*|Q|}{a^2} - \frac{k*|Q|*|Q|}{(a\sqrt{2})^2}
\end{equation}

Adding these two together, we get:
\begin{equation}
    0 = \frac{k*|Q|*|Q|}{(a\sqrt{2})^2} - \frac{k*|q|*|q|}{(a\sqrt{2})^2}
\end{equation}

We can then substitute in a prior value of Q. $Q/q = -2\sqrt{2} \rightarrow Q = -2\sqrt{2}q$. 
\begin{equation}
    0 = \frac{k*Q^2 - k*q^2}{2a^2} = \frac{k(8q^2 - q^2)}{2a^2} = \frac{7kq^2}{2a^2}
\end{equation}

Since $\frac{7k}{2a^2}$ is made up only of constants, this means that we end up with $q^2 = 0$, so there is \boxed{\text{no other solution besides } q = 0}.


\pagebreak
\section*{Problem 11}
\begin{wrapfigure}{r}{0.25\textwidth}
    \vspace{-30pt}
    \includegraphics[width=0.25\textwidth]{picture_4.png} 
    % \label{fig:wrapfig}
\end{wrapfigure}
In Fig. 21-25, the particles have charges $q_1 = -q_2 = 100 \unit{\nano\coulomb}$ and $q_3 = -q_4 = 200 \unit{\nano\coulomb}$, and distance $a = 5.0\unit{\centi\meter}$. What are the (a) x and (b) y components of the net electrostatic force on particle 3?

\subsection*{Solution}
The charges of the four points are as follows: 
\begin{align*}
    &q_1 = 100\unit{\nano\coulomb}
    &q_2 = -100\unit{\nano\coulomb} &
    &q_3 = 200\unit{\nano\coulomb} &
    &q_4 = -200\unit{\nano\coulomb}
\end{align*}
This means that $q_3$ is repelled by $q_1$, but attracted by $q_2$ and $q_4$. 
\subsubsection*{(a) x-force on particle 3}
From the previous problem, we have determined the formula for the x-value of the force.
\begin{align*}
    F_{net} &=  \frac{k*|q_3||q_4|}{a^2}*\hat{r}_{34;x} + \frac{k*|q_3||q_2|}{2a^2}*\hat{r}_{32;x}\\
            &=  \frac{k*|q_3||q_4|}{a^2} + \frac{k*|q_3||q_2|}{2a^2}*\frac{\sqrt{2}}{2}\\
            &=  \frac{k*|200 \times 10^{-9}||-200 \times 10^{-9}|}{0.05^2} + \frac{k*|200 \times 10^{-9}||-100 \times 10^{-9}|}{2*0.05^2}*\frac{\sqrt{2}}{2}\\
            &=  \frac{k*40000 \times 10^{-18}}{0.0025} + \frac{k*20000 \times 10^{-18}*\sqrt{2}}{0.01}\\
            &=  \frac{k*160000 \times 10^{-18} + k*20000 \times 10^{-18}*\sqrt{2}}{10^{-2}}\\
            &=  k*160000 \times 10^{-16} + k*20000 \times 10^{-16}*\sqrt{2}\ \unit{\coulomb^2/\meter^2}\\
            &=  k(160000 + 20000*\sqrt{2}) \times 10^{-16}\ \unit{\coulomb^2/\meter^2}
            =   8.99*1882.842 \times 10^{-5}\ \unit{\newton}\\
            &=  \boxed{1.6927 \times 10^{-1} \unit{\newton}}
\end{align*}

\subsubsection*{(b) y-force on particle 3}
I'll be using the vector notation for this, because it makes more sense to me. 
\begin{align*}
    \vec{F}_{net,3}   &=  \vec{F}_{13} + \vec{F}_{23} + \vec{F}_{43}
        =   F_{13}\left[\begin{smallmatrix}0\\-1\end{smallmatrix}\right] + F_{23}\left[\begin{smallmatrix}-\sqrt{2}/2\\-\sqrt{2}/2\end{smallmatrix}\right] + F_{43}\left[\begin{smallmatrix}-1\\0\end{smallmatrix}\right]\\
        &=  k\left( \frac{q_1 q_3}{0.05^2}\begin{bmatrix}0\\-1\end{bmatrix} + \frac{q_2 q_3}{(0.05\sqrt{2})^2}\begin{bmatrix}-\sqrt{2}/2\\-\sqrt{2}/2\end{bmatrix} + \frac{q_4 q_3}{0.05^2}\begin{bmatrix}-1\\0\end{bmatrix} \right)\\
        &=  \frac{k q_3}{0.05^2}\left( q_1\begin{bmatrix}0\\-1\end{bmatrix} + \frac{q_2}{2}\begin{bmatrix}-\sqrt{2}/2\\-\sqrt{2}/2\end{bmatrix} + q_4\begin{bmatrix}-1\\0\end{bmatrix} \right)\\
        &=  \frac{k q_3}{0.05^2}\left( \begin{bmatrix}0\\-q_1\end{bmatrix} + \begin{bmatrix}-q_2 \sqrt{2}/4\\-q_2 \sqrt{2}/4\end{bmatrix} + \begin{bmatrix}-q_4\\0\end{bmatrix} \right)\\
        &=  \frac{8.99 \times 10^{9} * 200 \times 10^{-9}}{25 \times 10^{-4}}\begin{bmatrix}- q_4 - q_2 \sqrt{2}/4\\- q_1 - q_2 \sqrt{2}/4\end{bmatrix}\\
        &=  \frac{8.99 * 2 \times 10^2}{2.5 \times 10^{-3}}\begin{bmatrix} 200 - 25 \sqrt{2} \\ - 100 - 25 \sqrt{2} \end{bmatrix} \times 10^{-9}\\
        &=  \frac{1.789 \times 10^5 \times 10^{-9}}{2.5}\begin{bmatrix} 200 - 25 \sqrt{2} \\ - 100 - 25 \sqrt{2} \end{bmatrix}\\
        &=  \begin{bmatrix} 1.6927 \times 10^{-1} \\ -4.649 \times 10^{-2} \end{bmatrix} \unit{\newton}
\end{align*}
This means that the answer is \boxed{-4.649 \times 10^{-2} \unit{\newton}}. It also confirms the answer from part (a). 


\pagebreak
\section*{Problem 23}
\begin{wrapfigure}{r}{0.25\textwidth}
    \vspace{-30pt}
    \includegraphics[width=0.25\textwidth]{picture_6.png} 
    % \label{fig:wrapfig}
\end{wrapfigure}
Fig. 21-32, particles 1 and 2 of charge $q_1 = q_2 = +3.20 \times 10^{-19} \unit{\coulomb}$ are on a y axis at distance $d = 17.0 \unit{\centi\meter}$ from the origin. Particle 3 of charge $q_3 = +6.40 \times 10^{-19} \unit{\coulomb}$ is moved gradually along the x-axis from x = 0 to x = +5.0 m. At what values of x will the magnitude of the electrostatic force on the third particle from the other two particles be (a) minimum and (b) maximum? What are the (c) minimum and (d) maximum magnitudes?

\subsection*{Solution}
% \subsubsection*{(a) Position for minimum force}
% The formula for electrostatic force is $F = \frac{k |q_1| |q_2|}{||r||^2}$. This means the formula for the total electrostatic force on particle 3 is negatively dependant to its distance from the other two particles. Since it is always equally distant from the two other particles vertically and they have the same charge, it experiences no net vertical force. As a result, the minimum magnitude is where $r^2$ is maximum, so at \boxed{r = 5\unit{\meter}}.

% \subsubsection*{(b) Position for maximum force}
% The formula for electrostatic force is $F = \frac{k |q_1| |q_2|}{||r||^2}$. This means the formula for the total electrostatic force on particle 3 is negatively dependant to its distance from the other two particles. Since it is always equally distant from the two other particles vertically and they have the same charge, it experiences no net vertical force. As a result, the maximum magnitude is where $r^2$ is minimum, so at \boxed{r = 0\unit{\meter}}.

% \subsubsection*{(c) Minimum force}
% \begin{align*}
%     F   &=  \frac{k q_1 q_3}{(x^2 + y^2)^{3/2}}\hat{r}_{13} + \frac{k q_2 q_3}{(x^2 + y^2)^{3/2}}\hat{r}_{23}\\
%         &=  \frac{k(3.20 \times 10^{-19})(6.40 \times 10^{-19})}{(0.17^2 + 5^2)^{3/2}}\left(\begin{bmatrix} 5 \\ -0.17 \end{bmatrix} + \begin{bmatrix} 5 \\ 0.17 \end{bmatrix}\right)\\
%     F_x &=  \frac{k * 2.048 \times 10^{-37}}{25.0289^{3/2}} * 10
%         =   \frac{8.99 * 2.048 \times 10^{-27}}{25.0289^{3/2}} * 10
%         =   \boxed{1.47037 \times 10^{-28} \unit{\newton}}
% \end{align*}

\pagebreak
\section*{Problem 31}
\includegraphics[width=\textwidth]{picture_7.png} 

\subsection*{Solution}
We assume that the earth has a surface of roughly $5.1 \times 10^{14} \unit{\meter^2}$ and that a proton has a charge of $1.602 \times 10^{-19} \unit{\coulomb}$, we can calculate the total electric current, with the formula $i = \frac{dq}{dt}$. The total current in a square meter per second is the number of protons per second times the charge of a proton.
\begin{equation*}
    i_{m^2} = 1.5 \times 10^3 * 1.602 \times 10^{-19} = 2.403 \times 10^{-16} \unit{\frac{\coulomb}{\meter^2 \second}}
\end{equation*}
Multiplying this by the surface area of the earth, we get the answer.
\begin{equation*}
    i = 5.1 \times 10^{14} * 2.403 \times 10^{-16} = 1.22553 \times 10^1 \times 10^{-2} = \boxed{1.22553 \times 10^{-1} \unit{\coulomb/\second}}
\end{equation*}

\pagebreak
\section*{Problem 36}
\includegraphics[width=\textwidth]{picture_8.png} 

\subsection*{Solution}
a) Positron. \\
b) Electron.

\end{document}