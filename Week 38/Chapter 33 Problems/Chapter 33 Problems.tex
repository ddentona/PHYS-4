\documentclass[12pt]{article}
\usepackage{amsmath}
\usepackage{amssymb}
\usepackage{cancel}
\usepackage{graphicx}
% \usepackage{physics}
\usepackage{siunitx}
\usepackage{wrapfig}

% \AtBeginDocument{\RenewCommandCopy\qty\SI}

\newcommand{\E}[1]{\times 10^{#1}}

\title{
    Chapter 33 End-of-Chapter Problems
    \\ \small
    Halliday \& Resnick, 10th Edition
}

\author{Donald Aingworth IV}

\date{\small Hit me where it Matters}

\begin{document}
    \DeclareSIUnit{\atm}{atm}
    \DeclareSIUnit{\cal}{\ cal}
    \DeclareSIUnit{\Cal}{\ Cal}
    \DeclareSIUnit{\calorie}{\ cal}
    \DeclareSIUnit{\Calorie}{\ Cal}
    \DeclareSIUnit{\celsiusdegree}{C^\circ}
    \DeclareSIUnit{\fahrenheit}{^\circ F}
    \DeclareSIUnit{\fahrenheitdegree}{F^\circ}
    \DeclareSIUnit{\torr}{\ torr}

    \maketitle

    \pagebreak
    \section{Problem 1}
        A certain helium-neon laser emits red light in a narrow band of wavelengths centered at 632.8 nm and with a “wavelength width” (such as on the scale of Fig. 33-1) of 0.0100 nm. 
        What is the corresponding “frequency width” for the emission?

        \subsection{Solution}
            Use the traditional formula for the wavelength.
            Here, the speed of the wave is the speed of light.
            We can treat this like an error and raw value issue.
            \begin{gather}
                v   =   \lambda f   \to
                f   =   \frac{c}{\lambda}\\
                \frac{\delta f}{f} = \frac{\delta \lambda}{\lambda}\\
                \begin{align}
                    \delta f    &=  f * \frac{\delta \lambda}{\lambda}
                        =   c * \frac{\delta \lambda}{\lambda^2}\\
                        &=  2.998\E{8}\,\unit{\meter/\second} * \frac{0.0100\,\unit{\nano\meter}}{(632.8\,\unit{\nano\meter})^2}\\
                        &=  \boxed{7.49\,\unit{\giga\hertz}}
                \end{align}
            \end{gather}

    \pagebreak
    \section{Problem 5}

        \subsection{Solution}

    \pagebreak
    \section{Problem 7}

        \subsection{Solution}

    \pagebreak
    \section{Problem 9}

        \subsection{Solution}

    \pagebreak
    \section{Problem 11}

        \subsection{Solution}

    \pagebreak
    \section{Problem 13}

        \subsection{Solution}

    \pagebreak
    \section{Problem 19}

        \subsection{Solution}

    \pagebreak
    \section{Problem 21}

        \subsection{Solution}

    \pagebreak
    \section{Problem 29}

        \subsection{Solution}

    \pagebreak
    \section{Problem 33}

        \subsection{Solution}

    \pagebreak
    \section{Problem 37}

        \subsection{Solution}

    \pagebreak
    \section{Problem 41}

        \subsection{Solution}

    \pagebreak
    \section{Problem 43}

        \subsection{Solution}

    \pagebreak
    \section{Problem 45}

        \subsection{Solution}

    \pagebreak
    \section{Problem 49}

        \subsection{Solution}

    \pagebreak
    \section{Problem 51}

        \subsection{Solution}

    \pagebreak
    \section{Problem 55}

        \subsection{Solution}

    \pagebreak
    \section{Problem 59}

        \subsection{Solution}

    \pagebreak
    \section{Problem 65}

        \subsection{Solution}

    \pagebreak
    \section{Problem 69}

        \subsection{Solution}

    \pagebreak
    \section{Problem 71}

        \subsection{Solution}

    \pagebreak
    \section{Problem 75}

        \subsection{Solution}

    \pagebreak
    \section{Problem 83}

        \subsection{Solution}

    \pagebreak
    \section{Problem 87}

        \subsection{Solution}

    \pagebreak
    \section{Problem 89}

        \subsection{Solution}

    \pagebreak
    \section{Problem 91}

        \subsection{Solution}

    \pagebreak
    \section{Problem 105}

        \subsection{Solution}

    \pagebreak
    \tableofcontents
\end{document}