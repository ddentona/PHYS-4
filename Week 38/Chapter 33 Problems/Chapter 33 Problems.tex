\documentclass[12pt]{article}
\usepackage{amsmath}
\usepackage{amssymb}
\usepackage{cancel}
\usepackage{graphicx}
% \usepackage{physics}
\usepackage{siunitx}
\usepackage{wrapfig}

% \AtBeginDocument{\RenewCommandCopy\qty\SI}

\newcommand{\E}[1]{\times 10^{#1}}

\title{
    Chapter 33 End-of-Chapter Problems
    \\ \small
    Halliday \& Resnick, 10th Edition
}

\author{Donald Aingworth IV}

\date{\small Hit me where it Matters}

\begin{document}
    \DeclareSIUnit{\atm}{atm}
    \DeclareSIUnit{\cal}{\ cal}
    \DeclareSIUnit{\Cal}{\ Cal}
    \DeclareSIUnit{\calorie}{\ cal}
    \DeclareSIUnit{\Calorie}{\ Cal}
    \DeclareSIUnit{\celsiusdegree}{C^\circ}
    \DeclareSIUnit{\fahrenheit}{^\circ F}
    \DeclareSIUnit{\fahrenheitdegree}{F^\circ}
    \DeclareSIUnit{\torr}{\ torr}

    \maketitle

    \pagebreak
    \section{Problem 1}
        A certain helium-neon laser emits red light in a narrow band of wavelengths centered at 632.8 nm and with a “wavelength width” (such as on the scale of Fig. 33-1) of 0.0100 nm. 
        What is the corresponding “frequency width” for the emission?

        \subsection{Solution}
            Use the traditional formula for the wavelength.
            Here, the speed of the wave is the speed of light.
            We can treat this like an error and raw value issue.
            \begin{gather}
                v   =   \lambda f   \to
                f   =   \frac{c}{\lambda}\\
                \frac{\delta f}{f} = \frac{\delta \lambda}{\lambda}\\
                \begin{align}
                    \delta f    &=  f * \frac{\delta \lambda}{\lambda}
                        =   c * \frac{\delta \lambda}{\lambda^2}\\
                        &=  2.998\E{8}\,\unit{\meter/\second} * \frac{0.0100\,\unit{\nano\meter}}{(632.8\,\unit{\nano\meter})^2}\\
                        &=  \boxed{7.49\,\unit{\giga\hertz}}
                \end{align}
            \end{gather}

    \pagebreak
    \section{Problem 5}
        What inductance must be connected to a 17 pF capacitor in an oscillator capable of generating 550 nm (i.e., visible) electromagnetic waves? 
        Comment on your answer.

        \subsection{Solution}
            For an $LC$ circuit, the angular frequency is $\frac{1}{\sqrt{LC}}$.
            \begin{equation}
                \omega  =   \frac{1}{\sqrt{LC}}
            \end{equation}

            The linear frequency is calculatable from this.
            \begin{equation}
                f   =   \frac{\omega}{2\pi}
                    =   \frac{1}{2\pi\sqrt{LC}}
            \end{equation}

            This can be relatable to the wave speed (the speed of light in the case of an EM wave).
            \begin{gather}
                v   =   \lambda f\\
                c   =   \lambda f
                    =   \frac{\lambda}{2\pi\sqrt{LC}}
            \end{gather}

            Ths can be solved for the inductance ($L$) and found a solution for.
            \begin{gather}
                c   =   \frac{\lambda}{2\pi\sqrt{L}\,\sqrt{C}}\\
                \sqrt{L}    =   \frac{\lambda}{2\pi c\sqrt{C}}\\
                \begin{align}
                    L   &=  \frac{\lambda^2}{(2\pi)^2 c^2 C}\\
                        &=  \frac{(550\,\unit{\nano\meter})^2}{4\pi^2 (2.998\E{8}\,\unit{\meter/\second})^2 * 17\,\unit{\pico\farad}}\\
                        &=  \boxed{5.015\E{-21}\,\unit{\henry}}
                \end{align}
            \end{gather}

    \pagebreak
    \section{Problem 7}
        What is the intensity of a traveling plane electromagnetic wave if $B_m$ is $1.0\E{-4}\,\unit{\tesla}$?

        \subsection{Solution}

    \pagebreak
    \section{Problem 9}
        Some neodymium-glass lasers can provide $100\,\unit{\tera\watt}$ of power in 1.0 ns pulses at a wavelength of 0.26 \unit{\micro\meter}. 
        How much energy is contained in a single pulse?

        \subsection{Solution}

    \pagebreak
    \section{Problem 11}
        A plane electromagnetic wave traveling in the positive direction of an x axis in vacuum has components $E_x = E_y = 0$ and $E_z$ has the below value. 
        \begin{equation}
            E_z = (2.0\,\unit{\volt/\meter})\cos\left[ (\pi \E{15}\,\unit{\second^{-1}})(t - x/c) \right] 
        \end{equation}
        (a) What is the amplitude of the magnetic field component? 
        (b) Parallel to which axis does the magnetic field oscillate? 
        (c) When the electric field component is in the positive direction of the z axis at a certain point P, what is the direction of the magnetic field component there?

        \subsection{Solution}

    \pagebreak
    \section{Problem 13}

        \subsection{Solution}

    \pagebreak
    \section{Problem 19}

        \subsection{Solution}

    \pagebreak
    \section{Problem 21}

        \subsection{Solution}

    \pagebreak
    \section{Problem 29}

        \subsection{Solution}

    \pagebreak
    \section{Problem 33}

        \subsection{Solution}

    \pagebreak
    \section{Problem 37}

        \subsection{Solution}

    \pagebreak
    \section{Problem 41}

        \subsection{Solution}

    \pagebreak
    \section{Problem 43}

        \subsection{Solution}

    \pagebreak
    \section{Problem 45}

        \subsection{Solution}

    \pagebreak
    \section{Problem 49}

        \subsection{Solution}

    \pagebreak
    \section{Problem 51}

        \subsection{Solution}

    \pagebreak
    \section{Problem 55}

        \subsection{Solution}

    \pagebreak
    \section{Problem 59}

        \subsection{Solution}

    \pagebreak
    \section{Problem 65}

        \subsection{Solution}

    \pagebreak
    \section{Problem 69}

        \subsection{Solution}

    \pagebreak
    \section{Problem 71}

        \subsection{Solution}

    \pagebreak
    \section{Problem 75}

        \subsection{Solution}

    \pagebreak
    \section{Problem 83}

        \subsection{Solution}

    \pagebreak
    \section{Problem 87}

        \subsection{Solution}

    \pagebreak
    \section{Problem 89}

        \subsection{Solution}

    \pagebreak
    \section{Problem 91}

        \subsection{Solution}

    \pagebreak
    \section{Problem 105}

        \subsection{Solution}

    \pagebreak
    \tableofcontents
\end{document}