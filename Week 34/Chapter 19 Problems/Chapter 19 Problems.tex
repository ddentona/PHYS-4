\documentclass[12pt]{article}
\usepackage{amsmath}
\usepackage{amssymb}
\usepackage{graphicx}
\usepackage{physics}
\usepackage{siunitx}
\usepackage{wrapfig}

\AtBeginDocument{\RenewCommandCopy\qty\SI}

\begin{document}
    \DeclareSIUnit{\atm}{atm}
    \DeclareSIUnit{\cal}{\ cal}
    \DeclareSIUnit{\Cal}{\ Cal}
    \DeclareSIUnit{\calorie}{\ cal}
    \DeclareSIUnit{\Calorie}{\ Cal}
    \DeclareSIUnit{\fahrenheit}{^\circ F}
    \DeclareSIUnit{\torr}{\ torr}

    \section{Problem 1}
        Find the mass in kilograms of $7.50 \times 10^{24}$ atoms of arsenic, which has a molar mass of $74.9 \unit{\gram/\mole}$.

        \subsection{Solution}
            Convert atoms to moles.
            \begin{equation}
                \frac{7.50 \times 10^{24} \text{ atoms}}{6.02 \times 10^{23} \text{ atoms/mol}} = 12.458 \unit{\mole}
            \end{equation}

            Using the molar mass, convert moles to grams.
            \begin{equation}
                12.458 \unit{\mole} * 74.9 \unit{\gram/\mole} = 933.140 \unit{\gram} = \boxed{0.933 \unit{\kilo\gram}}
            \end{equation}

    \pagebreak
    \section{Problem 3}
        Oxygen gas having a volume of $1000 \unit{\centi\meter^3}$ at $40.0\unit{\celsius}$ and $1.01 \times 10^5 \unit{\pascal}$ expands until its volume is $1500 \unit{\centi\meter^3}$ and its pressure is $1.06 \times 10^5 \unit{\pascal}$. 
        Find (a) the number of moles of oxygen present and (b) the final temperature of the sample.

        \subsection{Solution (a)}
            We can use the ideal gas law for this.
            We apply it to the first case, converting the 40.0\unit{\celsius} to Kelvin.
            \begin{align}
                40\unit{\celsius}   &=  313.15 \unit{\kelvin}\\
                1000 \unit{\centi\meter^3}  &=  1000 \times 10^{-6} \unit{\meter^3} = 1 \times 10^{-3} \unit{\meter^3}\\
                pV  &=  nRT\\
                n   &=  \frac{pV}{RT}
                    =   \frac{1.01 \times 10^5 \unit{\pascal} * 10^{-3} \unit{\centi\meter^3}}{8.31 \unit{\joule/\mole\cdot\kelvin} * 313.15 \unit{\kelvin}}\\
                    &=  \frac{1.01 \times 10^2 \unit{\newton \cdot \meter}}{2602.2765 \unit{\joule/\mole}}
                    =   \boxed{0.038812 \unit{\mole}}
            \end{align}

        \subsection{Solution (b)}
            The ideal gas law (or an equivalent) will be used here.
            The number of moles does not change here, neither does the gas constant $R$. 
            We can use ths to solve for the final value of the temperature.
            \begin{gather}
                \frac{p_1 V_1}{n_1 T_1} =   \frac{p_2 V_2}{n_2 T_2}\\
                \frac{p_1 V_1}{T_1} =   \frac{p_2 V_2}{T_2}\\
                T_2 =   T_1 * \frac{p_2 V_2}{p_1 V_1}
            \end{gather}

            We can substitute in values now.
            \begin{align}
                T_2 &=  T_1 * \frac{p_2 V_2}{p_1 V_1}
                    =   313.15 * \frac{1.06 \times 10^5 * 1500}{1.01 \times 10^5 * 1000}\\
                    &=  313.15 * \frac{1.06 * 1.5}{1.01}
                    =   \boxed{492.97 \unit{\kelvin} \approx 220 \unit{\celsius}}
            \end{align}

    \pagebreak
    \section{Problem 5}
        The best laboratory vacuum has a pressure of about $1.00 \times 10^{-18} \unit{\atm}$, or $1.01 \times 10^{-13} \unit{\pascal}$. 
        How many gas molecules are there per cubic centimeter in such a vacuum at 293 K?

        \subsection{Solution}
            Use the ideal gas law, the version with Boltzmann's constant.
            We can solve for $\frac{N}{V}$.
            \begin{gather}
                pV  =   NkT\\
                \frac{N}{V} =   \frac{p}{kT}
            \end{gather}

            From here, we can just plug and chug, so to speak.
            \begin{align}
                \frac{N}{V} &=  \frac{1.01 \times 10^{-13} \unit{\newton/\meter^2}}{1.38 \times 10^{-23} \unit{\newton\cdot\meter/\kelvin} * 293 \unit{\kelvin}}
                    =   \frac{1.01 \times 10^{10}}{404.34} \unit{\meter^{-3}}\\
                    &=  24978978.09 \times \unit{\meter^{-3}}
                    =   \boxed{24.979 \unit{\centi\meter^{-3}}}
            \end{align}

    \pagebreak
    \section{Problem 7}
        Suppose 1.80 mol of an ideal gas is taken from a volume of 3.00 \unit{\meter^3} to a volume of 1.50 \unit{\meter^3} via an isothermal compression at 30\unit{\celsius}. 
        (a) How much energy is transferred as heat during the compression, and (b) is the transfer to or from the gas?

        \subsection{Solution}

    \pagebreak
    \section{Problem 9}
        An automobile tire has a volume of $1.64 \times 10^{-2} \unit{\meter^3}$ and contains air at a gauge pressure (pressure above atmospheric pressure) of 165 kPa when the temperature is 0.00\unit{\celsius}. 
        What is the gauge pressure of the air in the tires when its temperature rises to 27.0\unit{\celsius} and its volume increases to $1.67 \times 10^{-2} \unit{\meter^3}$? 
        Assume atmospheric pressure is $1.01 \times 10^5 \unit{\pascal}$.

        \subsection{Solution}

    \pagebreak
    \section{Problem 11}
        Air that initially occupies 0.140 m3 at a
gauge pressure of 103.0 kPa is expanded isothermally to a pressure
of 101.3 kPa and then cooled at constant pressure until it reaches
its initial volume. Compute the work done by the air. (Gauge pres-
sure is the difference between the actual pressure and atmospheric
pressure.)

        \subsection{Solution}

    \pagebreak
    \section{Problem 13}
        A sample of an ideal gas is
taken through the cyclic process abca
shown in Fig. 19-20. The scale of the
vertical axis is set by pb = 7.5 kPa and
p ac = 2.5 kPa. At point a, T = 200 K.
(a) How many moles of gas are in
the sample? What are (b) the tem-
perature of the gas at point b, (c) the
temperature of the gas at point c, and
(d) the net energy added to the gas as
heat during the cycle?

        \subsection{Solution}

    \pagebreak
    \section{Problem 17}
        Container A in Fig. 19-22
holds an ideal gas at a pressure of
$5.0 \times 10 5 Pa$ and a temperature of
300 K. It is connected by a thin tube
(and a closed valve) to container B,
with four times the volume of A.
Container B holds the same ideal
gas at a pressure of $1.0 \times 10 5 Pa$ and
a temperature of 400 K. The valve is
opened to allow the pressures to equalize, but the temperature of
each container is maintained. What then is the pressure?

        \subsection{Solution}

    \pagebreak
    \section{Problem 19}
        (a) Compute the rms speed of a nitrogen molecule at 20.0\unit{\celsius}. 
        The molar mass of nitrogen molecules ($N_2$) is given in Table 19-1.
        At what temperatures will the rms speed be (b) half that value and (c) twice that value?

        \subsection{Solution}

    \pagebreak
    \section{Problem 23}

        \subsection{Solution}

    \pagebreak
    \section{Problem 25}
        Determine the average value of the translational kinetic energy of the gas's molecules of an ideal gas at temperatures (a) 0.00\unit{\celsius} and (b) 100\unit{\celsius}.
        What is the translational kinetic energy per mole of an ideal gas at (c) 0.00\unit{\celsius} and (d) 100\unit{\celsius}?

        \subsection{Solution}

    \pagebreak
    \section{Problem 27}

        \subsection{Solution}

    \pagebreak
    \section{Problem 31}
        In a certain particle accelerator, protons travel around a circular path of diameter 23.0 m in an evacuated chamber, whose residual gas is at 295 K and $1.00 \times 10^{-6} \unit{\torr}$ pressure. (a) Calculate the number of gas molecules per cubic centimeter at this pressure.
(b) What is the mean free path of the gas molecules if the molecu-
lar diameter is $2.00 \times 10^{-8} \unit{\centi\meter}$?

        \subsection{Solution}

    \pagebreak
    \section{Problem 35}
        Ten particles are moving with the following speeds: four at $200 \unit{\meter/\second}$, two at $500 \unit{\meter/\second}$, and four at $600 \unit{\meter/\second}$. 
        Calculate their (a) average and (b) rms speeds. 
        (c) Is $v_{rms} > v_{avg}$?

        \subsection{Solution}

    \pagebreak
    \section{Problem 37}

        \subsection{Solution}

    \pagebreak
    \section{Problem 39}

        \subsection{Solution}

    \pagebreak
    \section{Problem 43}
        The temperature of 3.00 mol of an ideal diatomic gas is increased by 40.0 C° without the pressure of the gas changing.
        The molecules in the gas rotate but do not oscillate. 
        (a) How much energy is transferred to the gas as heat? 
        (b) What is the change in the internal energy of the gas? 
        (c) How much work is done by the gas? 
        (d) By how much does the rotational kinetic energy of the gas increase?

        \subsection{Solution}

    \pagebreak
    \section{Problem 45}

        \subsection{Solution}

    \pagebreak
    \section{Problem 47}

        \subsection{Solution}

    \pagebreak
    \section{Problem 51}
        When 1.0 mol of oxygen ($O_2$) gas is heated at constant pressure starting at 0°C, how much energy must be added to the gas as heat to double its volume? 
        The molecules rotate but do not oscillate.

        \subsection{Solution}

    \pagebreak
    \section{Problem 55}

        \subsection{Solution}

    \pagebreak
    \section{Problem 57}

        \subsection{Solution}

    \pagebreak
    \section{Problem 59}

        \subsection{Solution}

    \pagebreak
    \section{Problem 63}

        \subsection{Solution}

    \pagebreak
    \section{Problem 69}

        \subsection{Solution}

    \pagebreak
    \section{Problem 75}

        \subsection{Solution}

    \pagebreak
    \section{Problem 77}

        \subsection{Solution}

    \pagebreak
    
    \tableofcontents
\end{document}