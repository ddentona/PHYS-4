\documentclass[12pt]{article}
\usepackage{amsmath}
\usepackage{amssymb}
\usepackage{graphicx}
\usepackage{physics}
\usepackage{siunitx}
\usepackage{wrapfig}

\AtBeginDocument{\RenewCommandCopy\qty\SI}

\begin{document}
    \DeclareSIUnit{\atm}{atm}
    \DeclareSIUnit{\cal}{\ cal}
    \DeclareSIUnit{\Cal}{\ Cal}
    \DeclareSIUnit{\calorie}{\ cal}
    \DeclareSIUnit{\Calorie}{\ Cal}
    \DeclareSIUnit{\fahrenheit}{^\circ F}
    \DeclareSIUnit{\torr}{\ torr}

    \section{Problem 1}
        Find the mass in kilograms of $7.50 \times 10^{24}$ atoms of arsenic, which has a molar mass of $74.9 \unit{\gram/\mole}$.

        \subsection{Solution}
            Convert atoms to moles.
            \begin{equation}
                \frac{7.50 \times 10^{24} \text{ atoms}}{6.02 \times 10^{23} \text{ atoms/mol}} = 12.458 \unit{\mole}
            \end{equation}

            Using the molar mass, convert moles to grams.
            \begin{equation}
                12.458 \unit{\mole} * 74.9 \unit{\gram/\mole} = 933.140 \unit{\gram} = \boxed{0.933 \unit{\kilo\gram}}
            \end{equation}

    \pagebreak
    \section{Problem 3}
        Oxygen gas having a volume of $1000 \unit{\centi\meter^3}$ at $40.0\unit{\celsius}$ and $1.01 \times 10^5 \unit{\pascal}$ expands until its volume is $1500 \unit{\centi\meter^3}$ and its pressure is $1.06 \times 10^5 \unit{\pascal}$. 
        Find (a) the number of moles of oxygen present and (b) the final temperature of the sample.

        \subsection{Solution}

    \pagebreak
    \section{Problem 5}
        The best laboratory vacuum has a pressure of about $1.00 \times 10^{-18} \unit{\atm}$, or $1.01 \times 10^{-13} \unit{\pascal}$. 
        How many gas molecules are there per cubic centimeter in such a vacuum at 293 K?

        \subsection{Solution}

    \pagebreak
    \section{Problem 7}

        \subsection{Solution}

    \pagebreak
    \section{Problem 9}

        \subsection{Solution}

    \pagebreak
    \section{Problem 11}

        \subsection{Solution}

    \pagebreak
    \section{Problem 13}

        \subsection{Solution}

    \pagebreak
    \section{Problem 17}

        \subsection{Solution}

    \pagebreak
    \section{Problem 19}
        (a) Compute the rms speed of a nitrogen molecule at 20.0\unit{\celsius}. 
        The molar mass of nitrogen molecules ($N_2$) is given in Table 19-1.
        At what temperatures will the rms speed be (b) half that value and (c) twice that value?

        \subsection{Solution}

    \pagebreak
    \section{Problem 23}

        \subsection{Solution}

    \pagebreak
    \section{Problem 25}
        Determine the average value of the translational kinetic energy of the molecules of an ideal gas at temperatures (a) 0.00\unit{\celsius} and (b) 100\unit{\celsius}.
        What is the translational kinetic energy per mole of an ideal gas at (c) 0.00\unit{\celsius} and (d) 100\unit{\celsius}?

        \subsection{Solution}

    \pagebreak
    \section{Problem 27}

        \subsection{Solution}

    \pagebreak
    \section{Problem 31}
        In a certain particle accelerator, protons travel around a circular path of diameter 23.0 m in an evacuated chamber, whose residual gas is at 295 K and $1.00 \times 10^{-6} \unit{\torr}$ pressure. (a) Calculate the number of gas molecules per cubic centimeter at this pressure.
(b) What is the mean free path of the gas molecules if the molecu-
lar diameter is $2.00 \times 10^{-8} \unit{\centi\meter}$?

        \subsection{Solution}

    \pagebreak
    \section{Problem 35}
        Ten particles are moving with the following speeds: four at $200 \unit{\meter/\second}$, two at $500 \unit{\meter/\second}$, and four at $600 \unit{\meter/\second}$. 
        Calculate their (a) average and (b) rms speeds. 
        (c) Is $v_{rms} > v_{avg}$?

        \subsection{Solution}

    \pagebreak
    \section{Problem 37}

        \subsection{Solution}

    \pagebreak
    \section{Problem 39}

        \subsection{Solution}

    \pagebreak
    \section{Problem 43}
        The temperature of 3.00 mol of an ideal diatomic gas is increased by 40.0 C° without the pressure of the gas changing.
        The molecules in the gas rotate but do not oscillate. 
        (a) How much energy is transferred to the gas as heat? 
        (b) What is the change in the internal energy of the gas? 
        (c) How much work is done by the gas? 
        (d) By how much does the rotational kinetic energy of the gas increase?

        \subsection{Solution}

    \pagebreak
    \section{Problem 45}

        \subsection{Solution}

    \pagebreak
    \section{Problem 47}

        \subsection{Solution}

    \pagebreak
    \section{Problem 51}
        When 1.0 mol of oxygen ($O_2$) gas is heated at constant pressure starting at 0°C, how much energy must be added to the gas as heat to double its volume? 
        The molecules rotate but do not oscillate.

        \subsection{Solution}

    \pagebreak
    \section{Problem 55}

        \subsection{Solution}

    \pagebreak
    \section{Problem 57}

        \subsection{Solution}

    \pagebreak
    \section{Problem 59}

        \subsection{Solution}

    \pagebreak
    \section{Problem 63}

        \subsection{Solution}

    \pagebreak
    \section{Problem 69}

        \subsection{Solution}

    \pagebreak
    \section{Problem 75}

        \subsection{Solution}

    \pagebreak
    \section{Problem 77}

        \subsection{Solution}

    \pagebreak
    
    \tableofcontents
\end{document}