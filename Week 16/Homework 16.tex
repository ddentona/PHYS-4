\documentclass[12pt]{article}
\usepackage{amsmath}
\usepackage{array}
% \usepackage{gensymb}
\usepackage{geometry}
\usepackage{graphicx}
\usepackage{pgfplots}
\usepackage{siunitx}
\usepackage{wrapfig}

\title{Homework \#16}
\author{Donald Aingworth IV}
\date{December 11, 2024}

\pgfplotsset{width=8cm,compat=1.9}
\usepgfplotslibrary{external}
% \tikzexternalize

\begin{document}

\DeclareSIUnit{\mile}{mi}
\DeclareSIUnit{\gal}{gal}
\DeclareSIUnit{\foot}{ft}
\DeclareSIUnit{\hour}{h}
\DeclareSIUnit{\rad}{rad}
\DeclareSIUnit{\unit}{u}
\DeclareSIUnit{\dyne}{dyn}

\maketitle

\pagebreak
\section{Problem 1}
Two disks are mounted (like a merry-go-round) on low-friction bearings on the same axle and can be brought together so that they couple and rotate as one unit. The first disk, with rotational inertia 3.30 \unit{\kilo\gram*\meter^2} about its central axis, is set spinning counterclockwise at 450 rev/min. The second disk, with rotational inertia 6.60 \unit{\kilo\gram*\meter^2} about its central axis, is set spinning counterclockwise at 900 rev/min. They then couple together. (a) What is their angular speed after coupling? If instead the second disk is set spinning clockwise at 900 rev/min, what are their (b) angular speed and (c) direction of rotation after they couple together?

\subsection{Solution}
\subsubsection{Section (a)}
We have a concept called conservation of angular momentum. 
\begin{align}
    L_i &=  L_f\\
    L_f &=  l_1 + l_2
        =   I_1\omega_1 + I_2\omega_2\\
    \omega_f    &=  \frac{I_1\omega_1 + I_2\omega_2}{I_1 + I_2}
        =   \frac{3.3*450 + 6.6*900}{3.3 + 6.6}\\
        &=  \frac{1485 + 5940}{9.9}
        =   \boxed{750\text{rev/min}}
\end{align}

\subsubsection{Section (b)}
We just need to change a positive to a negative. 
\begin{align}
    \omega_f    &=  \frac{I_1\omega_1 + I_2\omega_2}{I_1 + I_2}
        =   \frac{3.3*450 - 6.6*900}{3.3 + 6.6}\\
        &=  \frac{1485 - 5940}{9.9}
        =   \boxed{-450\text{rev/min}}
\end{align}

\subsubsection{Section (c)}
Since the magnitude is negative and negative angular velocity corresponds to clockwise motion, the angular motion is \boxed{clockwise}.

\pagebreak
\section{Problem 2}
The Sun's mass is 2.0 \texttimes $10^{30}$ kg, its radius is 7.0 \texttimes $10^5$ km, and it has a rotational period of approximately 28 days. If the Sun should collapse into a white dwarf of radius 3.5 \texttimes $10^3$ km, what would its period be if no mass were ejected and a sphere of uniform density can model the Sun both before and after?

\subsection{Solution}
% We first find the moment of inertia of the sun, which is a solid sphere rotating about its diameter, so the formula is \( I = \frac{2}{5}MR^2 \). 
% \begin{align}
%     I   &=  \frac{2}{5}MR^2
%         =   \frac{2}{5} * (2.0 \times 10^{30}) * (7.0 \times 10^{5})^2
%         =   \frac{2}{5} * (2.0 \times 10^{30}) * (49.0 \times 10^{10})\\
%         &=  \frac{1}{5} * (98.0 \times 10^{40})
%         =   19.6 \times 10^{40} \unit{\kilo\gram*\meter^2}
% \end{align}

We can calculate the angular frequency of the sun by using the period formula \(T = \frac{2\pi}{\omega}\).
\begin{align}
    T   &=  \frac{2\pi}{\omega}\\
    \omega  &=  \frac{2\pi}{T}
\end{align}

Next, we can use the conservation of angular momentum and the formula for the inertia of the dwarf sun to find a formula for the final angular velocity and then final period.
\begin{align}
    L_f &=  L_i\\
    I_f\omega_f &=  I_i\omega_i\\
    I_f\frac{2\pi}{T_f} &=  I_i\frac{2\pi}{T_i}\\
    \frac{I_f}{I_i}\cdot\frac{2\pi}{2\pi}   &=  \frac{T_f}{T_i}\\
    \frac{I_f}{I_i}*T_i &=  T_f\\
    \frac{\frac{2}{5}MR_f^2}{\frac{2}{5}MR_i^2}*T_i =
    \frac{R_f^2}{R_i^2}*T_i =
    \frac{(3.5 \times 10^{3})^2}{(7.0 \times 10^{5})^2}*28\text{days}   &=  T_f\\
    \frac{12.25 \times 10^{6}}{49.0 \times 10^{10}}*28\text{days}   =
    \frac{28\text{days}}{4 \times 10^4} =
    7 \times 10^{-4} \text{days}    &=  T_f
\end{align}

This means that the period is \boxed{7 \times 10^{-4} \text{ days}}.
\pagebreak
\section{Problem 3}
The displacement from equilibrium of a particle is given by \(x(t) = A \cos\left(\omega t - \frac{\pi}{3}\right)\). Which, if any, of the following are equivalent expressions:
\begin{align}
    a)\ x(t)    &=  A\cos\left(\omega t + \frac{\pi}{3}\right)\\
    b)\ x(t)    &=  A\cos\left(\omega t + \frac{5\pi}{3}\right)\\
    c)\ x(t)    &=  A\cos\left(\omega t + \frac{\pi}{6}\right)\\
    d)\ x(t)    &=  A\cos\left(\omega t - \frac{5\pi}{6}\right)
\end{align}

\subsection{Solution}
We can see that the only change here is the part labeled $\phi$ in the format of simple harmonic motion. For an equivalent value, the value of the cosine must be the same at every point, which can only be true if \(\phi = -\frac{\pi}{3} \mod 2\pi\).

\begin{tabular}{c | c | c | c}
        &   $\phi$  &   $\phi\mod 2\pi$ &   Correct?\\ \hline
        &   $-\frac{\pi}{3}$    &   $\frac{5\pi}{3}$    &   Yes\\
    a)  &   $\frac{\pi}{3}$     &   $\frac{\pi}{3}$     &   No\\
    b)  &   $\frac{5\pi}{3}$    &   $\frac{5\pi}{3}$    &   Yes\\
    c)  &   $\frac{\pi}{6}$     &   $\frac{\pi}{6}$     &   No\\
    d)  &   $\frac{5\pi}{6}$    &   $\frac{5\pi}{6}$    &   No
\end{tabular}

\pagebreak
\section{Problem 4}
In a block and spring system $m = 0.250 \unit{\kilo\gram}$ and $k = 4.00 \unit{\newton/\meter}$. At $t = 0.150 s$, the velocity is$v = -0.174 \unit{\meter/\second}$ and the acceleration $a = +0.877 \unit{\meter/\second^2}$. Write an expression for the displacement as a function of time, x(t). (Hint, remember that the inverse tan function only returns the principal value, but there is a secondary value as well.)

\subsection{Solution}



\pagebreak
\section{Problem 8}
A uniform rod of mass $M$ and length $L = 1.20 \unit{\meter}$ oscillates about a horizontal axis at one end. What is the length of the simple pendulum that would have the same period? The rotational inertia is $\frac{ML^2}{3}$.

\subsection{Solution}

\pagebreak
\section{Problem 4}
In a block and spring system $m = 0.250 \unit{\kilo\gram}$ and $k = 4.00 \unit{\newton/\meter}$. At $t = 0.150 \unit{\second}$, the velocity is $v = -0.174 \unit{\meter/\second}$ and the acceleration $a = +0.877 \unit{\meter/\second^2}$. Write an expression for the displacement as a function of time, $x(t)$. (Hint, remember that the inverse tan function only returns the principal value, but there is a secondary value as well.)

\subsection{Solution}
We have some formulas for velocity and acceleration that we can use.
\begin{align}
    \omega  &=  \sqrt{\frac{k}{m}}
        =   \sqrt{\frac{4.0}{0.25}} =   \sqrt{4^2}
        =   4\unit{\second^{-1}}\\
    v(t)    &=  -\omega x_m \sin(\omega t + \phi)\ \ \rightarrow
    v(0.15) =   -0.174 \unit{\meter/\second}
        =   -4 x_m \sin(0.6 + \phi)\\
    a(t)    &=  -\omega^2 x_m \cos(\omega t + \phi)\rightarrow
    a(0.15) =   \ 0.877 \unit{\meter/\second^2}
        =   -16 x_m \cos(0.6 + \phi)\\
    \frac{a(0.15)}{v(0.15)} &=  \frac{-16 x_m \cos(0.6 + \phi)}{-4 x_m \sin(0.6 + \phi)}
        =   4*\frac{\cos(0.6 + \phi)}{\sin(0.6 + \phi)}\\
    \tan(\phi)  &=  \frac{\sin(\phi)}{\cos(\phi)}
        =   \frac{v(0)\sqrt{k}}{a(0)\sqrt{m}}\\
    0.6 + \phi  &=  \arctan\left(4*\frac{v(0)}{a(0)}\right)
        =   \arctan\left(4*\frac{-0.174}{0.877}\right)\\
        &=  \arctan\left(-\frac{0.696}{0.877}\right)
        =   \begin{matrix} 3.812 \\ 6.954 \end{matrix}
\end{align}

One of these is in the second quadrant, the other is in the fourth quadrant. Knowing that $\omega$ is positive and trusting that $x_m$ is positive, since the negative cosine is positive and the negative sine is negative, the cosine is negative and the sine is positive, so $0.6 + \phi$ is in the second quadrant. This means $0.6 + \phi = 3.812$ and $\phi = 3.212$. Last, we just neeed to find the value of $x_m$, which we will find using the value of $a(0)$.
\begin{align}
    a(0.15)    &=  -16 x_m \cos(0.6 + 3.212)\\
    x_m &=  -\frac{a(0)}{16\cos(3.812)}
        =   \frac{0.877}{0.7833}
        =   0.06998\unit{\meter}
\end{align}

Lastly, we find the value of $\omega$ and use that to finalize the formula for $x(t)$. 
\begin{equation}
    \boxed{x(t) =  0.06998*\cos(4t + 3.212)}
\end{equation}
\end{document}