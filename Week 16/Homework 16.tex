\documentclass[12pt]{article}
\usepackage{amsmath}
\usepackage{array}
% \usepackage{gensymb}
\usepackage{geometry}
\usepackage{graphicx}
\usepackage{pgfplots}
\usepackage{siunitx}
\usepackage{wrapfig}

\title{Homework \#16}
\author{Donald Aingworth IV}
\date{December 11, 2024}

\pgfplotsset{width=8cm,compat=1.9}
\usepgfplotslibrary{external}
% \tikzexternalize

\begin{document}

\DeclareSIUnit{\mile}{mi}
\DeclareSIUnit{\gal}{gal}
\DeclareSIUnit{\foot}{ft}
\DeclareSIUnit{\hour}{h}
\DeclareSIUnit{\rad}{rad}
\DeclareSIUnit{\unit}{u}
\DeclareSIUnit{\dyne}{dyn}

\maketitle

\pagebreak
\section{Problem 1}
Two disks are mounted (like a merry-go-round) on low-friction bearings on the same axle and can be brought together so that they couple and rotate as one unit. The first disk, with rotational inertia 3.30 \unit{\kilo\gram*\meter^2} about its central axis, is set spinning counterclockwise at 450 rev/min. The second disk, with rotational inertia 6.60 \unit{\kilo\gram*\meter^2} about its central axis, is set spinning counterclockwise at 900 rev/min. They then couple together. (a) What is their angular speed after coupling? If instead the second disk is set spinning clockwise at 900 rev/min, what are their (b) angular speed and (c) direction of rotation after they couple together?

\subsection{Solution}
\subsubsection{Section (a)}
We have a concept called conservation of angular momentum. 
\begin{align}
    L_i &=  L_f\\
    L_f &=  l_1 + l_2
        =   I_1\omega_1 + I_2\omega_2\\
    \omega_f    &=  \frac{I_1\omega_1 + I_2\omega_2}{I_1 + I_2}
        =   \frac{3.3*450 + 6.6*900}{3.3 + 6.6}\\
        &=  \frac{1485 + 5940}{9.9}
        =   \boxed{750\text{rev/min}}
\end{align}

\subsubsection{Section (b)}
We just need to change a positive to a negative. 
\begin{align}
    \omega_f    &=  \frac{I_1\omega_1 + I_2\omega_2}{I_1 + I_2}
        =   \frac{3.3*450 - 6.6*900}{3.3 + 6.6}\\
        &=  \frac{1485 - 5940}{9.9}
        =   \boxed{-450\text{rev/min}}
\end{align}

\subsubsection{Section (c)}
Since the magnitude is negative and negative angular velocity corresponds to clockwise motion, the angular motion is \boxed{clockwise}.

\section{Problem 2}
The Sun's mass is 2.0 \texttimes $10^{30}$ kg, its radius is 7.0 \texttimes $10^5$ km, and it has a rotational period of approximately 28 days. If the Sun should collapse into a white dwarf of radius 3.5 \texttimes $10^3$ km, what would its period be if no mass were ejected and a sphere of uniform density can model the Sun both before and after?

\subsection{Solution}

\end{document}