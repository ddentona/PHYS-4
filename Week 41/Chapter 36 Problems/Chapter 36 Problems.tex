\documentclass[12pt]{article}
\usepackage{amsmath}
\usepackage{amssymb}
\usepackage{cancel}
\usepackage{graphicx}
% \usepackage{physics}
\usepackage{siunitx}
\usepackage{wrapfig}

% \AtBeginDocument{\RenewCommandCopy\qty\SI}

\newcommand{\E}[1]{\times 10^{#1}}

\title{
    Chapter 36 End-of-Chapter Problems
    \\ \small
    Halliday \& Resnick, 10th Edition
}

\author{Donald Aingworth IV}

\date{\small Hit me where it Matters}

\begin{document}
    \DeclareSIUnit{\atm}{atm}
    \DeclareSIUnit{\cal}{\ cal}
    \DeclareSIUnit{\Cal}{\ Cal}
    \DeclareSIUnit{\calorie}{\ cal}
    \DeclareSIUnit{\Calorie}{\ Cal}
    \DeclareSIUnit{\celsiusdegree}{C^\circ}
    \DeclareSIUnit{\fahrenheit}{^\circ F}
    \DeclareSIUnit{\fahrenheitdegree}{F^\circ}
    \DeclareSIUnit{\torr}{\ torr}

    \maketitle

    % Problems start page 1107

    \pagebreak
    \section{Problem 1}
        The distance between the first and fifth minima of a single-slit diffraction pattern is 0.35 mm with the screen 40 cm away from the slit, when light of wavelength 550 nm is used. 
        (a) Find the slit width. 
        (b) Calculate the angle $\theta$ of the first diffraction minimum.

        \subsection{Solution (a)}
            Estimate $\sin \theta \approx \theta \approx \tan \theta = \frac{y}{D}$.
            $D = 40\,\unit{\centi\meter}$ and never changes in this case.
            $y$ does change, so we can create a $\Delta y$.
            \begin{equation}
                a \sin \theta = a \frac{y}{D} = m\lambda
            \end{equation}

            We can create a $\Delta y$ on one side and $\Delta m$ on the other, for a separation between the first and fifth fringes.
            \begin{gather}
                a \frac{y}{D} = m\lambda\\
                a \frac{\Delta y}{D} = \Delta m \lambda\\
                \begin{align}
                    a   &=  \frac{\Delta m}{\Delta y} \lambda D
                        =   \frac{4}{0.35\,\unit{\milli\meter}} * 550\,\unit{\nano\meter} * 0.4\,\unit{\meter}
                        =   \boxed{2.51\,\unit{\milli\meter}}
                \end{align}
            \end{gather}

        \subsection{Solution (b)}
            Divide the distance between fringes by four.
            \begin{equation}
                \Delta y = \frac{0.35\,\unit{\milli\meter}}{4} = 87.5\E{-6}\,\unit{\meter}
            \end{equation}
            
            Divide this by the distance to the screen to find the approximate angle.
            \begin{equation}
                \theta = \frac{87.5\E{-6}\,\unit{\meter}}{0.4\,\unit{\meter}} 
                    =   \boxed{2.2\E{-4}}
            \end{equation}

    \pagebreak
    \section{Problem 5}

        \subsection{Solution}

    \pagebreak
    \section{Problem 9}

        \subsection{Solution}

    \pagebreak
    \section{Problem 13}

        \subsection{Solution}

    \pagebreak
    \section{Problem 15}

        \subsection{Solution}

    \pagebreak
    \section{Problem 19}

        \subsection{Solution}

    \pagebreak
    \section{Problem 21}

        \subsection{Solution}

    \pagebreak
    \section{Problem 37}

        \subsection{Solution}

    \pagebreak
    \section{Problem 39}

        \subsection{Solution}

    \pagebreak
    \section{Problem 45}

        \subsection{Solution}

    \pagebreak
    \section{Problem 47}

        \subsection{Solution}

    \pagebreak
    \section{Problem 49}

        \subsection{Solution}

    \pagebreak
    \section{Problem 59}

        \subsection{Solution}

    \pagebreak
    \section{Problem 63}

        \subsection{Solution}

    \pagebreak
    \section{Problem 65}

        \subsection{Solution}

    \pagebreak
    \section{Problem 69}

        \subsection{Solution}

    \pagebreak
    \section{Problem 75}

        \subsection{Solution}

    \pagebreak
    \section{Problem 77}

        \subsection{Solution}

    \pagebreak
    \section{Problem 93}

        \subsection{Solution}

    \pagebreak
    \tableofcontents
\end{document}