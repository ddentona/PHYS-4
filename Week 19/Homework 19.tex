\documentclass[12pt]{article}
\usepackage{amsmath}
\usepackage{array}
\usepackage[thinc]{esdiff}
% \usepackage{gensymb}
\usepackage{geometry}
\usepackage{graphicx}
\usepackage{pgfplots}
\usepackage{siunitx}
\usepackage{wrapfig}

\title{Homework \#3, 4B}
\author{Donald Aingworth IV}
\date{February 5, 2025}

\pgfplotsset{width=8cm,compat=1.9}
\usepgfplotslibrary{external}
% \tikzexternalize

\begin{document}

\DeclareSIUnit{\mile}{mi}
\DeclareSIUnit{\gal}{gal}
\DeclareSIUnit{\foot}{ft}
\DeclareSIUnit{\hour}{h}
\DeclareSIUnit{\rad}{rad}
\DeclareSIUnit{\unit}{u}
\DeclareSIUnit{\dyne}{dyn}

\maketitle

\pagebreak
\section{Question 1}
Figure 22-22 shows three arrangements of electric field lines. In each arrangement, a proton is released from rest at point A and is then accelerated through point B by the electric field. Points A and B have equal separations in the three arrangements. Rank the arrangements according to the linear momentum of the proton at point B, greatest first.\\
\includegraphics[width=\textwidth]{picture_1.png}

\subsection*{Solution}
Case (a) has the particle in a constant high-magnitude electric field.\\
Case (b) has the particle in an electric field whose magnitude starts high but decreases, resulting in lower linear momentum from lower total force applied. $(a) > (b)$.\\
Case (c) has the particle in a constant-magnitude electric field with a lower electric field to (a) and equivalent to the end of (b). \boxed{(a) > (b) > (c)}. 

\pagebreak
\section{Question 6}
\begin{wrapfigure}{r}{0.5\textwidth}
    \vspace{-30pt}
    \includegraphics[width=0.5\textwidth]{picture_2.png} 
    % \label{fig:wrapfig}
\end{wrapfigure}
In Fig. 22-27, two identical circular nonconducting rings are centered on the same line with their planes perpendicular to the line. Each ring has charge that is uniformly distributed along its circumference. The rings each produce electric fields at points along the line. For three situations, the charges on rings $A$ and $B$ are, respectively, (1) $q_0$ and $q_0$, (2) $-q_0$ and $-q_0$, and (3) $-q_0$ and $q_0$. Rank the situations according to the magnitude of the net electric field at (a) point $P_1$ midway between the rings, (b) point $P_2$ at the center of ring B, and (c) point $P_3$ to the right of ring B, greatest first.

\subsection{(a) $P_1$}
\boxed{(3) > (2) = (1)}\\
(1): A pushes right, B pushes left. Net of 0.\\
(2): A pulls left, B pulls right. Net of 0.\\
(3): A pulls left, B pushes left. Net of strong left.

\subsection{(b) $P_2$}
\boxed{(3) = (2) = (1)}\\
(1): A pushes right, B pushes not. Net of right.\\
(2): A pulls left, B pushes not. Net of left, same magnitude.\\
(3): A pulls left, B pushes not. Net of left, same magnitude.

\subsection{(c) $P_3$}
\boxed{(1) = (2) > (3)}\\
(1): A pushes right, B pushes right. Net of strong right.\\
(2): A pulls left, B pulls left. Net of strong left.\\
(3): A pulls weak left, B pushes right. Net of weak right.

\pagebreak
\section{Question 9}
\begin{wrapfigure}{r}{0.5\textwidth}
    \vspace{-30pt}
    \includegraphics[width=0.5\textwidth]{picture_3.png} 
    % \label{fig:wrapfig}
\end{wrapfigure}
Figure 22-28 shows two disks and a flat ring, each with the same uniform charge Q. Rank the objects according to the magnitude of the electric field they create at points $P$ (which are at the same vertical heights), greatest first.

\subsection*{Solution}
(a) The charge is distributed about a small radius.\\
(b) The charge is distributed about a large radius, giving a weaker electric field. $(a) > (b)$\\
(c) The charge is distributed about a large radius but not a small inner radius, giving an even weaker electric fielddue to greater distance. \boxed{(a) > (b) > (c)}

\pagebreak
\section{Question 13}
\begin{wrapfigure}{r}{0.5\textwidth}
    \vspace{-30pt}
    \includegraphics[width=0.5\textwidth]{picture_4.png} 
    % \label{fig:wrapfig}
\end{wrapfigure}
Figure 22-32 shows three rods, each with the same charge Q spread uniformly along its length. Rods $a$ (of length $L$) and $b$ (of length $\frac{L}{2}$) are straight, and points $P$ are aligned with their midpoints. Rod c (of length $\frac{L}{2}$) forms a complete circle about point $P$. Rank the rods according to the magnitude of the electric field they create at points $P$, greatest first.

\subsection*{Solution}
(a) The charge is distributed below the point along a large distance, giving a small electric field.\\
(b) The same charge is distributed below the point along a smaller distance, giving a more condensed and consequently a larger electric field. $(b) > (a)$\\
(c) The same charge is applied from all directions. It consequently cancels out in every direction. \boxed{(b) > (a) > (c)}


\pagebreak
\section{Problem 24}
\begin{wrapfigure}{r}{0.35\textwidth}
    \vspace{-30pt}
    \includegraphics[width=0.35\textwidth]{picture_5.png} 
    % \label{fig:wrapfig}
\end{wrapfigure}
A thin nonconducting rod with a uniform distribution of positive charge Q is bent into a complete circle of radius R (Fig. 22-48). The central perpendicular axis through the ring is a z axis, with the origin at the center of the ring. What is the magnitude of the electric field due to the rod at (a) z = 0 and (b) z = $\infty$? (c) In terms of R, at what positive value of z is that magnitude maximum? (d) If R = 2.00 cm and Q = 4.00 \unit{\micro\coulomb}, what is the maximum magnitude?

\subsection*{Solution}
\subsubsection*{(a) z = 0}
In the chapter, we established a formula for the electric field from a ring.
\begin{equation*}
    E_{ring}(z) =   \frac{kqz}{(z^2 + R^2)^{3/2}}
\end{equation*}

As such, we can set a value of z.
\begin{equation*}
    E_{ring}(0) =   \frac{kq*0}{(0^2 + R^2)^{3/2}}
        =   \boxed{0}
\end{equation*}

\subsubsection*{(b) z = $\infty$}
We have the equation, established in part (a). Both top and bottom have $\lim_{z\rightarrow\infty}$ equal to $\infty$, so we can use l'Hopital's rule to find the limit.
\begin{align*}
    E_{ring}(z) &=  \frac{\diff{}{z}(kqz)}{\diff{}{z}\left((z^2 + R^2)^{3/2}\right)}
        =   \frac{kq}{2z * \frac{3}{2}(z^2 + R^2)^{1/2}}\\
    E_{ring}(\infty)    &=  \lim_{z\rightarrow\infty} \frac{kq}{3z (z^2 + R^2)^{1/2}}
        =   \boxed{0}
\end{align*}

\pagebreak
\subsubsection*{(c) z for maximum magnitude in terms of R}
We can find this by taking the derivatve of the magnitude of the electric field, then find the values for which the derivative is 0.
\begin{align*}
    \diff{E_{ring}(z)}{z}   &=  \diff{}{z}\frac{kqz}{(z^2 + R^2)^{3/2}}
        =   \frac{kq(z^2 + R^2)^{3/2} - 3kqz^2 (z^2 + R^2)^{1/2}}{(z^2 + R^2)^{3}}\\
        &=  \frac{kq(z^2 + R^2) - 3kqz^2}{(z^2 + R^2)^{5/2}}
        =   \frac{kqR^2 - 2kqz^2}{(z^2 + R^2)^{5/2}}
        =   \frac{kq(R^2 - 2z^2)}{(z^2 + R^2)^{5/2}}
\end{align*}

If it is 0, then \(R^2 - 2z^2 = 0\).
\begin{gather*}
    R^2 - 2z^2 = 0\\
    R^2 =   2z^2\\
    z^2 =   \frac{R^2}{2}\\
    \boxed{z   =   \frac{R}{\sqrt{2}}}
\end{gather*}

\subsubsection*{(d) Maximum value at R = 2.00 cm and Q = 4.00 C}
We know from the previous problem that the maximum value of $E$ comes from $z = \frac{R}{\sqrt{2}}$. We can plug this in with the other known values for the maximum value of the electric field.
\begin{align*}
    E_{ring}(z) &=  \frac{(8.99 \times 10^9)*(4 \times 10^{-6})*\frac{2 \times 10^{-2}}{\sqrt{2}}}{((\frac{2 \times 10^{-2}}{\sqrt{2}})^2 + (2 \times 10^{-2})^2)^{3/2}}
        =   \frac{(8.99 \times 10^9)*(4\sqrt{2} \times 10^{-8})}{((2 \times 10^{-4} + 4 \times 10^{-4})^{3/2}}\\
        &=  \frac{35.96\sqrt{2} \times 10^{1}}{(6 \times 10^{-4})^{3/2}}
        =   \frac{35.96\sqrt{2} \times 10^1}{6\sqrt{6} \times 10^{-6}}
        =   5.99\bar{3}\frac{\sqrt{3}}{3} \times 10^{7}\\
        &=  1.99\bar{7} \sqrt{3} \times 10^{7} \unit{\newton/\coulomb}
        =   \boxed{3.460 \times 10^{7} \unit{\newton/\coulomb}}
\end{align*}

\pagebreak
\section{Problem 26}
\begin{wrapfigure}{r}{0.2\textwidth}
    \vspace{-30pt}
    \includegraphics[width=0.2\textwidth]{picture_6.png} 
    % \label{fig:wrapfig}
\end{wrapfigure}
Fig. 22-50, a thin glass rod forms a semicircle of radius $r = 5.00 \unit{\centi\meter}$. Charge is uniformly distributed along the rod, with $+q = 4.50 \unit{\pico\coulomb}$ in the upper half and $-q = -4.50 \unit{\pico\coulomb}$ in the lower half. What are the (a) magnitude and (b) direction (relative to the positive direction of the x axis) of the electric field E at P, the center of the semicircle?

\subsection*{Solution}
\subsubsection*{(a) Magnitude}
We begin with our basic equation for the electric field, which we can then separate into the two halfs of the half-circle.
\begin{equation*}
    \vec{E}_{net}   =   \int_{\frac{\pi}{2}}^{\frac{3\pi}{2}} d\vec{E}
        =   \int_{\frac{\pi}{2}}^{\pi} d\vec{E} + \int_{\pi}^{\frac{3\pi}{2}} d\vec{E}
\end{equation*}

We then have equations for the different $d\vec{E}$.
\[ \vec{E} = \frac{kq}{r^2}\hat{r} \rightarrow d\vec{E} = \frac{kq}{r^2}\begin{pmatrix} -\cos(\theta)\\ -\sin(\theta) \end{pmatrix} d\theta \]
\begin{equation*}
    \vec{E}_{net}   =   \int_{\frac{\pi}{2}}^{\pi} \frac{kq}{r^2}\begin{pmatrix} -\cos(\theta)\\ -\sin(\theta) \end{pmatrix} d\theta + \int_{\pi}^{\frac{3\pi}{2}} \frac{k(-q)}{r^2}\begin{pmatrix} -\cos(\theta)\\ -\sin(\theta) \end{pmatrix} d\theta
\end{equation*}

We can now finish by integrating.
\begin{align*}
    \vec{E}_{net}   &=  \frac{kq}{r^2} \int_{\frac{\pi}{2}}^{\pi} \begin{pmatrix} -\cos(\theta)\\ -\sin(\theta) \end{pmatrix} d\theta + \frac{kq}{r^2} \int_{\pi}^{\frac{3\pi}{2}} \begin{pmatrix} \cos(\theta)\\ \sin(\theta) \end{pmatrix} d\theta\\
        &=  \frac{kq}{r^2}\left[ \begin{pmatrix} -\sin(\theta)\\ \cos(\theta) \end{pmatrix}_{\frac{\pi}{2}}^{\pi} + \begin{pmatrix} \sin(\theta)\\ -\cos(\theta) \end{pmatrix}_{\pi}^{\frac{3\pi}{2}} \right]\\
        &=  \frac{kq}{r^2}\left[ \begin{pmatrix} 1\\ -1 \end{pmatrix} + \begin{pmatrix} -1 \\ -1 \end{pmatrix} \right]
        =   \frac{kq}{r^2}\begin{pmatrix} 0\\-2 \end{pmatrix}
\end{align*}

Lastly, we can simply plug in values.
\begin{align*}
    \vec{E}_{net}   &=  \frac{kq}{r^2}\begin{pmatrix} 0\\-2 \end{pmatrix}
        =   \frac{(8.99 \times 10^{9})*(4.50 \times 10^{-12})}{(5 \times 10^{-2})^2}\begin{pmatrix} 0\\-2 \end{pmatrix}\\
        &=  \frac{40.455 \times 10^{-3}}{25 \times 10^{-4}}\begin{pmatrix} 0\\-2 \end{pmatrix}
        =   16.182 * \begin{pmatrix} 0\\-2 \end{pmatrix} = \begin{pmatrix} 0\\-32.364 \end{pmatrix}
\end{align*}

This means that the magnitude is \boxed{32.364 \unit{\newton/\coulomb}}.

\subsubsection*{Direction}
Relative to the positive direction of the x-axis, the direction is \boxed{\frac{3\pi}{2}}

\end{document}