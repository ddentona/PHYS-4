\documentclass[12pt]{article}
\usepackage{amsmath}
\usepackage{amssymb}
\usepackage{cancel}
\usepackage{enumitem}
\usepackage{esdiff}
\usepackage{graphicx}
\usepackage{siunitx}
% \usepackage{pgfplots}
\usepackage{wrapfig}

\newcommand{\e}[1]{e^{i(#1)}}
\newcommand{\E}[1]{\times 10^{#1}}

\title{
    Worksheet \#11
    \\  \small
    PHYS 4C: Waves and Thermodynamics
    }
\author{Donald Aingworth IV}
\date{November 3, 2025}

\begin{document}
    \DeclareSIUnit{\celsiusdegree}{C^\circ}
    \DeclareSIUnit{\atm}{ atm}

    \maketitle

    \setcounter{section}{0}
    \section{Problem 1}
        Two sound sources are placed at (0, +1.25 m, 0) and (0, -1.25 m, 0). 
        They both emit isotropic sound waves with a wavelength of 1.00 m at the same total power and in phase with one another. 
        Let I represent the intensity of the sound wave emitted by either source by itself at 100 m distance.

        \subsection{Part (a)}
            As a multiple of I, determine the intensity of the combined sound waves at $(100\,\unit{\meter}, 0, 0)$, $(0, 100\,\unit{\meter}, 0)$ and $(100/\sqrt{2}\,\unit{\meter}, 100/\sqrt{2}\,\unit{\meter}, 0)$. 
            For this part, you may assume that the amplitude of each individual wave is equal to what it would be at 100 m distance.

            \subsubsection{Solution}
                Suppose we call the point at (0, +1.25 m, 0) $r_1$ and the point at (0, -1.25 m, 0) $r_2$.
                First, we do $(100\,\unit{\meter}, 0, 0)$.
                Note the distances between $(100\,\unit{\meter}, 0, 0)$ and both $r_1$ (as $L_1$) and $r_2$ (as $L_2$).
                \begin{gather}
                    L_1 = \sqrt{100^2 + 1.25^2}\\
                    L_2 = \sqrt{100^2 + 1.25^2}\\
                    L_1 = L_2
                \end{gather}

                The two waves travel an identical distance, so they will be in phase. 
                This means that the amplitude will be the amplitude of the other two waves added together (in this case doubling).
                Since $I \varpropto A^2$, the intensity will in turn quadruple, leaving us with a final intensity of $4I$.

                Second, we do $(0, 100\,\unit{\meter}, 0)$. 
                We can calculate $\Delta L$ directly.
                \begin{equation}
                    \Delta L = 1.25\,\unit{\meter} + 1.25\,\unit{\meter} = 2.50\,\unit{\meter}  
                \end{equation}

        \subsection{Part (b)}
            If you were to walk along a 90\unit{\degree} arc in the x-y plane between (100 m, 0, 0) and (0, 100 m, 0), how many interference maxima would you encounter?
            Interference minima?

        \subsection{Part (c)}
            Now take into account the fact that the sound sources are not exactly 100 m away from the point (0, 100 m, 0) and the impact that has on the amplitude of the individual sound waves, and calculate the intensity of the combined wave at that point.

    \section{Problem 2}
        Two opera singers (standing still) are singing a high A (f = 880 Hz) and low A (f = 220 Hz), respectively, when a car carrying a sound-reflecting wall is travelling towards them at 0.100 times the speed of sound. 
        Determine the frequencies of the reflected sound waves, as would be heard by the opera singers.
        
        Note: No opera singers were harmed during the construction of this question or its solution.

        \subsection{Solution}


\end{document}