\documentclass[12pt]{article}
\usepackage{amsmath}
\usepackage{amssymb}
\usepackage{amsthm}
\usepackage{cancel}
\usepackage{enumitem}
\usepackage{esdiff}
\usepackage{graphicx}
\usepackage{siunitx}
% \usepackage{pgfplots}
\usepackage{wrapfig}

\newcommand{\e}[1]{e^{i\left(#1\right)}}
\newcommand{\E}[1]{\times 10^{#1}}

\renewcommand\qedsymbol{TENA}

\title{
    Homework \#2
    \\  \small
    PHYS 4D: Modern Physics
    }
\author{Donald Aingworth IV}
\date{January 26, 2026}

\begin{document}
    \maketitle

    \section{Question 2}
        Which physical effect or experiment shows that light has a wave nature?
        
        \subsection{Solution}
            One experiment that demonstrates the wave nature of light is the double slit experiment. 
            It results in multiple points of constructive and destructive interference, which occur for waves and not particles.

    \section{Question 3}
        Express the kinetic energy \textit{KE} of a particle in terms of its momentum \textbf{p}.

        \subsection{Solution}
            There are two answers to this. 
            The first assumes we express it in terms of both momentum and mass.
            \begin{equation}
                KE = \boxed{\frac{\vec{p}\cdot\vec{p}}{2m}}
            \end{equation}

            The second requires we assume momentum has a velocity component to it and uses an antiderivative or an integral.
            \begin{equation}
                KE = \diff[-1]{p}{v} = \int p(v)\,dv
            \end{equation}

    \section{Question 4}
        What would be the wavelength of a wave described by the function $u(x,t) = A \sin(2x/\unit{\centi\meter} - 10\unit{\second})$?

        \subsection{Solution}
            There is a formula relating $\lambda$ and $k$.
            \begin{gather}
                k = \frac{2\pi}{\lambda}\\
                \begin{align}
                    \lambda &=  \frac{2\pi}{k}
                        =   \frac{2\pi}{2\,\unit{\centi\meter^{-1}}}
                        =   \boxed{\pi\E{2}\,\unit{\meter}}
                \end{align}
            \end{gather}

    \section{Question 5}
        What would be the frequency of a wave described by the function $u(x,t) = A \sin(2x/\unit{\centi\meter} - 10t/\unit{\second})$?
        
        \subsection{Solution}
            \begin{gather}
                \omega = 2\pi f\\
                f = \frac{\omega}{2\pi} = \frac{10}{2\pi} = \boxed{\frac{5}{\pi}\,\unit{\hertz}}
            \end{gather}

    \section{Question 6}
        What would be the phase velocity of a wave described by the function $u(x,t) = A \sin(2x/\unit{\centi\meter} - 10t/\unit{\second})$?

        \subsection{Solution}
            The phase velocity is just the speed of the wave in this case.
            \begin{equation}
                v = \frac{\omega}{k} = \frac{10\,\unit{\second^{-1}}}{2\E{-2}\,\unit{\meter^{-1}}} = \boxed{500\,\unit{\meter/\second}}
            \end{equation}

    \section{Question 7}
        Write down the exponential function corresponding to a traveling wave with a wavelength of 10 cm and a frequency of 10 Hz.
        
        \subsection{Solution}
            The exponential function would have a format involving $k$ and $\omega$.
            \begin{equation}
                \psi(x,t) = \psi_0 \e{kx - \omega t} = \psi_0 \e{\frac{2\pi}{\lambda}x - 2\pi f\,t}
            \end{equation}

            We have values of $f$ and $\lambda$, whch we can substitute in.
            \begin{gather}
                \psi(x,t) = \psi_0 \e{\frac{2\pi}{0.1\,\unit{\meter}}x - 2\pi * 10\,\unit{\hertz} * t}\\
                \boxed{\psi(x,t) = \psi_0 \e{(20\pi\,\unit{\meter})\,x - (20\pi\,\unit{\hertz})\,t}}
            \end{gather}

    \section{Question 8}
        Write down a trigonometric function describing a stationary wave with a wavelength of 10 cm.

        \subsection{Solution}
            I'll be using sine.
            \begin{equation}
                \boxed{\psi(x,t) = \psi_0 \sin\left( (20\pi\,\unit{\meter})\,x \right)}
            \end{equation}

    \section{Question 11}
        Which forms of electromagnetic radiation have a wavelength shorter than visible light?

        \subsection{Solution}
            Looking at the textbook, those with shorter wavelength than visible light are \underbar{ultraviolet, x-rays, and gamma rays}.

    \section{Question 12}
        Calculate the frequency of electromagnetic radiation having a wavelength of 10 nm.

        \subsection{Solution}
            Frequency is inversely proportional to wavelength.
            \begin{gather}
                c = \lambda f\\
                \begin{align}
                    f   &=  \frac{c}{\lambda}
                        =   \frac{2.998\E{8}\,\unit{\meter/\second}}{10\E{-9}\,\unit{\meter}}
                        =   \boxed{2.998\E{16}\,\unit{\hertz}}
                \end{align}
            \end{gather}

    \section{Question 13}
        Write down a formula expressing the energy of a photon in terms of the frequency of light.

        \subsection{Solution}
            All credit goes to a German guy named Albert.
            $f$ refers to Planck's constant.
            \begin{equation}
                E = hf
            \end{equation}

    \section{Question 14}
        Write down a formula expressing the energy of a photon in terms of the wavelength of light.

        \subsection{Solution}
            We have two equations known: one for the energy in terms of frequency, and the other relating frequency and wavelength.
            \begin{gather}
                E = hf\\
                c = \lambda f \to f = \frac{c}{\lambda}
            \end{gather}

            We can substitute the latter into the former.
            \begin{equation}
                \boxed{E = \frac{hc}{\lambda}}
            \end{equation}

    \section{Question 15}
        What is the energy of the photons for light with a wavelength of 0.1 nm?
        
        \subsection{Solution}
            Use the equation from Question 14.
            \begin{align}
                E   &=  \frac{hc}{\lambda}
                    =   \frac{6.62607015\E{-34} \cdot 2.998\E{8}}{10^{-10}}
                    =   \boxed{1.986\E{-15}\,\unit{\joule}}
            \end{align}

    \section{Problem 1}
        Calculate the frequency of light having a wavelength $\lambda = 500\,\unit{\nano\meter}$.

        \subsection{Solution}
            Use the relationship between frequency and wavelength.
            \begin{gather}
                c = \lambda f\\
                f   =   \frac{c}{\lambda}
                    =   \frac{2.998\E{8}}{500\E{-9}}
                    =   \boxed{5.996\E{14}\,\unit{\hertz}}
            \end{gather}

    \section{Problem 2}
        Calculate the energy of the photons for light having a wavelength $\lambda = 500\,\unit{\nano\meter}$.

        \subsection{Solution}
            Use the relationhip between wavelength and energy.
            \begin{align}
                E   &=  \frac{hc}{\lambda}
                    =   \frac{6.626\E{-34} * 2.998\E{8}}{500\E{-9}}
                    =   \boxed{3.973\E{-19}\,\unit{\joule}}
            \end{align}

    \section{Problem 3}
        Suppose that a beam of light consists of photons having an energy of 5.4 eV. 
        What is the wavelength of the light?

        \subsection{Solution}
            I will be using a new value of $hc$.
            \begin{align}
                \lambda &=  \frac{hc}{E}
                    =   \frac{1240\,\unit{\eV\,\nano\meter}}{5.4\,\unit{\eV}}
                    =   \boxed{229.6\,\unit{\nano\meter}}
            \end{align}

    \section{Problem 10}
        Using Euler's identity, Eq. (I.24) show that adding two waves given by $y_1 = A\e{kx-\omega t}$ and $y_2 = A\e{kx+\omega t}$ gives a new wave with time-dependent amplitude and position dependence $\e{kx}$.
        \begin{equation} \label{I.24}\tag{I.24}
            e^{i\theta} = \cos\theta + i\sin\theta
        \end{equation}

        \subsection{Solution}
        \begin{proof}
            Time to play the game!
            Take our two equations.
            \begin{gather}
                y_1 = A\e{kx - \omega t}\\
                y_2 = A\e{kx + \omega t}
            \end{gather}

            Add these two together in superposition.
            \begin{gather}
                y_{\Sigma} = y_1 + y_2 = A\e{kx - \omega t} + A\e{kx + \omega t}
            \end{gather}

            To begin our manipulation, we can separate values of exponentials.
            \begin{gather}
                y_{\Sigma} = A\e{kx}\e{-\omega t} + A\e{kx}\e{\omega t}
            \end{gather}

            Here, we combine like terms.
            \begin{gather}
                y_{\Sigma} = A\e{kx}\left( \e{-\omega t} + \e{\omega t} \right)
            \end{gather}

            Using \ref{I.24}, we can convert the values inside the parentheses to trigonometric values.
            We can then use trig identities and positive/negative functions to cancel out values.
            \begin{align}
                y_{\Sigma}  &=  A\e{kx}\left( \cos(-\omega t) + \sin(-\omega t) + \cos(\omega t) + \sin(\omega t) \right)\\
                    &=  A\e{kx}\left( \cos(\omega t) - \sin(\omega t) + \cos(\omega t) + \sin(\omega t) \right)\\
                    &=  A\e{kx}\left( 2\cos(\omega t) \right)
                    =   \boxed{2A\cos(\omega t)\e{kx}}
            \end{align}

            It's not pretty, but it works. 
            This is indeed a wave with time-dependent amplitude with poposition-dependence $\e{kx}$.
        \end{proof}

    \section{Problem 11}
        A very sensitive detector measures the energy of a single photon from starlight at 2.5 eV and at the same time measures its wavelength at 495 nm. 
        What is the value of Planck's constant at that far-away star?

        \subsection{Solution}
            Assume the speed of light to be constant everywhere.
            Radical, I know.
            Use this in the equation for the energy from wavelength.
            \begin{equation}
                E = \frac{hc}{\lambda}
            \end{equation} 

            Reorder this equation to isolate $h$.
            \begin{align}
                h   &=  \frac{E\lambda}{c}
                    =   \frac{2.5\,\unit{\eV} \cdot 495\,\unit{\nano\meter}}{2.998\E{17}\,\unit{\nano\meter/\second}}\\
                    &=  \boxed{4.128\E{-15}\,\unit{\eV\cdot\second}}
            \end{align}

            Doing the math with this, it is remarkably close to what the accepted value of Planck's constant is in our galaxy, off by $0.008\E{-15}\,\unit{\eV\cdot\second}$.
            Maybe our detector is just not precise enough for measuring Planck's constant from 500 light years away. 
\end{document}