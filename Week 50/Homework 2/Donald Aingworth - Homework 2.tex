\documentclass[12pt]{article}
\usepackage{amsmath}
\usepackage{amssymb}
\usepackage{cancel}
\usepackage{enumitem}
\usepackage{esdiff}
\usepackage{graphicx}
\usepackage{siunitx}
% \usepackage{pgfplots}
\usepackage{wrapfig}

\newcommand{\e}[1]{e^{i(#1)}}
\newcommand{\E}[1]{\times 10^{#1}}

\title{
    Homework \#2
    \\  \small
    PHYS 4D: Modern Physics
    }
\author{Donald Aingworth IV}
\date{January 26, 2026}

\begin{document}
    \maketitle

    \section{Question 2}
        Which physical effect or experiment shows that light has a wave nature?
        
        \subsection{Solution}
            One experiment that demonstrates the wave nature of light is the double slit experiment. 
            It results in multiple points of constructive and destructive interference, which occur for waves and not particles.

    \section{Question 3}
        Express the kinetic energy \textit{KE} of a particle in terms of its momentum \textbf{p}.

        \subsection{Solution}
            There are two answers to this. 
            The first assumes we express it in terms of both momentum and mass.
            \begin{equation}
                KE = \boxed{\frac{\vec{p}\cdot\vec{p}}{2m}}
            \end{equation}

            The second requires we assume momentum has a velocity component to it and uses an antiderivative.
            \begin{equation}
                KE = \diff[-1]{p}{v}
            \end{equation}

    \section{Question 4}
        What would be the wavelength of a wave described by the function $u(x,t) = A \sin(2x/\unit{\centi\meter} - 10\unit{\second})$?

        \subsection{Solution}
            There is a formula relating $\lambda$ and $k$.
            \begin{gather}
                k = \frac{2\pi}{\lambda}\\
                \begin{align}
                    \lambda &=  \frac{2\pi}{k}
                        =   \frac{2\pi}{2\,\unit{\centi\meter^{-1}}}
                        =   \boxed{\pi\E{2}\,\unit{\meter}}
                \end{align}
            \end{gather}

    \section{Question 5}
        What would be the frequency of a wave described by the function $u(x,t) = A \sin(2x/\unit{\centi\meter} - 10t/\unit{\second})$?
        
        \subsection{Solution}
            \begin{gather}
                \omega = 2\pi f\\
                f = \frac{\omega}{2\pi} = \frac{10}{2\pi} = \boxed{\frac{5}{\pi}\,\unit{\hertz}}
            \end{gather}

    \section{Question 6}
        What would be the phase velocity of a wave described by the function $u(x,t) = A \sin(2x/\unit{\centi\meter} - 10t/\unit{\second})$?

        \subsection{Solution}
            The phase velocity is just the speed of the wave in this case.
            \begin{equation}
                v = \frac{\omega}{k} = \frac{10\,\unit{\second^{-1}}}{2\E{-2}\,\unit{\meter^{-1}}} = \boxed{500\,\unit{\meter/\second}}
            \end{equation}

    \section{Question 7}
        Write down the exponential function corresponding to a traveling wave with a wavelength of 10 cm and a frequency of 10 Hz.
        
        \subsection{Solution}
            The exponential function would have a format involving $k$ and $\omega$.
            \begin{equation}
                \psi(x,t) = \psi_0 \e{kx - \omega t} = \psi_0 \e{\frac{2\pi}{\lambda}x - 2\pi f\,t}
            \end{equation}

    \section{Question 8}
        Write down a trigonometric function describing a stationary wave with a wavelength of 10 cm.

    \section{Question 11}
        Which forms of electromagnetic radiation have a wavelength shorter than visible light?

    \section{Question 12}
        Calculate the frequency of electromagnetic radiation having a wavelength of 10 nm.

    \section{Question 13}
        Write down a formula expressing the energy of a photon in terms of the frequency of light.

    \section{Question 14}
        Write down a formula expressing the energy of a photon in terms of the wavelength of light.

    \section{Question 15}
        What is the energy of the photons for light with a wavelength of 0.1 nm?

    \section{Problem 1}
        Calculate the frequency of light having a wavelength $\lambda = 500\,\unit{\nano\meter}$.

    \section{Problem 2}
        Calculate the energy of the photons for light having a wavelength $\lambda = 500\,\unit{\nano\meter}$.

    \section{Problem 3}
        Suppose that a beam of light consists of photons having an energy of 5.4 eV. 
        What is the wavelength of the light?

    \section{Problem 10}
        Using Euler's identity, Eq. (I.24) shows that adding two waves given by $y_1 = A\e{kx-\omega t}$ and $y_2 = A\e{kx+\omega t}$ gives a new wave with time-dependent amplitude and position dependence $\e{kx}$.

    \section{Problem 11}
        A very sensitive detector measures the energy of a single photon from starlight at 2.5 eV and at the same time measures its wavelength at 495 nm. 
        What is the value of Planck's constant at that far-away star?
\end{document}