\documentclass[12pt]{article}
\usepackage{amsmath}
\usepackage{amssymb}
\usepackage{cancel}
\usepackage{enumitem}
\usepackage{esdiff}
\usepackage{graphicx}
\usepackage{siunitx}
% \usepackage{pgfplots}
\usepackage{wrapfig}

\newcommand{\e}[1]{e^{i(#1)}}
\newcommand{\E}[1]{\times 10^{#1}}

\title{
    Homework \#1
    \\  \small
    PHYS 4D: Modern Physics
    }
\author{Donald Aingworth IV}
\date{January 26, 2026}

\begin{document}
    \maketitle

    \section{Exercise 1}
        The shape drawn is a parabola with its lowest point at $(0,0)$, as shown below.\\
        \includegraphics[width=\textwidth]{1-1.png}

        Scrolling (pinching on a tablet) zooms in and out.
        Adding $h=(x-4)^2$ adds a second blue parabola 4 units to the right and blue.\\
        \includegraphics[width=\textwidth]{1-2.png}

        Click the symbol next to the graph name to show or hide it.
        A variable not indicated will require a new variable (e.g. $s$) that can use a slider.
        Bounds can be set and value variation can be automated.\\
        \includegraphics[width=\textwidth]{1-3.png}

    \section{Exercise 2}
        The function $f = \sin(x + t)$ is a sine wave. 
        As $t$ increases, the function appears to go left, and vice versa.\\
        \includegraphics[width=\textwidth]{2-1.png}

        Switching to $f = 5\sin(x + t)$ increases the amplitude of the wave. 
        Changing the $+$ to a $-$ changes the direction the wave moves when animated.\\
        \includegraphics[width=\textwidth]{2-2.png}

    \section{Exercise 3}
        Superposition can be done by adding functions.\\
        \includegraphics[width=\textwidth]{3-1.png}

        Sine functions have a phase of $2\pi$, so they repeat every $2\pi$ radians.\\
        \includegraphics[width=\textwidth]{3-2.png}

        Constructive interference means waves work together for total value.
        Destructive interference means waves work against e/o.
\end{document}