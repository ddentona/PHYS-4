\documentclass[8pt]{minimal}
\usepackage{amsmath}
\usepackage{array}
\usepackage{siunitx}

\begin{document}
\setlength{\parindent}{0pt}
% \setlength{\voffset}{-50pt}
\setlength{\columnsep}{1cm}
\twocolumn
PHYS 4A Exam 4 Cheat Sheet (with \LaTeX)

\underline{Angular Kinematics}
\begin{gather*}
    \theta = \frac{S}{r}; \omega = \frac{d\theta}{dt}; \alpha = \frac{d^2\theta}{dt^2}; (1)\ \omega(t) = \omega_0 + \alpha t\\
    (2)\ \theta = \theta_0 + \omega_0t + \frac{1}{2}\alpha t^2; (3)\ \omega^2 = \omega_0^2 + 2\alpha\Delta\theta\\
    v_t = \omega r; a_t = \alpha r; a_c = \omega r^2; T = \frac{2 \pi}{\omega}
\end{gather*}

\underline{Electric Fields and Forces}
\begin{gather*}
    e = 1.602 \times 10^{-19} \unit{\coulomb}; \epsilon_0 = 8.85 \times 10^{-12} \unit{\frac{\coulomb^2}{\newton\meter^2}}\\
    k = 8.99 \times 10^9 \unit{\frac{\newton\meter^2}{\coulomb^2}}= \frac{1}{4\pi\epsilon_0}\\
    \vec{F} = \frac{kq_1q_2}{r^2}\hat{r} = \frac{kq_1q_2}{r^3}\vec{r};
    \vec{E} = \frac{kq}{r^2}\hat{r} = \frac{kq}{r^3}\vec{r}; F = qE
\end{gather*}

In a diagram, the direction of an electric field is represented by the direction of its arrows, while the strength of the field is represented by the proxmity of the lines.
\[\lambda = \frac{Q}{r} ; \sigma = \frac{Q}{A} ; \rho = \frac{Q}{V}\]
\[E = \int dE = \int \frac{k\ dq}{r^3}\vec{r} = \int \frac{k \lambda}{r^3}\vec{r}dr\]
\[ \vec{E}_{ring}(z) = \frac{kqz}{(z^2 + R^2)^{3/2}}\hat{k} \]
For a rod of length L, measured at a distance d from the close end from the rod of charge Q.
\[ \vec{E}_{rod;axis}(d) = -\frac{kQ}{d(d-L)}\hat{i} \]

For a rod of length L, measured perpendicular to the rod at a distance d from the close end from the rod of charge Q.
\[\vec{E} = k\lambda \left[ \frac{1}{z} - \frac{1}{L^2 + z^2} \right]\hat{i} + \frac{k\lambda L}{z\sqrt{L^2 + z^2}} \hat{j}\]
\begin{gather*}
    \vec{E}_{arc} = \frac{k\lambda}{r} \begin{pmatrix}2\sin(\frac{\theta}{2})\\0\end{pmatrix}\\
    \vec{E}_{disc} = \frac{\sigma}{2\epsilon_0} \left(1 - \frac{z}{\sqrt{z^2 + R^2}}\right)
\end{gather*}


\underline{Electric Dipoles}

\begin{align*}
    \vec{E} &= \left\{ \begin{matrix}
        < 0 \text{ if } -\frac{d}{2} < z < \frac{d}{2}\\
        > 0 \text{ otherwise}
    \end{matrix} \right.\\
        &=  \frac{2kQd}{z^3\left(1 - \frac{d^2}{4z^2}\right)^2}\hat{d}
\end{align*}

In an electric field:
\begin{gather*}
    \vec{p} = Q\vec{d}\\
    \vec{\tau} = \vec{p} \times \vec{E}\\
    U = -\vec{p} \cdot \vec{E}
\end{gather*}


\underline{Current}
\[ I = \frac{dq}{dt} \]

\end{document}