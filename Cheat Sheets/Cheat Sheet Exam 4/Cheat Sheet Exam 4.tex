\documentclass[8pt]{minimal}
\usepackage{amsmath}
\usepackage{array}
\usepackage{siunitx}

\begin{document}
\setlength{\parindent}{0pt}
% \setlength{\voffset}{-50pt}
\setlength{\columnsep}{1cm}
\twocolumn
PHYS 4A Exam 4 Cheat Sheet (with \LaTeX)
\[ e = 1.602 \times 10^{-19} \unit{\coulomb} \]
\[ k = 8.99 \times 10^9 \unit{\frac{\newton\meter^2}{\coulomb^2}}= \frac{1}{4\pi\epsilon_0} \]
\[ \epsilon_0 = 8.85 \times 10^{-12} \unit{\frac{\coulomb^2}{\newton\meter^2}} \]
\[\vec{F} = \frac{kq_1q_2}{r^2}\hat{r} = \frac{kq_1q_2}{r^3}\vec{r}\]
\[\vec{E} = \frac{kq}{r^2}\hat{r} = \frac{kq}{r^3}\vec{r}\]
\[F = qE\]
In a diagram, the direction of an electric field is represented by the direction of its arrows, while the strength of the field is represented by the proxmity of the lines.
\[\lambda = \frac{Q}{r} ; \rho = \frac{Q}{A} ; \sigma = \frac{Q}{V}\]
\[E = \int dE = \int \frac{k\ dq}{r^3}\vec{r} = \int \frac{k \lambda}{r^3}\vec{r}dr\]
\[ \vec{E}_{ring}(z) = \frac{kqz}{(z^2 + R^2)^{3/2}}\hat{k} \]
For a rod of length L, measured at a distance d from the close end from the rod of charge Q.
\[ \vec{E}_{rod;axis}(d) = -\frac{kQ}{d(d-L)}\hat{i} \]
\[ \vec{E}_{arc} = \frac{k\lambda}{r} \begin{pmatrix}2\sin(\frac{\theta}{2})\\0\end{pmatrix} \]

\underline{Electric Dipoles}

\[\vec{E} = \left\{ \begin{matrix}
    < 0 \text{ if } -\frac{d}{2} < z < \frac{d}{2}\\
    > 0 \text{ otherwise}
\end{matrix} \right. \]

\underline{Current}
\[ I = \frac{dq}{dt} \]

\end{document}