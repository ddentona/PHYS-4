\documentclass[8pt]{minimal}
\usepackage{amsmath}
\usepackage{array}
\usepackage{layout}
\usepackage{siunitx}

\setlength{\parindent}{-5pt}
\setlength{\voffset}{-50pt}
\setlength{\textheight}{724pt}
\setlength{\hoffset}{-25pt}
\setlength{\textwidth}{526pt}
\setlength{\columnsep}{1cm}

\begin{document}
\twocolumn
PHYS 4 Exam 5 Cheat Sheet (with \LaTeX)

\underline{Angular Kinematics}
\begin{gather*}
    \theta = \frac{S}{r}; \omega = \frac{d\theta}{dt}; \alpha = \frac{d^2\theta}{dt^2}; (1)\ \omega(t) = \omega_0 + \alpha t\\
    (2)\ \theta = \theta_0 + \omega_0t + \frac{1}{2}\alpha t^2; (3)\ \omega^2 = \omega_0^2 + 2\alpha\Delta\theta\\
    v_t = \omega r; a_t = \alpha r; a_c = \omega r^2; T = \frac{2 \pi}{\omega}
\end{gather*}

\underline{Electric Fields and Forces}
\begin{gather*}
    e = 1.602 \times 10^{-19} \unit{\coulomb}; \varepsilon_0 = 8.85 \times 10^{-12} \unit{\frac{\coulomb^2}{\newton\meter^2}}\\
    k = 8.99 \times 10^9 \unit{\frac{\newton\meter^2}{\coulomb^2}}= \frac{1}{4\pi\epsilon_0}\\
    \vec{F} = \frac{kq_1q_2}{r^2}\hat{r} = \frac{kq_1q_2}{r^3}\vec{r};
    \vec{E} = \frac{kq}{r^2}\hat{r} = \frac{kq}{r^3}\vec{r}; F = qE
\end{gather*}

In a diagram, the direction of an electric field is represented by the direction of its arrows, while the strength of the field is represented by the proxmity of the lines.
\[\lambda = \frac{Q}{r} ; \sigma = \frac{Q}{A} ; \rho = \frac{Q}{V}\]
\[E = \int dE = \int \frac{k\ dq}{r^3}\vec{r} = \int \frac{k \lambda}{r^3}\vec{r}dr\]
\[ \vec{E}_{ring}(z) = \frac{kqz}{(z^2 + R^2)^{3/2}}\hat{k} \]
For a rod of length L, measured at a distance d from the close end from the rod of charge Q.
\[ \vec{E}_{axis} = -\frac{kQ}{d(d-L)}\hat{i} \]

For a rod of length L, measured perpendicular to the rod at a distance d from the close end from the rod of charge Q.
\begin{gather*}
    \vec{E} = k\lambda \left[ \frac{1}{z} - \frac{1}{L^2 + z^2} \right]\hat{i} + \frac{k\lambda L}{z\sqrt{L^2 + z^2}} \hat{j}\\
    V = k\lambda\ln\left(\frac{L + \sqrt{L^2 + d^2}}{d}\right)
\end{gather*}
\begin{gather*}
    \vec{E}_{arc} = \frac{k\lambda}{r} \begin{pmatrix}2\sin(\frac{\theta}{2})\\0\end{pmatrix}\\
    \vec{E}_{disc} = \frac{\sigma}{2\epsilon_0} \left(1 - \frac{z}{\sqrt{z^2 + R^2}}\right)
\end{gather*}

For a spherical shell of radius R.
\begin{gather*}
    \vec{E} = \left\{ \begin{matrix}
        0 \text{ if } r < R\\
        \frac{R^2 \sigma}{\varepsilon_0 r^2}\hat{r} \text{ otherwise}
    \end{matrix} \right.
\end{gather*}
If $r < R$, $\Delta V = 0$. If $r \to \infty$, $V = 0$.

\underline{Gauss' Law}
\begin{gather*}
    \Phi = \frac{q_{enc}}{\varepsilon_0};
    \Phi = \oint \vec{E} \cdot d\vec{A}
\end{gather*}
$A$ must be a Gaussian surface. 
If $\vec{E}$ is constant on the surface, it can be simplified to $\Phi = E*A$.
Conductors in an electric field have $\vec{E} = 0$ inside. 
Electrons move to ensure this. 
Inside, $\Phi = 0$.


\underline{Electrical Potential Difference}\\
Path independent. For $\vec{E}(x,y,z)$:
\begin{gather*}
    \Delta V = \frac{\Delta U}{q} = -\int_{i}^{f} \vec{E}\cdot d\vec{x} = \int_{i}^{f}\,dV
\end{gather*}

Electric field lines go from more positive to more negative voltage.\\
Equipotential surface (ES): Surface with same $V$.\\
Conductors have equipotential volumes and $\vec{E} = 0$
\begin{gather*}
    V = \frac{kq}{r} = \int \frac{k\,dQ}{r};
    \vec{E} = -\nabla V
\end{gather*}


\underline{Capacitance (C)}\\
Relationship between charged separated and potential difference. $Q = C*\Delta V$
To find capacitance:\\
1. Draw a picture\\
2. Determine direction of $\vec{E}$\\
3. Determine $\vec{E}$ (Gauss' and determined distributions help), then $\Delta V = -\int \vec{E}\cdot d\vec{s}$\\
4. Calculate $C$ with $C = \frac{Q}{\Delta V}$

For parallel plates, $C = \frac{A \varepsilon_0}{d}$.\\
For cylindrical capacitor length $L$, $C = \frac{2\pi L\varepsilon_0}{\ln(b/a)}$.\\
Concentric spheres of radii $a$ and $b$, $C = 4\pi\varepsilon_0\frac{ab}{b - a}$.\\
Isolated sphere of radius $R$, $C = 4\pi\varepsilon_0 R$.

Since $W = q\Delta V$, $\Delta U = \frac{1}{2}C*\Delta V^2 = \frac{q^2}{2C}$ (Electric Potential Energy)\\
$u = \frac{1}{2}\varepsilon_0 E^2 = \frac{U}{Vol}$

A dielectric/material is in an electric field has a dielectric onstant $\kappa$.
In it, $\varepsilon_0$ is replaced with $\kappa\varepsilon_0$.
$\kappa$ of metals is considered $\infty$. $\kappa(vaccum) = 1$\\
If you put a dielectric in a capacitor, treat it like a network of capacitors in a creative alignment.

Add a dielectric to charged capacitor: \\\(Q_\kappa = Q_0; V_\kappa < V_0; C_\kappa > C_0; U_\kappa < U_0\)

Add a dielectric to battery-connected capacitor: \\\(V_\kappa = V_0; Q_\kappa > Q_0; C_\kappa > C_0; U_\kappa > U_0\)


\underline{Current}
\[ I = \frac{dq}{dt} \]

Ohm's Law: $V = IR$

Junction rule: For any point on a circuit, $I_{in} = I_{out}$

Stored charge at junction slows down $I_{in}$ \& speeds up $I_{out}$

\underline{Current Density}\\
For a cross-section $\vec{A}, dI = \vec{J} \cdot d\vec{A}$
\[\vec{J} = e*\vec{v}_d*n = \frac{\vec{E}}{\rho}\]


\underline{Circuits}\\
Batteries keep $\Delta V$ constant\\
Long end of battery diagram is + side\\
\begin{tabular}[width=\textwidth]{| c | c | c |}
                &Series                             &Parallel\\
    Capacitor   &$\frac{1}{C} = \sum \frac{1}{C_i}$ &$C = \sum C_i$\\
    Resistor    &$R = \sum R_i$                     &$\frac{1}{R} = \sum \frac{1}{R_i}$
\end{tabular}


\pagebreak
\underline{Electric Dipoles}

\begin{align*}
    \vec{E} &= \left\{ \begin{matrix}
        < 0 \text{ if } -\frac{d}{2} < z < \frac{d}{2}\\
        > 0 \text{ otherwise}
    \end{matrix} \right.\\
        &=  \frac{2kQd}{z^3\left(1 - \frac{d^2}{4z^2}\right)^2}\hat{d}
\end{align*}

ESs are $\perp$ to $\vec{p}$.
In an electric field:
\begin{gather*}
    \vec{p} = Q\vec{d}\\
    \vec{\tau} = \vec{p} \times \vec{E}\\
    U = -\vec{p} \cdot \vec{E}
\end{gather*}

\end{document}