\documentclass[12pt]{article}
\usepackage{amsmath}
\usepackage{amssymb}
\usepackage{graphicx}
% \usepackage{physics}
\usepackage{siunitx}
\usepackage{wrapfig}

% \AtBeginDocument{\RenewCommandCopy\qty\SI}

\newcommand{\E}[1]{\times 10^{#1}}

\title{
    Chapter 20 End-of-Chapter Problems
    \\ \small
    Halliday \& Resnick, 10th Edition
}

\author{Doanld Aingworth IV}

\date{\small Hit me where it Matters}

\begin{document}
    \DeclareSIUnit{\atm}{atm}
    \DeclareSIUnit{\cal}{\ cal}
    \DeclareSIUnit{\Cal}{\ Cal}
    \DeclareSIUnit{\calorie}{\ cal}
    \DeclareSIUnit{\Calorie}{\ Cal}
    \DeclareSIUnit{\celsiusdegree}{C^\circ}
    \DeclareSIUnit{\fahrenheit}{^\circ F}
    \DeclareSIUnit{\torr}{\ torr}

    \maketitle

    \pagebreak
    \section{Problem 1}
        Suppose 4.00 mol of an ideal gas undergoes a reversible isothermal expansion from volume $V_1$ to volume $V_2 = 2.00\,V_1$ at temperature $T = 400\,\unit{\kelvin}$. 
        Find (a) the work done by the gas and (b) the entropy change of the gas. 
        (c) If the expansion is reversible and adiabatic instead of isothermal, what is the entropy change of the gas?

        \subsection{Solution (a)}
            The work done by a system is defined as an integral.
            \begin{equation}
                W   =   \int_{i}^{f} p\,dV
            \end{equation}

            We do not know the value of the pressure, but the ideal gas law can be used here.
            \begin{equation}
                pV = nRT \to p = \frac{nRT}{V}
            \end{equation}

            This in turn can be put into the above equation.
            Bear in mind thet the number of moles and the temperture remain constant.
            \begin{align}
                W_{\rm in}  &=  -\int_{V_i}^{V_f} \frac{nRT}{V}\,dV
                    =   -nRT\,\int_{V_i}^{2V_i} \frac{1}{V}\,dV\\
                    &=  -nRT\,\left[ \ln(V) \right]_{V_i}^{2V_i}
                    =   -nRT\ln\left( \frac{2V_i}{V_i} \right)\\
                    &=  -nRT\ln\left( 2 \right)
                    =   -4*8.31*400*\ln(2)\\
                    &=  -13296*\ln(2)
                    =   \boxed{-9216.1\,\unit{\joule}}
            \end{align}

        \subsection{Solution (b)}
            The entropy change of a system is also defined by an equation.
            \begin{equation}
                \Delta S    =   \int_{i}^{f} \frac{1}{T}\,dQ
            \end{equation}

            This is an isothermal process, so the temperature remains constant. 
            Due to that, we can simplify our equation for the change in entropy.
            \begin{align}
                \Delta S    &=  \int_{i}^{f} \frac{1}{T}\,dQ
                    =   \frac{1}{T}\int_{i}^{f} \,dQ
                    =   \frac{Q}{T}
            \end{align}

            We know the work done on the gas.
            The change in internal energy is correlated with the change in temperature ($\Delta E_{\rm int} = nC_V \Delta T$).
            Since there is no change in temperature, there would be no change in internal energy ($\Delta E_{\rm int} = 0$).
            This can be used with the first law of thermodynamics to find the heat inserted.
            \begin{eqnarray}
                \Delta E_{\rm int}  =   Q_{\rm in} + W_{\rm in}\\
                0   =   Q_{\rm in}  -   9216.1\,\unit{\joule}\\
                Q_{\rm in}  =   9216.1\,\unit{\joule}
            \end{eqnarray}

            This can be used ot find the change in entropy.
            \begin{align}
                \Delta S    &=  \frac{Q}{T}
                    =   \frac{9216.1\,\unit{\joule}}{400\unit{\kelvin}}
                    =   \boxed{23.04\,\unit{\joule/\kelvin}}
            \end{align}

        \subsection{Solution (c)}
            If the process is adiabatic, there is no net heat inserted in at all.
            This means that $Q = 0$. 
            Since the temperature is also constant, the net formula would be solvable.
            \begin{align}
                \Delta S    &=  \frac{1}{T} \int_{i}^{f}\,dQ
                    =   \frac{Q_{\rm in}}{T}
                    =   \frac{0}{T}
                    =   \boxed{0\,\unit{\joule}}
            \end{align}

    \pagebreak
    \section{Problem 3}
        A 2.50 mol sample of an ideal gas expands reversibly and isothermally at 360 K until its volume is doubled. 
        What is the increase in entropy of the gas?

        \subsection{Solution}
            The process is isothermal, so $T$ is constant, $\Delta T = 0$, and $\Delta E_{\rm int} = 0$.
            We can use this to find the work done on the gas.
            Since we are looking to double the volume, the formula for the final volume would be $V_f = 2V_i$.
            We can also use the ideal gas law here.
            \begin{gather}
                pV  =   nRT \to p = \frac{nRT}{V}\\
                \begin{align}
                    W   &=  -\int_{V_i}^{V_f} p\,dV
                        =   -\int_{V_i}^{2V_i} \frac{nRT}{V}\,dV
                        =   -nRT\int_{V_i}^{2V_i} \frac{1}{V}\,dV\\
                        &=  -nRT\left[ \ln(V) \right]_{V_i}^{2V_i}
                        =   -nRT\ln\left( \frac{2V_i}{V_i} \right)
                        =   -nRT\ln(2)
                \end{align}
            \end{gather}

            The work inserted is directly related to heat inserted, itself directly related to enchange in entropy, so we can use that.
            Bear in mind this is an isothermal process.
            \begin{gather}
                \Delta E_{\rm int}  =   0
                    =   Q + W\\
                Q   =   -W
                    =   nRT\ln(2)\\
                \begin{align}
                    \Delta S    &=  \frac{Q}{T}
                        =   \frac{nRT\ln(2)}{T}
                        =   nR\ln(2)\\
                        &=  2.50 * 8.31 * \ln(2)
                        =   \boxed{14.4\,\unit{\joule}}
                \end{align}
            \end{gather}

    \pagebreak
    \section{Problem 5}
        Find (a) the energy absorbed as heat and (b) the change in entropy of a 2.00 kg block of copper whose temperature is increased reversibly from 25.0\unit{\celsius} to 100\unit{\celsius}. 
        The specific heat of copper is 386 \unit{\joule/\kilo\gram\cdot\kelvin}.

        \subsection{Solution (a)}
            First we can calculate the change in temperature.
            \begin{equation}
                \Delta T    =   100\unit{\celsius} - 25\unit{\celsius}
                    =   75\unit{\celsiusdegree}
                    =   75\unit{\kelvin}
            \end{equation}

            Heat absorbed is calculatable from there from the change in temperature.
            \begin{equation}
                Q   =   cm\Delta T
                    =   386\,\unit{\frac{\joule}{\kilo\gram\cdot\kelvin}} * 2\,\unit{\kilo\gram} * 75\unit{\kelvin}
                    =   \boxed{57900\,\unit{\joule}}
            \end{equation}

        \subsection{Solution (b)}
        The change in entropy is calculatable by an integral.
        \begin{align}
            \Delta S    &=  \int_{i}^{f}\frac{1}{T}\,dQ
        \end{align}

        We have our known equation of the heat, which we can adapt for use differentially.
        \begin{gather}
            Q   =   cm\Delta T\\
            dQ  =   cm\,dT
        \end{gather}

        This can be put into the integral above.
        \begin{align}
            \Delta S    &=  \int_{25\unit{\kelvin}}^{75\unit{\kelvin}}\frac{cm}{T}\,dT
                =   cm \int_{25\unit{\celsius}}^{75\unit{\celsius}}\frac{1}{T}\,dT
                =   cm\ln\left( \frac{100\unit{\celsius}}{25\unit{\celsius}} \right)\\
                &=  386\,\unit{\frac{\joule}{\kilo\gram\cdot\kelvin}} * 2\,\unit{\kilo\gram} * \ln\left( \frac{373\unit{\kelvin}}{298\unit{\kelvin}} \right)\\
                &=  \boxed{173.3\,\unit{\joule/\kelvin}}
        \end{align}

    \pagebreak
    \section{Problem 7}
        A 50.0 g block of copper whose temperature is 400 K is placed in an insulating box with a 100 g block of lead whose temperature is 200 K. 
        (a) What is the equilibrium temperature of the two-block system? 
        (b) What is the change in the internal energy of the system between the initial state and the equilibrium state? 
        (c) What is the change in the entropy of the system? (See Table 18-3.)

        \subsection{Solution (a)}
            There's an equivalence statement 

    \pagebreak
    \section{Problem 9}
        A 10 g ice cube at -10\unit{\celsius} is placed in a lake whose temperature is 15°C. 
        Calculate the change in entropy of the cube-lake system as the ice cube comes to thermal equilibrium with the lake. 
        The specific heat of ice is 2220 \unit{\joule/\kilo\gram\cdot\kelvin}. 
        (Hint: Will the ice cube affect the lake temperature?)

        \subsection{Solution}
            We can assume that the ice will only negligibly affect the lake's temperature.
            From this, we can devise the formula for the energy used to melt and warm up the ice.
            \begin{align}
                Q   &=  c_{\rm ice} m \Delta T_1 + L_{\rm fus} m + c_{\rm water} m \Delta T_2\\
                    &=  2220 * 0.010 * (10\,\unit{\kelvin}) + 334 * 10 + 4186 * 0.010 * (15\,\unit{\kelvin})\\
                    &=  222 + 3340 + 627.9
                    =   4189.9\,\unit{\joule}
            \end{align}

            This is the total heat required for the entire block of ice to melt and warm.
            We can turn this differentiable, with respect to the temperature when warming and with respect to the mass when melting.
            \begin{align}
                dQ  &=  c_{\rm ice} m\,dT + L_{\rm fus}\,dm + c_{\rm water} m\,dT
            \end{align}

            We can from here find the change in entropy by integral.
            The heat from fusion will be from 0g to 10g rather than from 10g to 0g because it is measuring the mass converted to water rather than the mass that is ice.
            \begin{align}
                \Delta S    &=  \int_{i}^{f}\frac{1}{T}\,dQ
                    =   \int_{263\unit{\kelvin}}^{273\unit{\kelvin}} \frac{c_{\rm ice} m}{T}\,dT
                        +   \int_{0\unit{\gram}}^{10\unit{\gram}} L_{\rm fus}\,dm
                        +   \int_{273\unit{\kelvin}}^{288\unit{\kelvin}} \frac{c_{\rm water} m}{T}\,dT\\
                    &=  c_{\rm ice} m\,\left[ \ln(T) \right]_{263\unit{\kelvin}}^{273\unit{\kelvin}} + L_{\rm fus}\,\left[ m \right]_{0\unit{\gram}}^{10\unit{\gram}} + c_{\rm ice} m\,\left[ \ln(T) \right]_{273\unit{\kelvin}}^{288\unit{\kelvin}}\\
                    &=  2220 * 0.010 * \ln\left( \frac{273}{263} \right) + \frac{334 * 10}{273} + 4186 * 0.010 * \ln\left( \frac{288}{273} \right)\\
                    &=  0.828\,\unit{\joule/\kelvin} + 12.234\,\unit{\joule/\kelvin} + 2.239\,\unit{\joule/\kelvin}
                    =   15.302\,\unit{\joule/\kelvin}
            \end{align}

            The lake also releases heat, which would itself give a change in entropy. 
            Since the lake's temperature is nearly constant, the process is isothermal.
            \begin{align}
                \Delta S    &=  \frac{Q}{T}
                    =   \frac{-4189.9\,\unit{\joule}}{288}
                    =   -14.548\,\unit{\joule/\kelvin}
            \end{align}

            Add the two changes in entropy together to find the ultimate change in entropy.
            \begin{align}
                \Delta S_{\Sigma}   &=  15.302\,\unit{\joule/\kelvin} - 14.548\unit{\joule/\kelvin}
                    =   \boxed{0.754\,\unit{\joule/\kelvin}}
            \end{align}

    \pagebreak
    \section{Problem 15}
        A mixture of 1773 g of water and 227 g of ice is in an initial equilibrium state at 0.000\unit{\celsius}. 
        The mixture is then, in a reversible process, brought to a second equilibrium state where the water-ice ratio, by mass, is 1.00 : 1.00 at 0.000\unit{\celsius}. 
        (a) Calculate the entropy change of the system during this process. 
        (The heat of fusion for water is 333 kJ/kg.) 
        (b) The system is then returned to the initial equilibrium state in an irreversible process (say, by using a Bunsen burner). 
        Calculate the entropy change of the system during this process. 
        (c) Are your answers consistent with the second law of thermodynamics?

        \subsection{Solution (a)}
            Mass is constant, so the total mass of the water and ice will be equal, itself the average of the total massses of ice and water.
            \begin{align}
                m_f &=  \frac{1773\,\unit{\gram} + 227\,\unit{\gram}}{2}
                    =   \frac{2000}{2}\,\unit{\gram}
                    =   1\,\unit{\kilo\gram}
            \end{align}

            That's a convenient value to use.
            We have an equation of the energy (heat in this case) used in fusion. 
            There is more water than ice here, so the energy would be causing water would be turning into ice.
            \begin{align}
                \Delta m    &=  m_f - m_i
                    =   1.000\,\unit{\kilo\gram} - 1.773\,\unit{\kilo\gram}
                    =   -0.773\,\unit{\kilo\gram}\\
                Q   &=  L_V m
                    =   333\,\unit{\joule/\gram} * (-773\,\unit{\gram})
                    =   -257409\,\unit{\joule}
            \end{align}

            This can in turn be used in the calculation of change of entropy.
            \begin{align}
                \Delta S    &=  \frac{Q}{T}
                    =   \frac{-257409\,\unit{\joule}}{273\unit{\kelvin}}
                    =   \boxed{-942\,\unit{\joule/\kelvin}}
            \end{align}

            The second law of thermodynamics seems to contradict this, with $\Delta S < 0$ being impossible.
            However, that only applies to closed systems, while this is not a closed system.

        \subsection{Solution (b)}
            The process would be of equal heat and opposite heat to the reversible process.
            Since the temperature would remain constant, the change in entropy would be equal and opposite as well.
            \begin{equation}
                \Delta S    =   \boxed{942\,\unit{\joule/\kelvin}}
            \end{equation}

        \subsection{Solution (c)}
            This would not be inconsistent with the second law of thermodynamics, so the answer is \boxed{yes}.
            As stated before, this would be an open system, or at least there is nothing saying it is not.

    \pagebreak
    \section{Problem 23}

        \subsection{Solution}

    \pagebreak
    \section{Problem 25}

        \subsection{Solution}

    \pagebreak
    \section{Problem 27}

        \subsection{Solution}

    \pagebreak
    \section{Problem 29}

        \subsection{Solution}

    \pagebreak
    \section{Problem 37}

        \subsection{Solution}

    \pagebreak
    \section{Problem 39}

        \subsection{Solution}

    \pagebreak
    \section{Problem 43}

        \subsection{Solution}

    \pagebreak
    \section{Problem 45}

        \subsection{Solution}

    \pagebreak
    \section{Problem 49}

        \subsection{Solution}

    \pagebreak
    \section{Problem 53}

        \subsection{Solution}

    \pagebreak
    \section{Problem 57}

        \subsection{Solution}

    \pagebreak
    \section{Problem 67}

        \subsection{Solution}

    \pagebreak
    \section{Problem 71}

        \subsection{Solution}

    \pagebreak

    \tableofcontents

\end{document}