\documentclass[12pt]{article}
\usepackage{amsmath}
\usepackage{amssymb}
\usepackage{cancel}
\usepackage{enumitem}
\usepackage{esdiff}
\usepackage{graphicx}
\usepackage{siunitx}
% \usepackage{pgfplots}
\usepackage{wrapfig}

\newcommand{\e}[1]{e^{i(#1)}}
\newcommand{\E}[1]{\times 10^{#1}}

\title{
    Worksheet \#12
    \\  \small
    PHYS 4C: Waves and Thermodynamics
    }
\author{Donald Aingworth IV}
\date{November 17, 2025}

\begin{document}
    \DeclareSIUnit{\celsiusdegree}{C^\circ}
    \DeclareSIUnit{\atm}{ atm}

    \maketitle

    \setcounter{section}{0}
    \section{Problem 1}
        Without looking anything up, determine what a light year is in meters.

        \subsection{Solution}
            I assume we know the speed of light to be $c = 2.998\E{8}\,\unit{\meter/\second}$.
            We first find the conversion factor of seconds per year.
            \begin{equation}
                1\text{ s} \times \frac{1\text{ min}}{60\text{ s}} \times \frac{1\text{ hr}}{60\text{ min}} \times \frac{1\text{ day}}{24\text{ hr}} \times \frac{1\text{ yr}}{365.25\text{ day}} = \frac{1\text{ yr}}{31 557 600}
            \end{equation}

            Multiply the speed of light by the reciprocal of the conversion factor to find the distance light travels in a year.
            \begin{equation}
                c   =   2.998\E{8}\,\unit{\meter/\second} \times 31 557 600\text{ s/yr}
                    =   9.46\E{15}\text{ m/yr}
            \end{equation}

            Multiply this by 1 year to find that one lightyear is equal to \boxed{9.46\E{15}\,\unit{\meter}}

    \section{Problem 2}
        Unpolarized light travelling in the z-direction is incident upon three polaroid filters with pass directions given by $+35\unit{\degree}$, $-40\unit{\degree}$, and $+25\unit{\degree}$ counter-clockwise from $+x$. 
        Determine the fraction of the original light which passes through the three filters.

        \subsection{Solution}
            Divide it into four parts with four intensities for the light: initially ($I_0$), then after passing through the first ($I_1$), second ($I_2$), and third ($I_3$) filters.
            First, the the first filter will filter out half the light's intensity.
            \begin{equation}
                I_1 = \frac{1}{2} I_0
            \end{equation}

            Next, the second filter is $35\unit{\degree} - (-40\unit{\degree}) = 75\unit{\degree}$ away from the first polarization direction. 
            This can be used to find $I_2$.
            \begin{equation}
                I_2 = I_1 \cos^2(75\unit{\degree}) = \frac{1}{2} \cos^2(75\unit{\degree}) I_0
            \end{equation}

            Last, the third filter is $-40\unit{\degree} - 25\unit{\degree} = -65\unit{\degree}$ away from the first polarization direction. 
            This can be used to find $I_3$.
            \begin{align}
                I_3 &=  I_2 \cos^2(-65\unit{\degree})
                    =   \frac{1}{2} \cos^2(-65\unit{\degree}) \cos^2(75\unit{\degree}) I_0\\
                    &=  0.00598 I_0
            \end{align}

            This tells is that the amount that passes through is \boxed{0.00598}.

    \section{Problem 3}
        (10 points) An electromagnetic wave has an electric field amplitude with a magnitude of $E_m = 12 \unit{\volt/\meter}$.
        \begin{enumerate}[label=\alph*)]
            \item   (1 point) Calculate the magnetic field amplitude.
            \item   (3 points) Calculate the (time averaged) energy density, intensity, momentum density (magnitude), and momentum current density of this wave.
            \item   (3 points) If this light is normally incident upon a $1.0\,\unit{\meter^2}$ surface, determine the rate of energy absorption and force acting on this surface. Answer this question for both a perfectly reflecting surface and a perfectly absorbing surface.
            \item   (3 points) Assume that this electromagnetic wave is located at the point (1000 m, 0, 0) and is generated by a dipole antenna located at (0, 0, 0) and oriented along the z-axis. If the oscillation frequency is 106 Hz, determine the electric dipole amplitude ($qz_m$) of this antenna.
        \end{enumerate}

        \subsection{Solution (a)}
            The magnitude of the magnetic field is calculatable from the magnitude of the electric field and the speed of light in a vacuum.
            \begin{gather}
                \frac{E_m}{M_m} = c \to M_m = \frac{E_m}{c} = \frac{12\,\unit{\volt/\meter}}{2.998\,\unit{\meter/\second}} = \boxed{4.003\E{-8}\,\unit{\tesla}}
            \end{gather}

        \subsection{Solution (b)}
            Energy density first.
            \begin{align}
                \left\langle u \right\rangle    &=  \frac{1}{2}\varepsilon_0 E_m^2
                    =   \frac{1}{2} * 8.854\E{-12}\,\unit{\farad/\meter} (12\,\unit{\volt/\meter})^2\\
                    &=  6.327\E{-10}\,\unit{\joule/\meter^3}
            \end{align}

            Next the intensity.
            \begin{align}
                I   &=  \frac{1}{2}c\varepsilon_0 E_m
                    =   c \left\langle u \right\rangle
                    =   0.191\,\unit{\watt/\meter^2}
            \end{align}

            Third the momentum density.
            \begin{align}
                g   &=  \frac{\left\langle u \right\rangle}{c}
                    =   \frac{6.327\E{-10}\,\unit{\joule/\meter^3}}{2.998\E{8}\,\unit{\meter/\second}}
            \end{align}
\end{document}