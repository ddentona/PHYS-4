\documentclass[12pt]{article}
\usepackage{amsmath}
\usepackage{array}
% \usepackage{gensymb}
\usepackage{geometry}
\usepackage{graphicx}
\usepackage{pgfplots}
\usepackage{siunitx}
\usepackage{wrapfig}

\title{Homework \#15}
\author{Donald Aingworth IV}
\date{December 4, 2024}

\pgfplotsset{width=8cm,compat=1.9}
\usepgfplotslibrary{external}
% \tikzexternalize

\begin{document}

\DeclareSIUnit{\mile}{mi}
\DeclareSIUnit{\gal}{gal}
\DeclareSIUnit{\foot}{ft}
\DeclareSIUnit{\hour}{h}
\DeclareSIUnit{\rad}{rad}
\DeclareSIUnit{\unit}{u}
\DeclareSIUnit{\dyne}{dyn}

\maketitle

\pagebreak
\section{Problem 1}
In unit-vector notation, what is the torque about the origin on a particle located at coordinates (0, -4.0 m, 3.0 m) if that torque is due to (a) force $\vec{F}_1$ with components $F_{1x}$ = 2.0 N, $F_{1y}$ = $F_{1z}$ = 0, and (b) force $\vec{F}_2$ with components $F_{2x}$ = 0, $F_{2y}$ = 2.0 N, $F_{2z}$ = 4.0 N?

\subsection{Solution}
In this instance, we can use the cross product for both parts. 
\subsubsection{Part (a)}
\begin{align}
    \tau    &=  \vec{r}\times\vec{F}
        =   \left(\begin{matrix} 0 \\ -4.0 \\ 3.0 \end{matrix}\right) \times \left(\begin{matrix} 2.0 \\ 0 \\ 0 \end{matrix}\right)
        =   \det\left|\begin{matrix}
            \hat{i}&\hat{j}&\hat{k}\\
            0   &   -4.0    &   3.0\\
            2.0 &   0       &   0
        \end{matrix}\right|\\
        &=  \begin{vmatrix} -4.0 & 3.0 \\ 0 & 0 \end{vmatrix}\hat{i} - 
            \begin{vmatrix} 0.0 & 3.0 \\ 2.0 & 0 \end{vmatrix}\hat{j} +
            \begin{vmatrix} 0.0 & -4.0 \\ 2.0 & 0 \end{vmatrix}\hat{k}\\
        &=  0\hat{i} + 6.0\hat{j} + 8.0\hat{k}
        =   \boxed{\begin{pmatrix}0\\6.0\\8.0\end{pmatrix}\unit{\newton*\meter}}
\end{align}

\subsubsection{Part (b)}
\begin{align}
    \tau    &=  \vec{r}\times\vec{F}
        =   \left(\begin{matrix} 0 \\ -4.0 \\ 3.0 \end{matrix}\right) \times \left(\begin{matrix} 0 \\ 2.0 \\ 4.0 \end{matrix}\right)
        =   \det\left|\begin{matrix}
            \hat{i}&\hat{j}&\hat{k}\\
            0   &   -4.0    &   3.0\\
            0   &   2.0     &   4.0
        \end{matrix}\right|\\
        &=  \begin{vmatrix} -4.0 & 3.0 \\ 2.0 & 4.0 \end{vmatrix}\hat{i} - 
            \begin{vmatrix} 0.0 & 3.0 \\ 0 & 4.0 \end{vmatrix}\hat{j} +
            \begin{vmatrix} 0.0 & -4.0 \\ 0 & 2.0 \end{vmatrix}\hat{k}\\
        &=  -22.0\hat{i} + 0\hat{j} + 0\hat{k}
        =   \boxed{\begin{pmatrix}-22.0\\0\\0\end{pmatrix}\unit{\newton*\meter}}
\end{align}

\pagebreak
\section{Problem 2}
At one instant, force $\vec{F} = 4.0 \hat{j}$ N acts on a 0.25 kg object that has position vector $r = (2.0\hat{i} - 2.0\hat{k})$ m and velocity vector $v = (-5.0\vec{i} + 5.0\vec{k})$ m/s. About the origin and in unit-vector notation, what are (a) the object's angular momentum and (b) the torque acting on the object?

\subsection{Solution}
\subsubsection{Section (a)}
The angular momentum is equivalent to \(\vec{\ell} = \vec{r}\times\vec{p}\).
\begin{align}
    \vec{\ell}  &=  \vec{r} \times \vec{p}
        =   \vec{r} \times m\vec{v}
        =   \begin{pmatrix}2\\0\\-2\end{pmatrix} \times 0.25\begin{pmatrix}-5\\0\\5\end{pmatrix}
        =   \begin{pmatrix}2\\0\\-2\end{pmatrix} \times \begin{pmatrix}-\frac{5}{4}\\0\\\frac{5}{4}\end{pmatrix}\\
    \vec{\ell}  &=  \det\begin{vmatrix}
                            \hat{i} &\hat{j}&\hat{k}\\
                            2       &0      &-2     \\
                            -\frac{5}{4}&0  &\frac{5}{4}
                        \end{vmatrix}
        =   \begin{vmatrix}0&-2\\0&\frac{5}{4}\end{vmatrix}\hat{i} -
            \begin{vmatrix}2&-2\\-\frac{5}{4}&\frac{5}{4}\end{vmatrix}\hat{j} + 
            \begin{vmatrix}2&0\\-\frac{5}{4}&0\end{vmatrix}\hat{k}\\
    \vec{\ell}  &=  0\hat{i} + 0\hat{j} + 0\hat{k} 
        = \boxed{\begin{pmatrix}0\\0\\0\end{pmatrix}\unit{\kilo\gram*\meter/\second}}
\end{align}
This means that the object is not moving around the origin at all, and is instead moving towards the origin. 

\subsubsection{Section (b)}
We can use the torque-force-moment-arm equation.
\begin{align}
    \tau    &=  -\vec{F} \times \vec{r}
        =   -\begin{pmatrix} 0 \\ 4.0 \\ 0 \end{pmatrix} \times \begin{pmatrix} 2.0 \\ 0 \\ -2.0 \end{pmatrix}
        =   -\det\begin{vmatrix}
            \hat{i} &\hat{j}&\hat{k}\\
            0       &4.0    &0      \\
            2.0     &0      &-2.0
        \end{vmatrix}\\
    \tau    &=  -\left(
                \begin{vmatrix}4.0&0\\0&-2.0\end{vmatrix}\hat{i} - 
                \begin{vmatrix}0&0\\2.0&-2.0\end{vmatrix}\hat{j} + 
                \begin{vmatrix}0&4.0\\2.0&0\end{vmatrix}\hat{k}
                \right)\\
    \tau    &=  -(-8.0\hat{i} - 0\hat{j} - 8.0\hat{k})
        =   -\begin{pmatrix}-8.0\\0\\-8.0\end{pmatrix}
        =   \boxed{\begin{pmatrix}8.0\\0\\8.0\end{pmatrix}\unit{\newton*\meter}}
\end{align}

\pagebreak
\section{Problem 3}
At the instant the displacement of a 2.00 kg object relative to the origin is \(d = (2.00 \unit{\meter})\hat{i} + (4.00 \unit{\meter})\hat{j} - (3.00 \unit{\meter})\hat{k}\), its velocity is \(v = -(6.00 \unit{\meter/\second})\hat{i} + (3.00 \unit{\meter/\second})\hat{j} + (3.00 \unit{\meter/\second})\hat{k}\) and it is subject to a force \(F = (6.00 \unit{\newton})\hat{i} - (8.00 \unit{\newton})\hat{j} + (4.00 \unit{\newton})\hat{k}\) . Find (a) the acceleration of the object, (b) the angular momentum of the object about the origin, (c) the torque about the origin acting on the object, and (d) the angle between the velocity of the object and the force acting on the object (See HW 3, problem 4).

\subsection{Solution}
\subsubsection{Section (a)}
We can use Newton's second law to find this.
\begin{align}
    \vec{a}_{com}   &=  \frac{\vec{F}_{com}}{m}
        =   \frac{(6.00 \unit{\newton})\hat{i} - (8.00 \unit{\newton})\hat{j} + (4.00 \unit{\newton})\hat{k}}{2.00\unit{\kilo\gram}}
        =   \boxed{\begin{pmatrix}3.00\\-4.00\\2.00\end{pmatrix}\unit{\meter/\second^2}}
\end{align}

\subsubsection{Section (b)}
The angular momentum is equivalent to \(\vec{\ell} = \vec{r}\times\vec{p}\).
\begin{align}
    \vec{\ell}  &=  \vec{r} \times \vec{p}
        =   \vec{r} \times m\vec{v}\\
    \vec{\ell}  &=  \begin{pmatrix}2.00\\4.00\\-3.00\end{pmatrix} \times 2.00\begin{pmatrix}-6.00\\3.00\\3.00\end{pmatrix}
        =   \begin{pmatrix}2.00\\4.00\\-3.00\end{pmatrix} \times \begin{pmatrix}-12.00\\6.00\\6.00\end{pmatrix}\\
    \vec{\ell}  &=  \det\begin{vmatrix}
                            \hat{i} &\hat{j}&\hat{k}\\
                            2.00    &4.00   &-3.00  \\
                            -12.00  &6.00   &6.00
                        \end{vmatrix}\\
    \vec{\ell}  &=  \begin{vmatrix}4.00&-3.00\\6.00&6.00\end{vmatrix}\hat{i} -
                    \begin{vmatrix}2.00&-3.00\\-12.00&6.00\end{vmatrix}\hat{j} + 
                    \begin{vmatrix}2.00&4.00\\-12.00&6.00\end{vmatrix}\hat{k}\\
    \vec{\ell}  &=  42.00\hat{i} + 24.00\hat{j} + 60.00\hat{k} 
        = \boxed{\begin{pmatrix}42\\24\\60\end{pmatrix}\unit{\kilo\gram*\meter/\second}}
\end{align}

\subsubsection{Section (c)}
We can use the torque-force-moment-arm equation.
\begin{align}
    \tau    &=  \vec{r} \times \vec{F}
        =   \begin{pmatrix}2.00\\4.00\\-3.00\end{pmatrix} \times \begin{pmatrix}6.00\\-8.00\\4.00\end{pmatrix}
        =   \det\begin{vmatrix}
            \hat{i} &\hat{j}&\hat{k}\\
            2.00    &4.00   &-3.00  \\
            6.00    &-8.00  &4.00
        \end{vmatrix}\\
    \tau    &=  \begin{vmatrix}4.00&-3.00\\-8.00&4.00\end{vmatrix}\hat{i} -
                \begin{vmatrix}2.00&-3.00\\6.00&4.00\end{vmatrix} \hat{j} +
                \begin{vmatrix}2.00&4.00\\6.00&-8.00\end{vmatrix} \hat{k}\\
    \tau    &=  -8.00\hat{i} + -26.00\hat{j} + -40.00\hat{k}
        =   \boxed{\begin{pmatrix}-8.00\\-26.00\\-40.00\end{pmatrix}\unit{\newton*\meter}}
\end{align}

\subsubsection{Section (d)}
We have two vector equations that we can use for finding an angle, one for the cross product \(\left|\vec{a}\times\vec{b}\right| = ab\sin(\phi)\), and one for the dot product \(\vec{a}\cdot\vec{b} = ab\cos(\phi)\). Since cosine is a continuous and inversible function between 0 and $\pi$, we will be using the dot product.
\begin{align}
    \vec{a}\cdot\vec{b} &=  ab\cos(\phi)\\
    \cos(\phi)  &=  \frac{\vec{a}\cdot\vec{b}}{ab}
        =   \frac{\vec{F}\cdot\vec{v}}{Fv}\\
    F   &=  \left|\vec{F}\right|
        =   \sqrt{6.0^2 + (-8.0)^2 + 4.0^2}\\
        &=  \sqrt{36 + 64 + 16}
        =   \sqrt{116}
        =   2\sqrt{29}\unit{\newton}\\
    v   &=  \left|\vec{v}\right|
        =   \sqrt{(-6.0)^2 + 3.0^2 + 3.0^2}\\
        &=  \sqrt{36 + 9 + 9}
        =   \sqrt{54}
        =   3\sqrt{5}\unit{\meter/\second}\\
    \vec{F}\cdot\vec{v} &=  6.0*(-6.0) + (-8.0)*3.0 + 4.0*3.0\\
        &=  - 36 - 24 + 12
        =   -48\unit{\newton*\meter/\second}\\
    \cos(\phi)  &=  \frac{\vec{F}\cdot\vec{v}}{Fv}
        =   -\frac{48\unit{\newton*\meter/\second}}{15\sqrt{29*5}\unit{\newton*\meter/\second}}\\
    \phi    &=  \arccos\left(-\frac{48}{15\sqrt{29*5}}\right)
        \approx \boxed{105.4\unit{\degree}}
\end{align}

\pagebreak
\section{Problem 4}
The angular momentum of a flywheel having a rotational inertia of 0.140 \unit{\kilo\gram*\meter^2} about its central axis decreases from 3.00 to 0.800 \unit{\kilo\gram*\meter^2/\second} in 1.50 \unit{\second}. (a) What is the magnitude of the average torque acting on the flywheel about its central axis during this period? (b) Assuming a constant angular acceleration, through what angle does the flywheel turn? (c) How much work is done on the wheel? (d) What is the average power of the flywheel?

\subsection{Solution}
\subsubsection{Section (a)}
We can find the average torque with the inertia and the average acceleration.
\begin{align}
    \tau_{net}  &=  \frac{d\ell}{dt}
        =   \frac{0.800 - 3.00 \unit{\kilo\gram*\meter^2/\second}}{1.50 \unit{\second}}
        =   -\frac{2.20}{1.50}\unit{\newton*\meter}
        =   \boxed{-1.467\unit{\newton*\meter}}
\end{align}

\subsubsection{Section (b)}
Assuming the listed angular momentum is the total angular momentum of the flywheel, we have the formula for the angular velocity and a formula for the angular acceleration from the torque. We can then apply those to a kinematic equation.
\begin{align}
    &L = I\omega    &\tau_{net} = I\alpha\\
    &\omega^2 = \omega_0^2 + 2\alpha\Delta \theta
\end{align}
\begin{gather}
    \omega_f    =   \frac{L_f}{I}\ \&\&\ 
    \omega_i    =   \frac{L_f}{I}\ \&\&\ 
    \alpha  =   \frac{\tau_{net}}{I}
\end{gather}
\begin{align}
    \Delta\theta    &=  \frac{\omega_f^2 - \omega_i^2}{2\alpha}
        =   \frac{\frac{L_f}{I}^2 - \frac{L_i}{I}^2}{2\frac{\tau_{net}}{I}}
        =   \frac{L_f^2 - L_i^2}{2\tau_{net}I}
        =   \frac{0.800^2 - 3.00^2}{2*(-\frac{22}{15})*0.140}\\
        &=  \frac{900 - 64}{\frac{22}{15}*28}
        =   \frac{836*15}{22*28}
        =   \frac{6270}{308}
        \approx \boxed{20.357}
\end{align}
Note: this is the radial measurement, not the angular measurement.

\subsubsection{Section (c)}
The work done here is equal to the change in kinetic energy. We have the angular velocity and moment of inertia. 
\begin{align}
    W   &=  \Delta K
        =   K_f - K_i
        =   \frac{1}{2}I\omega_f^2 - \frac{1}{2}I\omega_i^2
        =   \frac{I}{2}(\omega_f^2 - \omega_i^2)
        =   \frac{1}{2I}(L_f - L_i)(L_f + L_i)\\
        &=  \frac{1}{0.28}(0.80 - 3.0)(0.80 + 3.0)
        =   \frac{(-2.2)*3.8}{0.28}
        =   \boxed{-29.857\unit{\joule}}
\end{align}

\subsubsection{Section (d)}
The average power is equivalent to the change in energy divided by the change in time. 
\begin{align}
    \omega_f    &=  \omega_i + \alpha t
        \rightarrow t   =   \frac{\Delta \omega}{\alpha} = \frac{\Delta L}{I\alpha} = \frac{\Delta L}{\tau_{net}}\\
    P   &=  \frac{\Delta K}{\Delta t}
        =   \frac{W\tau_{net}}{\Delta L}
        =   \frac{\frac{(-2.2)*3.8}{0.28}*\frac{-2.2}{1.5}}{0.80 - 3.0}\\
        &=  \boxed{-19.905\unit{\joule*\second}}
\end{align}

\pagebreak
\section{Problem 5}
Force $\vec{F} = (-8.00\unit{\newton})\hat{i} + (6.00\unit{\newton})\hat{j}$ acts on a particle with position vector $\vec{r} = (3.00\unit{\meter})\hat{i} + (4.00\unit{\meter})\hat{j}$. What are (a) the torque on the particle about the origin, in unit-vector notation, and (b) the angle between the directions of $\vec{r}$ and $\vec{F}$?

\subsection{Solution}
\subsubsection{Section (a)}
\begin{align}
    \vec{\tau}  &=  \vec{r}\times\vec{F}
        =   \begin{pmatrix}3.00\\4.00\\0\end{pmatrix} \times \begin{pmatrix}-8.00\\6.00\\0\end{pmatrix}
        =   \begin{vmatrix}
                \hat{i}&\hat{j}&\hat{k}\\
                3.0&4.0&0\\
                -8.0&6.0&0
            \end{vmatrix}\\
        &=  \begin{vmatrix}3.0&4.0\\-8.0&6.0\end{vmatrix}\hat{k}
        =   (3*6 - (-8)*4)\hat{k}
        =   (18 + 32)\hat{k}\\
        &=  \boxed{50\hat{k}\ \unit{\newton*\meter}}
\end{align}

\subsubsection{Section (b)}
To save time and space, \(||\vec{F}|| = 10\unit{\newton}\), \(||\vec{r}|| = 5\unit{\newton}\), and \(||\vec{\tau}|| = 50\unit{\newton}\). See 3-4-5 triangles for my reasoning why.
\begin{align}
    \left|\vec{a}\times\vec{b}\right| &= ab\sin(\phi)\\
    \sin(\phi)  &=  \frac{\left|\vec{a}\times\vec{b}\right|}{ab}
        =   \frac{50}{5*10}
        =   \frac{50}{50}
        =   1\\
    \phi    &= \arcsin(1)
        =   \boxed{\frac{\pi}{2} = 90\unit{\degree}} 
\end{align}

\pagebreak
\section{Problem 6}
At time t, the vector $\vec{r} = 4.0t^2\hat{i} - \left(2.0t + 6.0t^2\right)\hat{j}$ gives the position of a 3.0 kg particle relative to the origin of an xy coordinate system ($\vec{r}$ is in meters and t is in seconds). (a) Find an expression for the torque acting on the particle relative to the origin. (b) Is the magnitude of the particle's angular momentum relative to the origin increasing, decreasing, or unchanging?

\subsection{Solution}
\subsubsection{Section (a)}
First, we take the second derivative of the radius to find the acceleration. Then, we multiply that by the mass to find the force. 
\begin{align}
    \vec{r} &=  4.0t^2\hat{i} - \left(2.0t + 6.0t^2\right)\hat{j}\\
    \vec{r}'    &=  \vec{v}
        =   8.0t\hat{i} - \left(2 + 12t\right)\hat{j} \unit{\meter/\second}\\
    \vec{r}''   &=  \vec{a}
        =   8.0\hat{i} - 12\hat{j} \unit{\meter/\second^2}\\
    \vec{F} &=  m*\vec{a}
        =   3.0*(8.0\hat{i} - 12\hat{j})
        =   24.0\hat{i} - 36.0\hat{j} \unit{\newton}
\end{align}

Next, we take the cross product of the radus and the force to get the torque.
\begin{align}
    \vec{\tau}  &=  \vec{r}\times\vec{F}
        =   \left(4.0t^2\hat{i} - \left(2.0t + 6.0t^2\right)\hat{j}\right) \times \left(24.0\hat{i} - 36.0\hat{j}\right)\\
        &=  \begin{vmatrix}
                \hat{i}&\hat{j}&\hat{k}\\
                4.0t^2&-2.0t - 6.0t^2&0\\
                24.0&-36.0&0
            \end{vmatrix}
        =   \begin{vmatrix}4.0t^2&-2.0t - 6.0t^2\\24.0&-36.0\end{vmatrix}\hat{k}\\
        &=  (4.0t^2*(-36.0) - 24.0*(-2.0t - 6.0t^2))\hat{k}\\
        &=  (-144t^2 + 48t + 144t^2)\hat{k}
        =   \boxed{(48t)\hat{k}\unit{\newton*\meter}}
\end{align}

\subsubsection{Section (b)}
Since the torque is equivalent to the rate of change of the angular momentum, it determines the answer to this. We can see that it is negative for $t<0$ but positive for $t>0$. As such, as long as $t > 0$, the magnitude of the angular momentum increases.


\pagebreak
\section{Problem 7}
A rigid body rotates with constant angular velocity about a fixed axis. Show that its kinetic energy K and angular momentum L are related according to $K = \frac{L^2}{2I}$, where I is the rotational inertia.

\subsection{Solution}
I bet I can do it in less than seven steps.
\begin{gather}
    L   =   I\omega\\
    \omega  =   \frac{L}{I}\\
    K   =   \frac{1}{2}I\omega^2\\
    K   =   \frac{I}{2}*\frac{L^2}{I^2}\\
    K   =   \frac{IL^2}{2I^2}\\
    \boxed{K   =   \frac{L^2}{2I}}
\end{gather}

\end{document}