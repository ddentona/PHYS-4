\documentclass[8pt]{minimal}
\usepackage{amsmath}

\begin{document}
\setlength{\parindent}{0pt}
% \setlength{\voffset}{-50pt}
\twocolumn
PHYS 4A Exam 2 Cheat Sheet

\underline{Write Units}

\underline{Kinematic Equations}
\begin{gather*}
    v_{avg}=\frac{\Delta x}{\Delta t}; s_{avg}=\frac{distance}{time};v=\frac{dx}{dt}\\
    a_{avg}=\frac{\Delta v}{\Delta t}; a = \frac{dv}{dt} = \frac{d^2x}{dt^2}; (1)\ v(t) = v_0 + at\\
    (2)\ x = x_0 + v_0t + \frac{1}{2}at^2; (3)\ v^2 = v_0^2 + 2a\Delta x
\end{gather*}
When doing a problem, account for all the variables you know the values of and all those you don't know the value of.\\

\underline{Freefall}

Object is in freefall iff only force acting on it is gravity\\
Kinematic eq'ns apply to freefall\\
Unless stated otherwise, gravitational acceleration $g=-9.81m/s^2$\\

\underline{Vectors}
\begin{gather*}
    \vec{a} \cdot \vec{b} = ab \cos(\theta); || \vec{a}\times \vec{b} || = ab \sin(\theta)\\
    \vec{a} \cdot \vec{b} = a_x b_x+a_y b_y\dots; \vec{a}\times \vec{b} = \det \left( \begin{smallmatrix} \hat{i} & \hat{j} &\hat{k} \\ a_x & a_y & a_z \\ b_x & b_y & b_z  \end{smallmatrix} \right)
\end{gather*}
Vectors work as their separate parts for kinematic eq'ns
\\

\underline{Project}

Motion in 2D+ (uses vectors)\\
Generally, vertical motion is freefall, horizontal motion is constant \\
x-value = magnitude times cosine of angle\\
y-value = magnitude times sine of angle\\
\begin{gather*}
    R=x-x_0=\frac{v_0^2*\sin(2\theta)}{g};t=\frac{R}{v_0  \cos(\theta)}\\
    \Delta y =\tan\theta \Delta x - \frac{g*\Delta x^2}{2(v_0  \cos\theta)^2 }
\end{gather*}

\underline{Uniform Circular Motion}
\begin{equation*}
    \vec{x}(t)=x*\cos\theta \hat{i} + x*\sin\theta \hat{j}  ;a=\frac{v^2}{r} ; F_c=\frac{mv^2}{r}
\end{equation*}

\underline{Force}

Force on an object is always represented on a FBD as starting from that object\\
Force on an object is calculated from that object's mass and consequent acceleration
\begin{equation*}
    F_{net}=ma      |      F_{AB}=-F_{BA}
\end{equation*}
There is no technical equation for the tension force. Treat it as an unknown when it is included.

\underline{Friction}
\begin{equation*}
    f_s \le \mu_s F_N ; f_k = \mu_k F_N
\end{equation*}

\underline{Spring force}
\begin{equation*}
    \vec{F}_s = -k\vec{d}
\end{equation*}
\end{document}