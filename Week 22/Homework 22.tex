\documentclass[12pt]{article}
\usepackage{amsmath}
\usepackage{array}
\usepackage[thinc]{esdiff}
% \usepackage{gensymb}
\usepackage{geometry}
\usepackage{graphicx}
\usepackage{pgfplots}
\usepackage{siunitx}
\usepackage{wrapfig}
\usepackage{xcolor}

\title{Homework \#4, 4B}
\author{Donald Aingworth IV}
\date{February 12, 2025}

\pgfplotsset{width=8cm,compat=1.9}
\usepgfplotslibrary{external}
% \tikzexternalize

\renewcommand\thesubsection{\alph{subsection}}

\begin{document}

\DeclareSIUnit{\mile}{mi}
\DeclareSIUnit{\gal}{gal}
\DeclareSIUnit{\foot}{ft}
\DeclareSIUnit{\hour}{h}
\DeclareSIUnit{\rad}{rad}
\DeclareSIUnit{\unit}{u}
\DeclareSIUnit{\dyne}{dyn}

\maketitle

\section{Question 1}
A surface that has the area vector $\vec{A} = \left(2\hat{i} + 3\hat{j}\right) \unit{\meter^2}$. What is the flux of a uniform electrc field that is (a) $\vec{E} = 4\hat{i} \unit{\newton/\coulomb}$ and (b) $\vec{E} = 4\hat{k} \unit{\newton/\coulomb}$?

\section{Question 3}

\pagebreak
\section{Problem 6}
Three infinite nonconducting sheets, with uniform positive surface charge densities $\sigma$, $2\sigma$, and $3\sigma$, are arranged to be parallel like the two sheets in \text{\color{blue} Fig. 23-19a}. What is their order, from left to right, if the electric field $\vec{E}$ produced by the arrangement has magnitude $E = 0$ in one region and $E = 2\sigma/\epsilon_0$ in another region?

\subsection*{Solution}
For an infinite nonconducting sheet of densty $\sigma$, the electric field from it is equal to $E = \sigma/2\epsilon_0$. We can use this to provide a system of equations for electric field strengths $(a, b, c)$, which have unique magnitudes in the set $(\sigma/2\epsilon_0, 2\sigma/2\epsilon_0, 3\sigma/2\epsilon_0)$ or alternatively $(E, 2E, 3E)$. 
\begin{align*}
    a - b - c &= 0\\
    a + b - c &= 2\sigma/\epsilon_0 = 4E\\
    0a + 2b + 0c &= 4E\\
    b &= 2E = 2\sigma/2\epsilon_0\\
    a - 2E - c &= 0 \rightarrow a - c = 2E
\end{align*}

There is only one combination of the remaining two that this works for: $a = 3E$ and $c = E$. Thus, the order is \boxed{\langle 3\sigma, 2\sigma, \sigma \rangle}.

\pagebreak
\section{Problem 8}

\section{Problem 10}

\section{Problem 12}

\section{Problem 18}

\section{Problem 22}

\section{Problem 28}

\section{Problem 34}

\end{document}