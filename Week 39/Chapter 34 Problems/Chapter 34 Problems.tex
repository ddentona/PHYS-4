\documentclass[12pt]{article}
\usepackage{amsmath}
\usepackage{amssymb}
\usepackage{cancel}
\usepackage{graphicx}
% \usepackage{physics}
\usepackage{siunitx}
\usepackage{wrapfig}

% \AtBeginDocument{\RenewCommandCopy\qty\SI}

\newcommand{\E}[1]{\times 10^{#1}}

\title{
    Chapter 34 End-of-Chapter Problems
    \\ \small
    Halliday \& Resnick, 10th Edition
}

\author{Donald Aingworth IV}

\date{\small Hit me where it Matters}

\begin{document}
    \DeclareSIUnit{\atm}{atm}
    \DeclareSIUnit{\cal}{\ cal}
    \DeclareSIUnit{\Cal}{\ Cal}
    \DeclareSIUnit{\calorie}{\ cal}
    \DeclareSIUnit{\Calorie}{\ Cal}
    \DeclareSIUnit{\celsiusdegree}{C^\circ}
    \DeclareSIUnit{\fahrenheit}{^\circ F}
    \DeclareSIUnit{\fahrenheitdegree}{F^\circ}
    \DeclareSIUnit{\torr}{\ torr}

    \maketitle

    \pagebreak
    \section{Problem 1}
        You look through a camera toward an image of a hummingbird in a plane mirror. 
        The camera is 4.30 m in front of the mirror. 
        The bird is at camera level, 5.00 m to your right and 3.30 m from the mirror. 
        What is the distance between the camera and the apparent position of the bird's image in the mirror?

        \subsection{Solution}
            In a plane mirror, the object will appear as far from the mirror as it is legitimately. 
            Add the distance from the mirror of the camera to the bird's distance to the mirror.
            \begin{equation}
                \Delta x = 4.30\,\unit{\meter} + 3.30\,\unit{\meter} = 7.60\,\unit{\meter}
            \end{equation}

            Here use the Pythagorean theorem to find the distance.
            \begin{align}
                s   &=  \sqrt{\Delta x^2 + \Delta y^2}
                    =   \sqrt{7.60^2 + 5.00^2}
                    =   \sqrt{57.76 + 25.00}
                    =   \sqrt{82.76}
                    =   \boxed{9.097\,\unit{\meter}}
            \end{align}

    \pagebreak
    \section{Problem 3}
        \begin{wrapfigure}{r}{0.15\textwidth}
            \vspace{-30pt}
            \includegraphics[width=0.15\textwidth]{34-32.png} 
            % \label{fig:wrapfig}
        \end{wrapfigure}
        In Fig. 34-32, an isotropic point source of light $S$ is positioned at distance $d$ from a viewing screen $A$ and the light intensity $I_P$ at point $P$ (level with $S$) is measured. 
        Then a plane mirror $M$ is placed behind $S$ at distance $d$.
        By how much is $I_P$ multiplied by the presence of the mirror?

        \subsection{Solution}
            The equation for the intensity of light.
            \begin{equation}
                I = \frac{P_s}{4\pi r^2}
            \end{equation}

            This gives us an equation for $I_P$.
            \begin{equation}
                I_P = \frac{P_s}{4\pi d^2}
            \end{equation}

            Using this, we can create an equation for the intensity at point $P$ from teh light that reflected from the mirror (call it $I_M$).
            \begin{equation}
                I_M = \frac{P_s}{4\pi (3d)^2}
            \end{equation}

            We can use these to calculate the total intensity and relate that to $I_P$.
            \begin{align}
                I_{net} &=  I_P + I_M
                    =   \frac{P_s}{4\pi d^2} + \frac{P_s}{4\pi (3d)^2}\\
                    &=  \frac{P_s}{4\pi d^2} \left( 1 + \frac{1}{3^2} \right)
                    =   I_P \left( \frac{10}{9} \right)\\
                \frac{I_{net}}{I_P} &=  \boxed{\frac{10}{9} = 1.11}
            \end{align}

    \pagebreak
    \section{Problem 7}
        A concave shaving mirror has a radius of curvature of 35.0 cm.
        It is positioned so that the (upright) image of a man's face is 2.50 times the size of the face. 
        How far is the mirror from the face?

        \subsection{Solution}
            The upright image size difference tells us the magnification and solve for the position of the image ($i$).
            \begin{gather}
                m = -\frac{i}{p} = 2.50\\
                p = -\frac{i}{2.50}
            \end{gather}

            Usng this and the focal point, we can find a value of $p$.
            \begin{gather}
                \frac{}{}
            \end{gather}

    \pagebreak
    \section{Problem 9-16}
        Object O
stands on the central axis of a spherical mirror. For this situation,
each problem in Table 34-3 gives object distance p s (centimeters),
the type of mirror, and then the distance (centimeters, without
proper sign) between the focal point and the mirror. Find (a) the
radius of curvature r (including sign), (b) the image distance i, and
(c) the lateral magnification m. Also, determine whether the image
is (d) real (R) or virtual (V), (e) inverted (I) from object O or non-
inverted (NI), and (f) on the same side of the mirror as O or on the
opposite side.
        \begin{center}
            \begin{tabular}{r c l}
                    &   $p$ &   Mirror\\
                \hline
                9   &   +18 &   Concave, 12\\
                10  &   +15 &   Concave, 10\\
                11  &   +8  &   Convex, 10\\
                12  &   +24 &   Concave, 36\\
                13  &   +12 &   Concave, 18\\
                14  &   +22 &   Convex, 35\\
                15  &   +10 &   Convex, 8\\
                16  &   +17 &   Convex, 14
            \end{tabular}
        \end{center}

        \subsection{Solution}

    \pagebreak
    \section{Problem 17-29}
        Object O
stands on the central axis of a spherical or plane mirror. For this
situation, each problem in Table 34-4 refers to (a) the type of
mirror, (b) the focal distance f, (c) the radius of curvature r, (d)
the object distance p, (e) the image distance i, and (f) the lateral
magnification m. (All distances are
in centimeters.) It also refers to
whether (g) the image is real (R) or
virtual (V), (h) inverted (I) or non-
inverted (NI) from O, and (i) on the
same side of the mirror as object O
or on the opposite side. Fill in the
missing information. Where only a
sign is missing, answer with the sign.
        \begin{center}
            \includegraphics[width=\textwidth]{T34-4.png}
        \end{center}

        \subsection{Solution}

    \pagebreak
    \section{Problem 32-38}
        An object O stands on the central axis of a spherical refract-
ing surface. For this situation, each problem in Table 34-5 refers to
the index of refraction n1 where the object is located, (a) the index
of refraction n2 on the other side of the refracting surface, (b) the
object distance p, (c) the radius of curvature r of the surface, and
(d) the image distance i. (All distances are in centimeters.) Fill in
the missing information, including whether the image is (e) real
(R) or virtual (V) and (f) on the same side of the surface as object
O or on the opposite side.
        \begin{center}
            \includegraphics[width=\textwidth]{T34-5.png}
        \end{center}

        \subsection{Solution}

    \pagebreak
    \section{Problem 39}
        \begin{wrapfigure}{r}{0.25\textwidth}
            \vspace{-30pt}
            \includegraphics[width=0.25\textwidth]{34-38.png} 
            % \label{fig:wrapfig}
        \end{wrapfigure}
        In Fig. 34-38, a beam of parallel light rays from a laser is
incident on a solid transparent sphere of index of refraction n.
(a) If a point image is produced at
the back of the sphere, what is the
index of refraction of the sphere?
(b) What index of refraction, if any,
will produce a point image at the
center of the sphere?

        \subsection{Solution}

    \pagebreak
    \section{Problem 41}
        A lens is made of glass having an index of refraction of
1.5. One side of the lens is flat, and the other is convex with a
radius of curvature of 20 cm. (a) Find the focal length of the
lens. (b) If an object is placed 40 cm in front of the lens, where
is the image?

        \subsection{Solution}

    \pagebreak
    \section{Problem 45}
        You produce an image of the
Sun on a screen, using a thin lens whose focal length is 20.0 cm.
What is the diameter of the image? (See Appendix C for needed
data on the Sun.)

        \subsection{Solution}

    \pagebreak
    \section{Problem 49}
        An illuminated slide is held 44 cm from a screen. How
far from the slide must a lens of focal length 11 cm be placed
(between the slide and the screen) to form an image of the slide's
picture on the screen?

        \subsection{Solution}

    \pagebreak
    \section{Problem 53}

        \subsection{Solution}

    \pagebreak
    \section{Problem 57}

        \subsection{Solution}

    \pagebreak
    \section{Problem 63}

        \subsection{Solution}

    \pagebreak
    \section{Problem 69}

        \subsection{Solution}

    \pagebreak
    \section{Problem 73}

        \subsection{Solution}

    \pagebreak
    \section{Problem 75}

        \subsection{Solution}

    \pagebreak
    \section{Problem 81}

        \subsection{Solution}

    \pagebreak
    \section{Problem 83}

        \subsection{Solution}

    \pagebreak
    \section{Problem 85}

        \subsection{Solution}

    \pagebreak
    \section{Problem 87}

        \subsection{Solution}

    \pagebreak
    \section{Problem 99}

        \subsection{Solution}

    \pagebreak
    \section{Problem 105}

        \subsection{Solution}

    \pagebreak
    \section{Problem 109}

        \subsection{Solution}

    \pagebreak
    \tableofcontents
\end{document}